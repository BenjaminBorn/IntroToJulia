
% Default to the notebook output style

    


% Inherit from the specified cell style.




    
\documentclass[11pt]{article}

    
    
    \usepackage[T1]{fontenc}
    % Nicer default font than Computer Modern for most use cases
    \usepackage{palatino}

    % Basic figure setup, for now with no caption control since it's done
    % automatically by Pandoc (which extracts ![](path) syntax from Markdown).
    \usepackage{graphicx}
    % We will generate all images so they have a width \maxwidth. This means
    % that they will get their normal width if they fit onto the page, but
    % are scaled down if they would overflow the margins.
    \makeatletter
    \def\maxwidth{\ifdim\Gin@nat@width>\linewidth\linewidth
    \else\Gin@nat@width\fi}
    \makeatother
    \let\Oldincludegraphics\includegraphics
    % Set max figure width to be 80% of text width, for now hardcoded.
    \renewcommand{\includegraphics}[1]{\Oldincludegraphics[width=.8\maxwidth]{#1}}
    % Ensure that by default, figures have no caption (until we provide a
    % proper Figure object with a Caption API and a way to capture that
    % in the conversion process - todo).
    \usepackage{caption}
    \DeclareCaptionLabelFormat{nolabel}{}
    \captionsetup{labelformat=nolabel}

    \usepackage{adjustbox} % Used to constrain images to a maximum size 
    \usepackage{xcolor} % Allow colors to be defined
    \usepackage{enumerate} % Needed for markdown enumerations to work
    \usepackage{geometry} % Used to adjust the document margins
    \usepackage{amsmath} % Equations
    \usepackage{amssymb} % Equations
    \usepackage{textcomp} % defines textquotesingle
    % Hack from http://tex.stackexchange.com/a/47451/13684:
    \AtBeginDocument{%
        \def\PYZsq{\textquotesingle}% Upright quotes in Pygmentized code
    }
    \usepackage{upquote} % Upright quotes for verbatim code
    \usepackage{eurosym} % defines \euro
    \usepackage[mathletters]{ucs} % Extended unicode (utf-8) support
    \usepackage[utf8x]{inputenc} % Allow utf-8 characters in the tex document
    \usepackage{fancyvrb} % verbatim replacement that allows latex
    \usepackage{grffile} % extends the file name processing of package graphics 
                         % to support a larger range 
    % The hyperref package gives us a pdf with properly built
    % internal navigation ('pdf bookmarks' for the table of contents,
    % internal cross-reference links, web links for URLs, etc.)
    \usepackage{hyperref}
    \usepackage{longtable} % longtable support required by pandoc >1.10
    \usepackage{booktabs}  % table support for pandoc > 1.12.2
    \usepackage[normalem]{ulem} % ulem is needed to support strikethroughs (\sout)
                                % normalem makes italics be italics, not underlines
    

    
    
    % Colors for the hyperref package
    \definecolor{urlcolor}{rgb}{0,.145,.698}
    \definecolor{linkcolor}{rgb}{.71,0.21,0.01}
    \definecolor{citecolor}{rgb}{.12,.54,.11}

    % ANSI colors
    \definecolor{ansi-black}{HTML}{3E424D}
    \definecolor{ansi-black-intense}{HTML}{282C36}
    \definecolor{ansi-red}{HTML}{E75C58}
    \definecolor{ansi-red-intense}{HTML}{B22B31}
    \definecolor{ansi-green}{HTML}{00A250}
    \definecolor{ansi-green-intense}{HTML}{007427}
    \definecolor{ansi-yellow}{HTML}{DDB62B}
    \definecolor{ansi-yellow-intense}{HTML}{B27D12}
    \definecolor{ansi-blue}{HTML}{208FFB}
    \definecolor{ansi-blue-intense}{HTML}{0065CA}
    \definecolor{ansi-magenta}{HTML}{D160C4}
    \definecolor{ansi-magenta-intense}{HTML}{A03196}
    \definecolor{ansi-cyan}{HTML}{60C6C8}
    \definecolor{ansi-cyan-intense}{HTML}{258F8F}
    \definecolor{ansi-white}{HTML}{C5C1B4}
    \definecolor{ansi-white-intense}{HTML}{A1A6B2}

    % commands and environments needed by pandoc snippets
    % extracted from the output of `pandoc -s`
    \providecommand{\tightlist}{%
      \setlength{\itemsep}{0pt}\setlength{\parskip}{0pt}}
    \DefineVerbatimEnvironment{Highlighting}{Verbatim}{commandchars=\\\{\}}
    % Add ',fontsize=\small' for more characters per line
    \newenvironment{Shaded}{}{}
    \newcommand{\KeywordTok}[1]{\textcolor[rgb]{0.00,0.44,0.13}{\textbf{{#1}}}}
    \newcommand{\DataTypeTok}[1]{\textcolor[rgb]{0.56,0.13,0.00}{{#1}}}
    \newcommand{\DecValTok}[1]{\textcolor[rgb]{0.25,0.63,0.44}{{#1}}}
    \newcommand{\BaseNTok}[1]{\textcolor[rgb]{0.25,0.63,0.44}{{#1}}}
    \newcommand{\FloatTok}[1]{\textcolor[rgb]{0.25,0.63,0.44}{{#1}}}
    \newcommand{\CharTok}[1]{\textcolor[rgb]{0.25,0.44,0.63}{{#1}}}
    \newcommand{\StringTok}[1]{\textcolor[rgb]{0.25,0.44,0.63}{{#1}}}
    \newcommand{\CommentTok}[1]{\textcolor[rgb]{0.38,0.63,0.69}{\textit{{#1}}}}
    \newcommand{\OtherTok}[1]{\textcolor[rgb]{0.00,0.44,0.13}{{#1}}}
    \newcommand{\AlertTok}[1]{\textcolor[rgb]{1.00,0.00,0.00}{\textbf{{#1}}}}
    \newcommand{\FunctionTok}[1]{\textcolor[rgb]{0.02,0.16,0.49}{{#1}}}
    \newcommand{\RegionMarkerTok}[1]{{#1}}
    \newcommand{\ErrorTok}[1]{\textcolor[rgb]{1.00,0.00,0.00}{\textbf{{#1}}}}
    \newcommand{\NormalTok}[1]{{#1}}
    
    % Additional commands for more recent versions of Pandoc
    \newcommand{\ConstantTok}[1]{\textcolor[rgb]{0.53,0.00,0.00}{{#1}}}
    \newcommand{\SpecialCharTok}[1]{\textcolor[rgb]{0.25,0.44,0.63}{{#1}}}
    \newcommand{\VerbatimStringTok}[1]{\textcolor[rgb]{0.25,0.44,0.63}{{#1}}}
    \newcommand{\SpecialStringTok}[1]{\textcolor[rgb]{0.73,0.40,0.53}{{#1}}}
    \newcommand{\ImportTok}[1]{{#1}}
    \newcommand{\DocumentationTok}[1]{\textcolor[rgb]{0.73,0.13,0.13}{\textit{{#1}}}}
    \newcommand{\AnnotationTok}[1]{\textcolor[rgb]{0.38,0.63,0.69}{\textbf{\textit{{#1}}}}}
    \newcommand{\CommentVarTok}[1]{\textcolor[rgb]{0.38,0.63,0.69}{\textbf{\textit{{#1}}}}}
    \newcommand{\VariableTok}[1]{\textcolor[rgb]{0.10,0.09,0.49}{{#1}}}
    \newcommand{\ControlFlowTok}[1]{\textcolor[rgb]{0.00,0.44,0.13}{\textbf{{#1}}}}
    \newcommand{\OperatorTok}[1]{\textcolor[rgb]{0.40,0.40,0.40}{{#1}}}
    \newcommand{\BuiltInTok}[1]{{#1}}
    \newcommand{\ExtensionTok}[1]{{#1}}
    \newcommand{\PreprocessorTok}[1]{\textcolor[rgb]{0.74,0.48,0.00}{{#1}}}
    \newcommand{\AttributeTok}[1]{\textcolor[rgb]{0.49,0.56,0.16}{{#1}}}
    \newcommand{\InformationTok}[1]{\textcolor[rgb]{0.38,0.63,0.69}{\textbf{\textit{{#1}}}}}
    \newcommand{\WarningTok}[1]{\textcolor[rgb]{0.38,0.63,0.69}{\textbf{\textit{{#1}}}}}
    
    
    % Define a nice break command that doesn't care if a line doesn't already
    % exist.
    \def\br{\hspace*{\fill} \\* }
    % Math Jax compatability definitions
    \def\gt{>}
    \def\lt{<}
    % Document parameters
    \title{BasicIntroduction}
    
    
    

    % Pygments definitions
    
\makeatletter
\def\PY@reset{\let\PY@it=\relax \let\PY@bf=\relax%
    \let\PY@ul=\relax \let\PY@tc=\relax%
    \let\PY@bc=\relax \let\PY@ff=\relax}
\def\PY@tok#1{\csname PY@tok@#1\endcsname}
\def\PY@toks#1+{\ifx\relax#1\empty\else%
    \PY@tok{#1}\expandafter\PY@toks\fi}
\def\PY@do#1{\PY@bc{\PY@tc{\PY@ul{%
    \PY@it{\PY@bf{\PY@ff{#1}}}}}}}
\def\PY#1#2{\PY@reset\PY@toks#1+\relax+\PY@do{#2}}

\expandafter\def\csname PY@tok@gd\endcsname{\def\PY@tc##1{\textcolor[rgb]{0.63,0.00,0.00}{##1}}}
\expandafter\def\csname PY@tok@gu\endcsname{\let\PY@bf=\textbf\def\PY@tc##1{\textcolor[rgb]{0.50,0.00,0.50}{##1}}}
\expandafter\def\csname PY@tok@gt\endcsname{\def\PY@tc##1{\textcolor[rgb]{0.00,0.27,0.87}{##1}}}
\expandafter\def\csname PY@tok@gs\endcsname{\let\PY@bf=\textbf}
\expandafter\def\csname PY@tok@gr\endcsname{\def\PY@tc##1{\textcolor[rgb]{1.00,0.00,0.00}{##1}}}
\expandafter\def\csname PY@tok@cm\endcsname{\let\PY@it=\textit\def\PY@tc##1{\textcolor[rgb]{0.25,0.50,0.50}{##1}}}
\expandafter\def\csname PY@tok@vg\endcsname{\def\PY@tc##1{\textcolor[rgb]{0.10,0.09,0.49}{##1}}}
\expandafter\def\csname PY@tok@vi\endcsname{\def\PY@tc##1{\textcolor[rgb]{0.10,0.09,0.49}{##1}}}
\expandafter\def\csname PY@tok@mh\endcsname{\def\PY@tc##1{\textcolor[rgb]{0.40,0.40,0.40}{##1}}}
\expandafter\def\csname PY@tok@cs\endcsname{\let\PY@it=\textit\def\PY@tc##1{\textcolor[rgb]{0.25,0.50,0.50}{##1}}}
\expandafter\def\csname PY@tok@ge\endcsname{\let\PY@it=\textit}
\expandafter\def\csname PY@tok@vc\endcsname{\def\PY@tc##1{\textcolor[rgb]{0.10,0.09,0.49}{##1}}}
\expandafter\def\csname PY@tok@il\endcsname{\def\PY@tc##1{\textcolor[rgb]{0.40,0.40,0.40}{##1}}}
\expandafter\def\csname PY@tok@go\endcsname{\def\PY@tc##1{\textcolor[rgb]{0.53,0.53,0.53}{##1}}}
\expandafter\def\csname PY@tok@cp\endcsname{\def\PY@tc##1{\textcolor[rgb]{0.74,0.48,0.00}{##1}}}
\expandafter\def\csname PY@tok@gi\endcsname{\def\PY@tc##1{\textcolor[rgb]{0.00,0.63,0.00}{##1}}}
\expandafter\def\csname PY@tok@gh\endcsname{\let\PY@bf=\textbf\def\PY@tc##1{\textcolor[rgb]{0.00,0.00,0.50}{##1}}}
\expandafter\def\csname PY@tok@ni\endcsname{\let\PY@bf=\textbf\def\PY@tc##1{\textcolor[rgb]{0.60,0.60,0.60}{##1}}}
\expandafter\def\csname PY@tok@nl\endcsname{\def\PY@tc##1{\textcolor[rgb]{0.63,0.63,0.00}{##1}}}
\expandafter\def\csname PY@tok@nn\endcsname{\let\PY@bf=\textbf\def\PY@tc##1{\textcolor[rgb]{0.00,0.00,1.00}{##1}}}
\expandafter\def\csname PY@tok@no\endcsname{\def\PY@tc##1{\textcolor[rgb]{0.53,0.00,0.00}{##1}}}
\expandafter\def\csname PY@tok@na\endcsname{\def\PY@tc##1{\textcolor[rgb]{0.49,0.56,0.16}{##1}}}
\expandafter\def\csname PY@tok@nb\endcsname{\def\PY@tc##1{\textcolor[rgb]{0.00,0.50,0.00}{##1}}}
\expandafter\def\csname PY@tok@nc\endcsname{\let\PY@bf=\textbf\def\PY@tc##1{\textcolor[rgb]{0.00,0.00,1.00}{##1}}}
\expandafter\def\csname PY@tok@nd\endcsname{\def\PY@tc##1{\textcolor[rgb]{0.67,0.13,1.00}{##1}}}
\expandafter\def\csname PY@tok@ne\endcsname{\let\PY@bf=\textbf\def\PY@tc##1{\textcolor[rgb]{0.82,0.25,0.23}{##1}}}
\expandafter\def\csname PY@tok@nf\endcsname{\def\PY@tc##1{\textcolor[rgb]{0.00,0.00,1.00}{##1}}}
\expandafter\def\csname PY@tok@si\endcsname{\let\PY@bf=\textbf\def\PY@tc##1{\textcolor[rgb]{0.73,0.40,0.53}{##1}}}
\expandafter\def\csname PY@tok@s2\endcsname{\def\PY@tc##1{\textcolor[rgb]{0.73,0.13,0.13}{##1}}}
\expandafter\def\csname PY@tok@nt\endcsname{\let\PY@bf=\textbf\def\PY@tc##1{\textcolor[rgb]{0.00,0.50,0.00}{##1}}}
\expandafter\def\csname PY@tok@nv\endcsname{\def\PY@tc##1{\textcolor[rgb]{0.10,0.09,0.49}{##1}}}
\expandafter\def\csname PY@tok@s1\endcsname{\def\PY@tc##1{\textcolor[rgb]{0.73,0.13,0.13}{##1}}}
\expandafter\def\csname PY@tok@ch\endcsname{\let\PY@it=\textit\def\PY@tc##1{\textcolor[rgb]{0.25,0.50,0.50}{##1}}}
\expandafter\def\csname PY@tok@m\endcsname{\def\PY@tc##1{\textcolor[rgb]{0.40,0.40,0.40}{##1}}}
\expandafter\def\csname PY@tok@gp\endcsname{\let\PY@bf=\textbf\def\PY@tc##1{\textcolor[rgb]{0.00,0.00,0.50}{##1}}}
\expandafter\def\csname PY@tok@sh\endcsname{\def\PY@tc##1{\textcolor[rgb]{0.73,0.13,0.13}{##1}}}
\expandafter\def\csname PY@tok@ow\endcsname{\let\PY@bf=\textbf\def\PY@tc##1{\textcolor[rgb]{0.67,0.13,1.00}{##1}}}
\expandafter\def\csname PY@tok@sx\endcsname{\def\PY@tc##1{\textcolor[rgb]{0.00,0.50,0.00}{##1}}}
\expandafter\def\csname PY@tok@bp\endcsname{\def\PY@tc##1{\textcolor[rgb]{0.00,0.50,0.00}{##1}}}
\expandafter\def\csname PY@tok@c1\endcsname{\let\PY@it=\textit\def\PY@tc##1{\textcolor[rgb]{0.25,0.50,0.50}{##1}}}
\expandafter\def\csname PY@tok@o\endcsname{\def\PY@tc##1{\textcolor[rgb]{0.40,0.40,0.40}{##1}}}
\expandafter\def\csname PY@tok@kc\endcsname{\let\PY@bf=\textbf\def\PY@tc##1{\textcolor[rgb]{0.00,0.50,0.00}{##1}}}
\expandafter\def\csname PY@tok@c\endcsname{\let\PY@it=\textit\def\PY@tc##1{\textcolor[rgb]{0.25,0.50,0.50}{##1}}}
\expandafter\def\csname PY@tok@mf\endcsname{\def\PY@tc##1{\textcolor[rgb]{0.40,0.40,0.40}{##1}}}
\expandafter\def\csname PY@tok@err\endcsname{\def\PY@bc##1{\setlength{\fboxsep}{0pt}\fcolorbox[rgb]{1.00,0.00,0.00}{1,1,1}{\strut ##1}}}
\expandafter\def\csname PY@tok@mb\endcsname{\def\PY@tc##1{\textcolor[rgb]{0.40,0.40,0.40}{##1}}}
\expandafter\def\csname PY@tok@ss\endcsname{\def\PY@tc##1{\textcolor[rgb]{0.10,0.09,0.49}{##1}}}
\expandafter\def\csname PY@tok@sr\endcsname{\def\PY@tc##1{\textcolor[rgb]{0.73,0.40,0.53}{##1}}}
\expandafter\def\csname PY@tok@mo\endcsname{\def\PY@tc##1{\textcolor[rgb]{0.40,0.40,0.40}{##1}}}
\expandafter\def\csname PY@tok@kd\endcsname{\let\PY@bf=\textbf\def\PY@tc##1{\textcolor[rgb]{0.00,0.50,0.00}{##1}}}
\expandafter\def\csname PY@tok@mi\endcsname{\def\PY@tc##1{\textcolor[rgb]{0.40,0.40,0.40}{##1}}}
\expandafter\def\csname PY@tok@kn\endcsname{\let\PY@bf=\textbf\def\PY@tc##1{\textcolor[rgb]{0.00,0.50,0.00}{##1}}}
\expandafter\def\csname PY@tok@cpf\endcsname{\let\PY@it=\textit\def\PY@tc##1{\textcolor[rgb]{0.25,0.50,0.50}{##1}}}
\expandafter\def\csname PY@tok@kr\endcsname{\let\PY@bf=\textbf\def\PY@tc##1{\textcolor[rgb]{0.00,0.50,0.00}{##1}}}
\expandafter\def\csname PY@tok@s\endcsname{\def\PY@tc##1{\textcolor[rgb]{0.73,0.13,0.13}{##1}}}
\expandafter\def\csname PY@tok@kp\endcsname{\def\PY@tc##1{\textcolor[rgb]{0.00,0.50,0.00}{##1}}}
\expandafter\def\csname PY@tok@w\endcsname{\def\PY@tc##1{\textcolor[rgb]{0.73,0.73,0.73}{##1}}}
\expandafter\def\csname PY@tok@kt\endcsname{\def\PY@tc##1{\textcolor[rgb]{0.69,0.00,0.25}{##1}}}
\expandafter\def\csname PY@tok@sc\endcsname{\def\PY@tc##1{\textcolor[rgb]{0.73,0.13,0.13}{##1}}}
\expandafter\def\csname PY@tok@sb\endcsname{\def\PY@tc##1{\textcolor[rgb]{0.73,0.13,0.13}{##1}}}
\expandafter\def\csname PY@tok@k\endcsname{\let\PY@bf=\textbf\def\PY@tc##1{\textcolor[rgb]{0.00,0.50,0.00}{##1}}}
\expandafter\def\csname PY@tok@se\endcsname{\let\PY@bf=\textbf\def\PY@tc##1{\textcolor[rgb]{0.73,0.40,0.13}{##1}}}
\expandafter\def\csname PY@tok@sd\endcsname{\let\PY@it=\textit\def\PY@tc##1{\textcolor[rgb]{0.73,0.13,0.13}{##1}}}

\def\PYZbs{\char`\\}
\def\PYZus{\char`\_}
\def\PYZob{\char`\{}
\def\PYZcb{\char`\}}
\def\PYZca{\char`\^}
\def\PYZam{\char`\&}
\def\PYZlt{\char`\<}
\def\PYZgt{\char`\>}
\def\PYZsh{\char`\#}
\def\PYZpc{\char`\%}
\def\PYZdl{\char`\$}
\def\PYZhy{\char`\-}
\def\PYZsq{\char`\'}
\def\PYZdq{\char`\"}
\def\PYZti{\char`\~}
% for compatibility with earlier versions
\def\PYZat{@}
\def\PYZlb{[}
\def\PYZrb{]}
\makeatother


    % Exact colors from NB
    \definecolor{incolor}{rgb}{0.0, 0.0, 0.5}
    \definecolor{outcolor}{rgb}{0.545, 0.0, 0.0}



    
    % Prevent overflowing lines due to hard-to-break entities
    \sloppy 
    % Setup hyperref package
    \hypersetup{
      breaklinks=true,  % so long urls are correctly broken across lines
      colorlinks=true,
      urlcolor=urlcolor,
      linkcolor=linkcolor,
      citecolor=citecolor,
      }
    % Slightly bigger margins than the latex defaults
    
    \geometry{verbose,tmargin=1in,bmargin=1in,lmargin=1in,rmargin=1in}
    
    

    \begin{document}
    
    
    \maketitle
    
    

    
    \subsection{A Basic Introduction to
Julia}\label{a-basic-introduction-to-julia}

This quick introduction assumes that you have basic knowledge of some
scripting language and provides an example of the Julia syntax. So
before we explain anything, let's just treat it like a scripting
language, take a head-first dive into Julia, and see what happens.

You'll notice that, given the right syntax, almost everything will
``just work''. There will be some peculiarities, and these we will be
the facts which we will study in much more depth. Usually, these
oddies/differences from other scripting languages are ``the source of
Julia's power''.

    \subsubsection{Problems}\label{problems}

Time to start using your noggin. Scattered in this document are problems
for you to solve using Julia. Many of the details for solving these
problems have been covered, some have not. You may need to use some
external resources:

http://docs.julialang.org/en/release-0.5/manual/

https://gitter.im/JuliaLang/julia

Solve as many or as few problems as you can during these times. Please
work at your own pace, or with others if that's how you're comfortable!

    \subsection{Documentation and
``Hunting''}\label{documentation-and-hunting}

The main source of information is the
\href{http://docs.julialang.org/en/latest/manual/}{Julia Documentation}.
Julia also provides lots of built-in documentation and ways to find out
what's going on. The number of tools for ``hunting down what's going on
/ available'' is too numerous to explain in full detail here, so instead
this will just touch on what's important. For example, the ? gets you to
the documentation for a type, function, etc.

    \begin{Verbatim}[commandchars=\\\{\}]
{\color{incolor}In [{\color{incolor}23}]:} \PY{o}{?}\PY{n}{copy}
\end{Verbatim}

    \begin{Verbatim}[commandchars=\\\{\}]
search: copy copy! copysign deepcopy unsafe\_copy! cospi complex Complex


    \end{Verbatim}
\texttt{\color{outcolor}Out[{\color{outcolor}23}]:}
    
    \begin{verbatim}
copy(x)
\end{verbatim}

Create a shallow copy of \texttt{x}: the outer structure is copied, but
not all internal values. For example, copying an array produces a new
array with identically-same elements as the original.

    

    To find out what methods are available, we can use the \texttt{methods}
function. For example, let's see how \texttt{+} is defined:

    \begin{Verbatim}[commandchars=\\\{\}]
{\color{incolor}In [{\color{incolor}25}]:} \PY{n}{methods}\PY{p}{(}\PY{o}{+}\PY{p}{)}
\end{Verbatim}

            \begin{Verbatim}[commandchars=\\\{\}]
{\color{outcolor}Out[{\color{outcolor}25}]:} \# 163 methods for generic function "+":
         +(x::Bool, z::Complex\{Bool\}) at complex.jl:136
         +(x::Bool, y::Bool) at bool.jl:48
         +(x::Bool) at bool.jl:45
         +\{T<:AbstractFloat\}(x::Bool, y::T) at bool.jl:55
         +(x::Bool, z::Complex) at complex.jl:143
         +(x::Bool, A::AbstractArray\{Bool,N<:Any\}) at arraymath.jl:91
         +(x::Float32, y::Float32) at float.jl:239
         +(x::Float64, y::Float64) at float.jl:240
         +(z::Complex\{Bool\}, x::Bool) at complex.jl:137
         +(z::Complex\{Bool\}, x::Real) at complex.jl:151
         +(a::Float16, b::Float16) at float16.jl:136
         +(x::Char, y::Integer) at char.jl:40
         +(c::BigInt, x::BigFloat) at mpfr.jl:240
         +(a::BigInt, b::BigInt, c::BigInt, d::BigInt, e::BigInt) at gmp.jl:298
         +(a::BigInt, b::BigInt, c::BigInt, d::BigInt) at gmp.jl:291
         +(a::BigInt, b::BigInt, c::BigInt) at gmp.jl:285
         +(x::BigInt, y::BigInt) at gmp.jl:255
         +(x::BigInt, c::Union\{UInt16,UInt32,UInt64,UInt8\}) at gmp.jl:310
         +(x::BigInt, c::Union\{Int16,Int32,Int64,Int8\}) at gmp.jl:326
         +(a::BigFloat, b::BigFloat, c::BigFloat, d::BigFloat, e::BigFloat) at mpfr.jl:388
         +(a::BigFloat, b::BigFloat, c::BigFloat, d::BigFloat) at mpfr.jl:381
         +(a::BigFloat, b::BigFloat, c::BigFloat) at mpfr.jl:375
         +(x::BigFloat, c::BigInt) at mpfr.jl:236
         +(x::BigFloat, y::BigFloat) at mpfr.jl:205
         +(x::BigFloat, c::Union\{UInt16,UInt32,UInt64,UInt8\}) at mpfr.jl:212
         +(x::BigFloat, c::Union\{Int16,Int32,Int64,Int8\}) at mpfr.jl:220
         +(x::BigFloat, c::Union\{Float16,Float32,Float64\}) at mpfr.jl:228
         +\{T\}(B::BitArray\{2\}, J::UniformScaling\{T\}) at linalg/uniformscaling.jl:38
         +(a::Base.Pkg.Resolve.VersionWeights.VWPreBuildItem, b::Base.Pkg.Resolve.VersionWeights.VWPreBuildItem) at pkg/resolve/versionweight.jl:85
         +(a::Base.Pkg.Resolve.VersionWeights.VWPreBuild, b::Base.Pkg.Resolve.VersionWeights.VWPreBuild) at pkg/resolve/versionweight.jl:131
         +(a::Base.Pkg.Resolve.VersionWeights.VersionWeight, b::Base.Pkg.Resolve.VersionWeights.VersionWeight) at pkg/resolve/versionweight.jl:185
         +(a::Base.Pkg.Resolve.MaxSum.FieldValues.FieldValue, b::Base.Pkg.Resolve.MaxSum.FieldValues.FieldValue) at pkg/resolve/fieldvalue.jl:44
         +(x::Base.Dates.CompoundPeriod, y::Base.Dates.CompoundPeriod) at dates/periods.jl:314
         +(x::Base.Dates.CompoundPeriod, y::Base.Dates.Period) at dates/periods.jl:312
         +(x::Base.Dates.CompoundPeriod, y::Base.Dates.TimeType) at dates/periods.jl:359
         +(dt::DateTime, z::Base.Dates.Month) at dates/arithmetic.jl:37
         +(dt::DateTime, y::Base.Dates.Year) at dates/arithmetic.jl:13
         +(x::DateTime, y::Base.Dates.Period) at dates/arithmetic.jl:64
         +(x::Date, y::Base.Dates.Day) at dates/arithmetic.jl:62
         +(x::Date, y::Base.Dates.Week) at dates/arithmetic.jl:60
         +(dt::Date, z::Base.Dates.Month) at dates/arithmetic.jl:43
         +(dt::Date, y::Base.Dates.Year) at dates/arithmetic.jl:17
         +(y::AbstractFloat, x::Bool) at bool.jl:57
         +\{T<:Union\{Int128,Int16,Int32,Int64,Int8,UInt128,UInt16,UInt32,UInt64,UInt8\}\}(x::T, y::T) at int.jl:32
         +(x::Integer, y::Ptr) at pointer.jl:108
         +(z::Complex, w::Complex) at complex.jl:125
         +(z::Complex, x::Bool) at complex.jl:144
         +(x::Real, z::Complex\{Bool\}) at complex.jl:150
         +(x::Real, z::Complex) at complex.jl:162
         +(z::Complex, x::Real) at complex.jl:163
         +(x::Rational, y::Rational) at rational.jl:199
         +(x::Integer, y::Char) at char.jl:41
         +\{N\}(i::Integer, index::CartesianIndex\{N\}) at multidimensional.jl:58
         +(c::Union\{UInt16,UInt32,UInt64,UInt8\}, x::BigInt) at gmp.jl:314
         +(c::Union\{Int16,Int32,Int64,Int8\}, x::BigInt) at gmp.jl:327
         +(c::Union\{UInt16,UInt32,UInt64,UInt8\}, x::BigFloat) at mpfr.jl:216
         +(c::Union\{Int16,Int32,Int64,Int8\}, x::BigFloat) at mpfr.jl:224
         +(c::Union\{Float16,Float32,Float64\}, x::BigFloat) at mpfr.jl:232
         +(x::Irrational, y::Irrational) at irrationals.jl:88
         +(x::Number) at operators.jl:115
         +\{T<:Number\}(x::T, y::T) at promotion.jl:255
         +(x::Number, y::Number) at promotion.jl:190
         +(r1::OrdinalRange, r2::OrdinalRange) at operators.jl:505
         +\{T<:AbstractFloat\}(r1::FloatRange\{T\}, r2::FloatRange\{T\}) at operators.jl:512
         +\{T<:AbstractFloat\}(r1::LinSpace\{T\}, r2::LinSpace\{T\}) at operators.jl:531
         +(r1::Union\{FloatRange,LinSpace,OrdinalRange\}, r2::Union\{FloatRange,LinSpace,OrdinalRange\}) at operators.jl:544
         +(x::Ptr, y::Integer) at pointer.jl:106
         +(A::BitArray, B::BitArray) at bitarray.jl:1042
         +(A::Array\{T<:Any,2\}, B::Diagonal) at linalg/special.jl:121
         +(A::Array\{T<:Any,2\}, B::Bidiagonal) at linalg/special.jl:121
         +(A::Array\{T<:Any,2\}, B::Tridiagonal) at linalg/special.jl:121
         +(A::Array\{T<:Any,2\}, B::SymTridiagonal) at linalg/special.jl:130
         +(A::Array\{T<:Any,2\}, B::Base.LinAlg.AbstractTriangular) at linalg/special.jl:158
         +(A::Array, B::SparseMatrixCSC) at sparse/sparsematrix.jl:1711
         +\{P<:Union\{Base.Dates.CompoundPeriod,Base.Dates.Period\}\}(x::Union\{Base.ReshapedArray\{P,N<:Any,A<:DenseArray,MI<:Tuple\{Vararg\{Base.MultiplicativeInverses.SignedMultiplicativeInverse\{Int64\},N<:Any\}\}\},DenseArray\{P,N<:Any\},SubArray\{P,N<:Any,A<:Union\{Base.ReshapedArray\{T<:Any,N<:Any,A<:DenseArray,MI<:Tuple\{Vararg\{Base.MultiplicativeInverses.SignedMultiplicativeInverse\{Int64\},N<:Any\}\}\},DenseArray\},I<:Tuple\{Vararg\{Union\{Base.AbstractCartesianIndex,Colon,Int64,Range\{Int64\}\},N<:Any\}\},L<:Any\}\}) at dates/periods.jl:323
         +(A::AbstractArray\{Bool,N<:Any\}, x::Bool) at arraymath.jl:90
         +(A::SymTridiagonal, B::SymTridiagonal) at linalg/tridiag.jl:96
         +(A::Tridiagonal, B::Tridiagonal) at linalg/tridiag.jl:494
         +(A::UpperTriangular, B::UpperTriangular) at linalg/triangular.jl:357
         +(A::LowerTriangular, B::LowerTriangular) at linalg/triangular.jl:358
         +(A::UpperTriangular, B::Base.LinAlg.UnitUpperTriangular) at linalg/triangular.jl:359
         +(A::LowerTriangular, B::Base.LinAlg.UnitLowerTriangular) at linalg/triangular.jl:360
         +(A::Base.LinAlg.UnitUpperTriangular, B::UpperTriangular) at linalg/triangular.jl:361
         +(A::Base.LinAlg.UnitLowerTriangular, B::LowerTriangular) at linalg/triangular.jl:362
         +(A::Base.LinAlg.UnitUpperTriangular, B::Base.LinAlg.UnitUpperTriangular) at linalg/triangular.jl:363
         +(A::Base.LinAlg.UnitLowerTriangular, B::Base.LinAlg.UnitLowerTriangular) at linalg/triangular.jl:364
         +(A::Base.LinAlg.AbstractTriangular, B::Base.LinAlg.AbstractTriangular) at linalg/triangular.jl:365
         +(Da::Diagonal, Db::Diagonal) at linalg/diagonal.jl:110
         +(A::Bidiagonal, B::Bidiagonal) at linalg/bidiag.jl:256
         +(UL::UpperTriangular, J::UniformScaling) at linalg/uniformscaling.jl:55
         +(UL::Base.LinAlg.UnitUpperTriangular, J::UniformScaling) at linalg/uniformscaling.jl:58
         +(UL::LowerTriangular, J::UniformScaling) at linalg/uniformscaling.jl:55
         +(UL::Base.LinAlg.UnitLowerTriangular, J::UniformScaling) at linalg/uniformscaling.jl:58
         +(A::Diagonal, B::Bidiagonal) at linalg/special.jl:120
         +(A::Bidiagonal, B::Diagonal) at linalg/special.jl:121
         +(A::Diagonal, B::Tridiagonal) at linalg/special.jl:120
         +(A::Tridiagonal, B::Diagonal) at linalg/special.jl:121
         +(A::Diagonal, B::Array\{T<:Any,2\}) at linalg/special.jl:120
         +(A::Bidiagonal, B::Tridiagonal) at linalg/special.jl:120
         +(A::Tridiagonal, B::Bidiagonal) at linalg/special.jl:121
         +(A::Bidiagonal, B::Array\{T<:Any,2\}) at linalg/special.jl:120
         +(A::Tridiagonal, B::Array\{T<:Any,2\}) at linalg/special.jl:120
         +(A::SymTridiagonal, B::Tridiagonal) at linalg/special.jl:129
         +(A::Tridiagonal, B::SymTridiagonal) at linalg/special.jl:130
         +(A::SymTridiagonal, B::Array\{T<:Any,2\}) at linalg/special.jl:129
         +(A::Diagonal, B::SymTridiagonal) at linalg/special.jl:138
         +(A::SymTridiagonal, B::Diagonal) at linalg/special.jl:139
         +(A::Bidiagonal, B::SymTridiagonal) at linalg/special.jl:138
         +(A::SymTridiagonal, B::Bidiagonal) at linalg/special.jl:139
         +(A::Diagonal, B::UpperTriangular) at linalg/special.jl:150
         +(A::UpperTriangular, B::Diagonal) at linalg/special.jl:151
         +(A::Diagonal, B::Base.LinAlg.UnitUpperTriangular) at linalg/special.jl:150
         +(A::Base.LinAlg.UnitUpperTriangular, B::Diagonal) at linalg/special.jl:151
         +(A::Diagonal, B::LowerTriangular) at linalg/special.jl:150
         +(A::LowerTriangular, B::Diagonal) at linalg/special.jl:151
         +(A::Diagonal, B::Base.LinAlg.UnitLowerTriangular) at linalg/special.jl:150
         +(A::Base.LinAlg.UnitLowerTriangular, B::Diagonal) at linalg/special.jl:151
         +(A::Base.LinAlg.AbstractTriangular, B::SymTridiagonal) at linalg/special.jl:157
         +(A::SymTridiagonal, B::Base.LinAlg.AbstractTriangular) at linalg/special.jl:158
         +(A::Base.LinAlg.AbstractTriangular, B::Tridiagonal) at linalg/special.jl:157
         +(A::Tridiagonal, B::Base.LinAlg.AbstractTriangular) at linalg/special.jl:158
         +(A::Base.LinAlg.AbstractTriangular, B::Bidiagonal) at linalg/special.jl:157
         +(A::Bidiagonal, B::Base.LinAlg.AbstractTriangular) at linalg/special.jl:158
         +(A::Base.LinAlg.AbstractTriangular, B::Array\{T<:Any,2\}) at linalg/special.jl:157
         +\{P<:Union\{Base.Dates.CompoundPeriod,Base.Dates.Period\}\}(Y::Union\{Base.ReshapedArray\{P,N<:Any,A<:DenseArray,MI<:Tuple\{Vararg\{Base.MultiplicativeInverses.SignedMultiplicativeInverse\{Int64\},N<:Any\}\}\},DenseArray\{P,N<:Any\},SubArray\{P,N<:Any,A<:Union\{Base.ReshapedArray\{T<:Any,N<:Any,A<:DenseArray,MI<:Tuple\{Vararg\{Base.MultiplicativeInverses.SignedMultiplicativeInverse\{Int64\},N<:Any\}\}\},DenseArray\},I<:Tuple\{Vararg\{Union\{Base.AbstractCartesianIndex,Colon,Int64,Range\{Int64\}\},N<:Any\}\},L<:Any\}\}, x::Union\{Base.Dates.CompoundPeriod,Base.Dates.Period\}) at dates/periods.jl:337
         +\{P<:Union\{Base.Dates.CompoundPeriod,Base.Dates.Period\},Q<:Union\{Base.Dates.CompoundPeriod,Base.Dates.Period\}\}(X::Union\{Base.ReshapedArray\{P,N<:Any,A<:DenseArray,MI<:Tuple\{Vararg\{Base.MultiplicativeInverses.SignedMultiplicativeInverse\{Int64\},N<:Any\}\}\},DenseArray\{P,N<:Any\},SubArray\{P,N<:Any,A<:Union\{Base.ReshapedArray\{T<:Any,N<:Any,A<:DenseArray,MI<:Tuple\{Vararg\{Base.MultiplicativeInverses.SignedMultiplicativeInverse\{Int64\},N<:Any\}\}\},DenseArray\},I<:Tuple\{Vararg\{Union\{Base.AbstractCartesianIndex,Colon,Int64,Range\{Int64\}\},N<:Any\}\},L<:Any\}\}, Y::Union\{Base.ReshapedArray\{Q,N<:Any,A<:DenseArray,MI<:Tuple\{Vararg\{Base.MultiplicativeInverses.SignedMultiplicativeInverse\{Int64\},N<:Any\}\}\},DenseArray\{Q,N<:Any\},SubArray\{Q,N<:Any,A<:Union\{Base.ReshapedArray\{T<:Any,N<:Any,A<:DenseArray,MI<:Tuple\{Vararg\{Base.MultiplicativeInverses.SignedMultiplicativeInverse\{Int64\},N<:Any\}\}\},DenseArray\},I<:Tuple\{Vararg\{Union\{Base.AbstractCartesianIndex,Colon,Int64,Range\{Int64\}\},N<:Any\}\},L<:Any\}\}) at dates/periods.jl:338
         +\{T<:Base.Dates.TimeType,P<:Union\{Base.Dates.CompoundPeriod,Base.Dates.Period\}\}(x::Union\{Base.ReshapedArray\{P,N<:Any,A<:DenseArray,MI<:Tuple\{Vararg\{Base.MultiplicativeInverses.SignedMultiplicativeInverse\{Int64\},N<:Any\}\}\},DenseArray\{P,N<:Any\},SubArray\{P,N<:Any,A<:Union\{Base.ReshapedArray\{T<:Any,N<:Any,A<:DenseArray,MI<:Tuple\{Vararg\{Base.MultiplicativeInverses.SignedMultiplicativeInverse\{Int64\},N<:Any\}\}\},DenseArray\},I<:Tuple\{Vararg\{Union\{Base.AbstractCartesianIndex,Colon,Int64,Range\{Int64\}\},N<:Any\}\},L<:Any\}\}, y::T) at dates/arithmetic.jl:83
         +\{T<:Base.Dates.TimeType\}(r::Range\{T\}, x::Base.Dates.Period) at dates/ranges.jl:39
         +\{Tv1,Ti1,Tv2,Ti2\}(A\_1::SparseMatrixCSC\{Tv1,Ti1\}, A\_2::SparseMatrixCSC\{Tv2,Ti2\}) at sparse/sparsematrix.jl:1697
         +(A::SparseMatrixCSC, B::Array) at sparse/sparsematrix.jl:1709
         +(A::SparseMatrixCSC, J::UniformScaling) at sparse/sparsematrix.jl:3811
         +(x::AbstractSparseArray\{Tv<:Any,Ti<:Any,1\}, y::AbstractSparseArray\{Tv<:Any,Ti<:Any,1\}) at sparse/sparsevector.jl:1179
         +(x::Union\{Base.ReshapedArray\{T<:Any,1,A<:DenseArray,MI<:Tuple\{Vararg\{Base.MultiplicativeInverses.SignedMultiplicativeInverse\{Int64\},N<:Any\}\}\},DenseArray\{T<:Any,1\},SubArray\{T<:Any,1,A<:Union\{Base.ReshapedArray\{T<:Any,N<:Any,A<:DenseArray,MI<:Tuple\{Vararg\{Base.MultiplicativeInverses.SignedMultiplicativeInverse\{Int64\},N<:Any\}\}\},DenseArray\},I<:Tuple\{Vararg\{Union\{Base.AbstractCartesianIndex,Colon,Int64,Range\{Int64\}\},N<:Any\}\},L<:Any\}\}, y::AbstractSparseArray\{Tv<:Any,Ti<:Any,1\}) at sparse/sparsevector.jl:1180
         +(x::AbstractSparseArray\{Tv<:Any,Ti<:Any,1\}, y::Union\{Base.ReshapedArray\{T<:Any,1,A<:DenseArray,MI<:Tuple\{Vararg\{Base.MultiplicativeInverses.SignedMultiplicativeInverse\{Int64\},N<:Any\}\}\},DenseArray\{T<:Any,1\},SubArray\{T<:Any,1,A<:Union\{Base.ReshapedArray\{T<:Any,N<:Any,A<:DenseArray,MI<:Tuple\{Vararg\{Base.MultiplicativeInverses.SignedMultiplicativeInverse\{Int64\},N<:Any\}\}\},DenseArray\},I<:Tuple\{Vararg\{Union\{Base.AbstractCartesianIndex,Colon,Int64,Range\{Int64\}\},N<:Any\}\},L<:Any\}\}) at sparse/sparsevector.jl:1181
         +\{T<:Number\}(x::AbstractArray\{T,N<:Any\}) at abstractarraymath.jl:91
         +\{R,S\}(A::AbstractArray\{R,N<:Any\}, B::AbstractArray\{S,N<:Any\}) at arraymath.jl:49
         +(A::AbstractArray, x::Number) at arraymath.jl:94
         +(x::Number, A::AbstractArray) at arraymath.jl:95
         +\{N\}(index1::CartesianIndex\{N\}, index2::CartesianIndex\{N\}) at multidimensional.jl:52
         +\{N\}(index::CartesianIndex\{N\}, i::Integer) at multidimensional.jl:57
         +(J1::UniformScaling, J2::UniformScaling) at linalg/uniformscaling.jl:37
         +(J::UniformScaling, B::BitArray\{2\}) at linalg/uniformscaling.jl:39
         +(J::UniformScaling, A::AbstractArray\{T<:Any,2\}) at linalg/uniformscaling.jl:40
         +(J::UniformScaling, x::Number) at linalg/uniformscaling.jl:41
         +(x::Number, J::UniformScaling) at linalg/uniformscaling.jl:42
         +\{TA,TJ\}(A::AbstractArray\{TA,2\}, J::UniformScaling\{TJ\}) at linalg/uniformscaling.jl:102
         +\{T\}(a::Base.Pkg.Resolve.VersionWeights.HierarchicalValue\{T\}, b::Base.Pkg.Resolve.VersionWeights.HierarchicalValue\{T\}) at pkg/resolve/versionweight.jl:23
         +\{P<:Base.Dates.Period\}(x::P, y::P) at dates/periods.jl:70
         +(x::Base.Dates.Period, y::Base.Dates.Period) at dates/periods.jl:311
         +(y::Base.Dates.Period, x::Base.Dates.CompoundPeriod) at dates/periods.jl:313
         +(y::Base.Dates.Period, x::Base.Dates.TimeType) at dates/arithmetic.jl:66
         +\{T<:Base.Dates.TimeType\}(x::Base.Dates.Period, r::Range\{T\}) at dates/ranges.jl:40
         +(x::Union\{Base.Dates.CompoundPeriod,Base.Dates.Period\}) at dates/periods.jl:322
         +\{P<:Union\{Base.Dates.CompoundPeriod,Base.Dates.Period\}\}(x::Union\{Base.Dates.CompoundPeriod,Base.Dates.Period\}, Y::Union\{Base.ReshapedArray\{P,N<:Any,A<:DenseArray,MI<:Tuple\{Vararg\{Base.MultiplicativeInverses.SignedMultiplicativeInverse\{Int64\},N<:Any\}\}\},DenseArray\{P,N<:Any\},SubArray\{P,N<:Any,A<:Union\{Base.ReshapedArray\{T<:Any,N<:Any,A<:DenseArray,MI<:Tuple\{Vararg\{Base.MultiplicativeInverses.SignedMultiplicativeInverse\{Int64\},N<:Any\}\}\},DenseArray\},I<:Tuple\{Vararg\{Union\{Base.AbstractCartesianIndex,Colon,Int64,Range\{Int64\}\},N<:Any\}\},L<:Any\}\}) at dates/periods.jl:336
         +(x::Base.Dates.TimeType) at dates/arithmetic.jl:8
         +(a::Base.Dates.TimeType, b::Base.Dates.Period, c::Base.Dates.Period) at dates/periods.jl:348
         +(a::Base.Dates.TimeType, b::Base.Dates.Period, c::Base.Dates.Period, d::Base.Dates.Period{\ldots}) at dates/periods.jl:350
         +(x::Base.Dates.TimeType, y::Base.Dates.CompoundPeriod) at dates/periods.jl:354
         +(x::Base.Dates.Instant) at dates/arithmetic.jl:4
         +\{T<:Base.Dates.TimeType\}(x::AbstractArray\{T,N<:Any\}, y::Union\{Base.Dates.CompoundPeriod,Base.Dates.Period\}) at dates/arithmetic.jl:76
         +\{T<:Base.Dates.TimeType\}(y::Union\{Base.Dates.CompoundPeriod,Base.Dates.Period\}, x::AbstractArray\{T,N<:Any\}) at dates/arithmetic.jl:77
         +\{P<:Union\{Base.Dates.CompoundPeriod,Base.Dates.Period\}\}(y::Base.Dates.TimeType, x::Union\{Base.ReshapedArray\{P,N<:Any,A<:DenseArray,MI<:Tuple\{Vararg\{Base.MultiplicativeInverses.SignedMultiplicativeInverse\{Int64\},N<:Any\}\}\},DenseArray\{P,N<:Any\},SubArray\{P,N<:Any,A<:Union\{Base.ReshapedArray\{T<:Any,N<:Any,A<:DenseArray,MI<:Tuple\{Vararg\{Base.MultiplicativeInverses.SignedMultiplicativeInverse\{Int64\},N<:Any\}\}\},DenseArray\},I<:Tuple\{Vararg\{Union\{Base.AbstractCartesianIndex,Colon,Int64,Range\{Int64\}\},N<:Any\}\},L<:Any\}\}) at dates/arithmetic.jl:84
         +(a, b, c, xs{\ldots}) at operators.jl:138
\end{Verbatim}
        
    We can inspect a type by finding its fields with \texttt{fieldnames}

    \begin{Verbatim}[commandchars=\\\{\}]
{\color{incolor}In [{\color{incolor}40}]:} \PY{n}{fieldnames}\PY{p}{(}\PY{n}{LinSpace}\PY{p}{)}
\end{Verbatim}

            \begin{Verbatim}[commandchars=\\\{\}]
{\color{outcolor}Out[{\color{outcolor}40}]:} 4-element Array\{Symbol,1\}:
          :start  
          :stop   
          :len    
          :divisor
\end{Verbatim}
        
    and find out which method was used with the \texttt{@which} macro:

    \begin{Verbatim}[commandchars=\\\{\}]
{\color{incolor}In [{\color{incolor}43}]:} \PY{p}{@}\PY{n}{which} \PY{n}{copy}\PY{p}{(}\PY{p}{[}\PY{l+m+mi}{1}\PY{p}{,}\PY{l+m+mi}{2}\PY{p}{,}\PY{l+m+mi}{3}\PY{p}{]}\PY{p}{)}
\end{Verbatim}

            \begin{Verbatim}[commandchars=\\\{\}]
{\color{outcolor}Out[{\color{outcolor}43}]:} copy\{T<:Array\{T,N\}\}(a::T) at array.jl:70
\end{Verbatim}
        
    Notice that this gives you a link to the source code where the function
is defined.

    Lastly, we can find out what type a variable is with the \texttt{typeof}
function:

    \begin{Verbatim}[commandchars=\\\{\}]
{\color{incolor}In [{\color{incolor}44}]:} \PY{n}{a} \PY{o}{=} \PY{p}{[}\PY{l+m+mi}{1}\PY{p}{;}\PY{l+m+mi}{2}\PY{p}{;}\PY{l+m+mi}{3}\PY{p}{]}
         \PY{n+nb}{typeof}\PY{p}{(}\PY{n}{a}\PY{p}{)}
\end{Verbatim}

            \begin{Verbatim}[commandchars=\\\{\}]
{\color{outcolor}Out[{\color{outcolor}44}]:} Array\{Int64,1\}
\end{Verbatim}
        
    \subsubsection{Array Syntax}\label{array-syntax}

The array syntax is similar to MATLAB's conventions.

    \begin{Verbatim}[commandchars=\\\{\}]
{\color{incolor}In [{\color{incolor}11}]:} \PY{n}{a} \PY{o}{=} \PY{n}{Vector}\PY{p}{\PYZob{}}\PY{k+kt}{Float64}\PY{p}{\PYZcb{}}\PY{p}{(}\PY{l+m+mi}{5}\PY{p}{)} \PY{c}{\PYZsh{} Create a length 5 Vector (dimension 1 array) of Float64\PYZsq{}s}
         
         \PY{n}{a} \PY{o}{=} \PY{p}{[}\PY{l+m+mi}{1}\PY{p}{;}\PY{l+m+mi}{2}\PY{p}{;}\PY{l+m+mi}{3}\PY{p}{;}\PY{l+m+mi}{4}\PY{p}{;}\PY{l+m+mi}{5}\PY{p}{]} \PY{c}{\PYZsh{} Create the column vector [1 2 3 4 5]}
         
         \PY{n}{a} \PY{o}{=} \PY{p}{[}\PY{l+m+mi}{1} \PY{l+m+mi}{2} \PY{l+m+mi}{3} \PY{l+m+mi}{4}\PY{p}{]} \PY{c}{\PYZsh{} Create the row vector [1 2 3 4]}
         
         \PY{n}{a}\PY{p}{[}\PY{l+m+mi}{3}\PY{p}{]} \PY{o}{=} \PY{l+m+mi}{2} \PY{c}{\PYZsh{} Change the third element of a (using linear indexing) to 2}
         
         \PY{n}{b} \PY{o}{=} \PY{n}{Matrix}\PY{p}{\PYZob{}}\PY{k+kt}{Float64}\PY{p}{\PYZcb{}}\PY{p}{(}\PY{l+m+mi}{4}\PY{p}{,}\PY{l+m+mi}{2}\PY{p}{)} \PY{c}{\PYZsh{} Define a Matrix of Float64\PYZsq{}s of size (4,2)}
         
         \PY{n}{c} \PY{o}{=} \PY{n}{Array}\PY{p}{\PYZob{}}\PY{k+kt}{Float64}\PY{p}{,}\PY{l+m+mi}{4}\PY{p}{\PYZcb{}}\PY{p}{(}\PY{l+m+mi}{4}\PY{p}{,}\PY{l+m+mi}{5}\PY{p}{,}\PY{l+m+mi}{6}\PY{p}{,}\PY{l+m+mi}{7}\PY{p}{)} \PY{c}{\PYZsh{} Define a (4,5,6,7) array of Float64\PYZsq{}s }
         
         \PY{n}{mat}    \PY{o}{=} \PY{p}{[}\PY{l+m+mi}{1} \PY{l+m+mi}{2} \PY{l+m+mi}{3} \PY{l+m+mi}{4}
                   \PY{l+m+mi}{3} \PY{l+m+mi}{4} \PY{l+m+mi}{5} \PY{l+m+mi}{6}
                   \PY{l+m+mi}{4} \PY{l+m+mi}{4} \PY{l+m+mi}{4} \PY{l+m+mi}{6}
                   \PY{l+m+mi}{3} \PY{l+m+mi}{3} \PY{l+m+mi}{3} \PY{l+m+mi}{3}\PY{p}{]} \PY{c}{\PYZsh{}Define the matrix inline }
         
         \PY{n}{mat}\PY{p}{[}\PY{l+m+mi}{1}\PY{p}{,}\PY{l+m+mi}{2}\PY{p}{]} \PY{o}{=} \PY{l+m+mi}{4} \PY{c}{\PYZsh{} Set element (1,2) (row 1, column 2) to 4}
         
         \PY{n}{mat}
\end{Verbatim}

            \begin{Verbatim}[commandchars=\\\{\}]
{\color{outcolor}Out[{\color{outcolor}11}]:} 4×4 Array\{Int64,2\}:
          1  4  3  4
          3  4  5  6
          4  4  4  6
          3  3  3  3
\end{Verbatim}
        
    Note that, in the console (called the REPL), you can use \texttt{;} to
surpress the output. In a script this is done automatically. Note that
the ``value'' of an array is its pointer to the memory location. This
means that arrays which are set equal affect the same values:

    \begin{Verbatim}[commandchars=\\\{\}]
{\color{incolor}In [{\color{incolor}12}]:} \PY{n}{a} \PY{o}{=} \PY{p}{[}\PY{l+m+mi}{1}\PY{p}{;}\PY{l+m+mi}{3}\PY{p}{;}\PY{l+m+mi}{4}\PY{p}{]}
         \PY{n}{b} \PY{o}{=} \PY{n}{a}
         \PY{n}{b}\PY{p}{[}\PY{l+m+mi}{1}\PY{p}{]} \PY{o}{=} \PY{l+m+mi}{10}
         \PY{n}{a}
\end{Verbatim}

            \begin{Verbatim}[commandchars=\\\{\}]
{\color{outcolor}Out[{\color{outcolor}12}]:} 3-element Array\{Int64,1\}:
          10
           3
           4
\end{Verbatim}
        
    To set an array equal to the values to another array, use copy

    \begin{Verbatim}[commandchars=\\\{\}]
{\color{incolor}In [{\color{incolor}13}]:} \PY{n}{a} \PY{o}{=} \PY{p}{[}\PY{l+m+mi}{1}\PY{p}{;}\PY{l+m+mi}{4}\PY{p}{;}\PY{l+m+mi}{5}\PY{p}{]}
         \PY{n}{b} \PY{o}{=} \PY{n}{copy}\PY{p}{(}\PY{n}{a}\PY{p}{)}
         \PY{n}{b}\PY{p}{[}\PY{l+m+mi}{1}\PY{p}{]} \PY{o}{=} \PY{l+m+mi}{10}
         \PY{n}{a}
\end{Verbatim}

            \begin{Verbatim}[commandchars=\\\{\}]
{\color{outcolor}Out[{\color{outcolor}13}]:} 3-element Array\{Int64,1\}:
          1
          4
          5
\end{Verbatim}
        
    We can also make an array of a similar size and shape via the function
\texttt{similar}, or make an array of zeros/ones with \texttt{zeros} or
\texttt{ones} respectively:

    \begin{Verbatim}[commandchars=\\\{\}]
{\color{incolor}In [{\color{incolor}14}]:} \PY{n}{c} \PY{o}{=} \PY{n}{similar}\PY{p}{(}\PY{n}{a}\PY{p}{)}
         \PY{n}{d} \PY{o}{=} \PY{n}{zeros}\PY{p}{(}\PY{n}{a}\PY{p}{)}
         \PY{n}{e} \PY{o}{=} \PY{n}{ones}\PY{p}{(}\PY{n}{a}\PY{p}{)}
         \PY{n}{println}\PY{p}{(}\PY{n}{c}\PY{p}{)}\PY{p}{;} \PY{n}{println}\PY{p}{(}\PY{n}{d}\PY{p}{)}\PY{p}{;} \PY{n}{println}\PY{p}{(}\PY{n}{e}\PY{p}{)}
\end{Verbatim}

    \begin{Verbatim}[commandchars=\\\{\}]
[786432,0,0]
[0,0,0]
[1,1,1]

    \end{Verbatim}

    Note that arrays can be index'd by arrays:

    \begin{Verbatim}[commandchars=\\\{\}]
{\color{incolor}In [{\color{incolor}15}]:} \PY{n}{a}\PY{p}{[}\PY{l+m+mi}{1}\PY{p}{:}\PY{l+m+mi}{2}\PY{p}{]}
\end{Verbatim}

            \begin{Verbatim}[commandchars=\\\{\}]
{\color{outcolor}Out[{\color{outcolor}15}]:} 2-element Array\{Int64,1\}:
          1
          4
\end{Verbatim}
        
    Arrays can be of any type, specified by the type parameter. One
interesting thing is that this means that arrays can be of arrays:

    \begin{Verbatim}[commandchars=\\\{\}]
{\color{incolor}In [{\color{incolor}12}]:} \PY{n}{a} \PY{o}{=} \PY{n}{Vector}\PY{p}{\PYZob{}}\PY{n}{Vector}\PY{p}{\PYZob{}}\PY{k+kt}{Float64}\PY{p}{\PYZcb{}}\PY{p}{\PYZcb{}}\PY{p}{(}\PY{l+m+mi}{3}\PY{p}{)}
         \PY{n}{a}\PY{p}{[}\PY{l+m+mi}{1}\PY{p}{]} \PY{o}{=} \PY{p}{[}\PY{l+m+mi}{1}\PY{p}{;}\PY{l+m+mi}{2}\PY{p}{;}\PY{l+m+mi}{3}\PY{p}{]}
         \PY{n}{a}\PY{p}{[}\PY{l+m+mi}{2}\PY{p}{]} \PY{o}{=} \PY{p}{[}\PY{l+m+mi}{1}\PY{p}{;}\PY{l+m+mi}{2}\PY{p}{]}
         \PY{n}{a}\PY{p}{[}\PY{l+m+mi}{3}\PY{p}{]} \PY{o}{=} \PY{p}{[}\PY{l+m+mi}{3}\PY{p}{;}\PY{l+m+mi}{4}\PY{p}{;}\PY{l+m+mi}{5}\PY{p}{]}
         \PY{n}{a}
\end{Verbatim}

            \begin{Verbatim}[commandchars=\\\{\}]
{\color{outcolor}Out[{\color{outcolor}12}]:} 3-element Array\{Array\{Float64,1\},1\}:
          [1.0,2.0,3.0]
          [1.0,2.0]    
          [3.0,4.0,5.0]
\end{Verbatim}
        
    \begin{center}\rule{3in}{0.4pt}\end{center}

\paragraph{Question 1}\label{question-1}

Can you explain the following behavior? Julia's community values
consistancy of the rules, so all of the behavior is deducible from
simple rules. (Hint: I have noted all of the rules involved here).

    \begin{Verbatim}[commandchars=\\\{\}]
{\color{incolor}In [{\color{incolor}14}]:} \PY{n}{b} \PY{o}{=} \PY{n}{a}
         \PY{n}{b}\PY{p}{[}\PY{l+m+mi}{1}\PY{p}{]} \PY{o}{=} \PY{p}{[}\PY{l+m+mi}{1}\PY{p}{;}\PY{l+m+mi}{4}\PY{p}{;}\PY{l+m+mi}{5}\PY{p}{]}
         \PY{n}{a}
\end{Verbatim}

            \begin{Verbatim}[commandchars=\\\{\}]
{\color{outcolor}Out[{\color{outcolor}14}]:} 3-element Array\{Array\{Float64,1\},1\}:
          [1.0,4.0,5.0]
          [1.0,2.0]    
          [3.0,4.0,5.0]
\end{Verbatim}
        
    \begin{center}\rule{3in}{0.4pt}\end{center}

To fix this, there is a recursive copy function: \texttt{deepcopy}

    \begin{Verbatim}[commandchars=\\\{\}]
{\color{incolor}In [{\color{incolor}16}]:} \PY{n}{b} \PY{o}{=} \PY{n}{deepcopy}\PY{p}{(}\PY{n}{a}\PY{p}{)}
         \PY{n}{b}\PY{p}{[}\PY{l+m+mi}{1}\PY{p}{]} \PY{o}{=} \PY{p}{[}\PY{l+m+mi}{1}\PY{p}{;}\PY{l+m+mi}{2}\PY{p}{;}\PY{l+m+mi}{3}\PY{p}{]}
         \PY{n}{a}
\end{Verbatim}

            \begin{Verbatim}[commandchars=\\\{\}]
{\color{outcolor}Out[{\color{outcolor}16}]:} 3-element Array\{Array\{Float64,1\},1\}:
          [1.0,4.0,5.0]
          [1.0,2.0]    
          [3.0,4.0,5.0]
\end{Verbatim}
        
    For high performance, Julia provides mutating functions. These functions
change the input values that are passed in, instead of returning a new
value. By convention, mutating functions tend to be defined with a
\texttt{!} at the end and tend to mutate their first argument. An
example of a mutating function in \texttt{scale!} which scales an array
by a scalar (or array)

    \begin{Verbatim}[commandchars=\\\{\}]
{\color{incolor}In [{\color{incolor}19}]:} \PY{n}{a} \PY{o}{=} \PY{p}{[}\PY{l+m+mi}{1}\PY{p}{;}\PY{l+m+mi}{6}\PY{p}{;}\PY{l+m+mi}{8}\PY{p}{]}
         \PY{n}{scale!}\PY{p}{(}\PY{n}{a}\PY{p}{,}\PY{l+m+mi}{2}\PY{p}{)} \PY{c}{\PYZsh{} a changes}
\end{Verbatim}

            \begin{Verbatim}[commandchars=\\\{\}]
{\color{outcolor}Out[{\color{outcolor}19}]:} 3-element Array\{Int64,1\}:
           2
          12
          16
\end{Verbatim}
        
    The purpose of mutating functions is that they allow one to reduce the
number of memory allocations which is crucial for achiving high
performance.

    \subsection{Control Flow}\label{control-flow}

Control flow in Julia is pretty standard. You have your basic for and
while loops, and your if statements. There's more in the documentation.

    \begin{Verbatim}[commandchars=\\\{\}]
{\color{incolor}In [{\color{incolor}16}]:} \PY{k}{for} \PY{n}{i}\PY{o}{=}\PY{l+m+mi}{1}\PY{p}{:}\PY{l+m+mi}{5} \PY{c}{\PYZsh{}for i goes from 1 to 5}
             \PY{n}{println}\PY{p}{(}\PY{n}{i}\PY{p}{)}
         \PY{k}{end}
         
         \PY{n}{t} \PY{o}{=} \PY{l+m+mi}{0}
         \PY{k}{while} \PY{n}{t}\PY{o}{\PYZlt{}}\PY{l+m+mi}{5}
             \PY{n}{println}\PY{p}{(}\PY{n}{t}\PY{p}{)}
             \PY{n}{t}\PY{o}{+}\PY{o}{=}\PY{l+m+mi}{1} \PY{c}{\PYZsh{} t = t + 1}
         \PY{k}{end}
         
         \PY{n}{school} \PY{o}{=} \PY{p}{:}\PY{n}{UCI}
         
         \PY{k}{if} \PY{n}{school}\PY{o}{==}\PY{p}{:}\PY{n}{UCI}
             \PY{n}{println}\PY{p}{(}\PY{l+s}{\PYZdq{}}\PY{l+s}{ZotZotZot}\PY{l+s}{\PYZdq{}}\PY{p}{)}
         \PY{k}{else}
             \PY{n}{println}\PY{p}{(}\PY{l+s}{\PYZdq{}}\PY{l+s}{Not even worth discussing.}\PY{l+s}{\PYZdq{}}\PY{p}{)}
         \PY{k}{end}
\end{Verbatim}

    \begin{Verbatim}[commandchars=\\\{\}]
1
2
3
4
5
0
1
2
3
4
ZotZotZot

    \end{Verbatim}

    One interesting feature about Julia control flow is that we can write
multiple loops in one line:

    \begin{Verbatim}[commandchars=\\\{\}]
{\color{incolor}In [{\color{incolor}17}]:} \PY{k}{for} \PY{n}{i}\PY{o}{=}\PY{l+m+mi}{1}\PY{p}{:}\PY{l+m+mi}{2}\PY{p}{,}\PY{n}{j}\PY{o}{=}\PY{l+m+mi}{2}\PY{p}{:}\PY{l+m+mi}{4}
             \PY{n}{println}\PY{p}{(}\PY{n}{i}\PY{o}{*}\PY{n}{j}\PY{p}{)}
         \PY{k}{end}
\end{Verbatim}

    \begin{Verbatim}[commandchars=\\\{\}]
2
3
4
4
6
8

    \end{Verbatim}

    \paragraph{Problem 1}\label{problem-1}

Use Julia's array and control flow syntax in order to define the NxN
Strang matrix:

\[ \left[\begin{array}{ccccc}
-2 & 1\\
1 & -2 & 1\\
 & \ddots & \ddots & \ddots\\
 &  & \ddots & \ddots & 1\\
 &  &  & 1 & -2
\end{array}\right] \]

    \begin{Verbatim}[commandchars=\\\{\}]
{\color{incolor}In [{\color{incolor} }]:} \PY{c}{\PYZsh{}\PYZsh{}\PYZsh{}\PYZsh{} Prepare Data}
        
        \PY{n}{X} \PY{o}{=} \PY{n}{rand}\PY{p}{(}\PY{l+m+mi}{1000}\PY{p}{,} \PY{l+m+mi}{3}\PY{p}{)}               \PY{c}{\PYZsh{} feature matrix}
        \PY{n}{a0} \PY{o}{=} \PY{n}{rand}\PY{p}{(}\PY{l+m+mi}{3}\PY{p}{)}                    \PY{c}{\PYZsh{} ground truths}
        \PY{n}{y} \PY{o}{=} \PY{n}{X} \PY{o}{*} \PY{n}{a0} \PY{o}{+} \PY{l+m+mf}{0.1} \PY{o}{*} \PY{n}{randn}\PY{p}{(}\PY{l+m+mi}{1000}\PY{p}{)}\PY{p}{;}  \PY{c}{\PYZsh{} generate response}
\end{Verbatim}

    \paragraph{Problem 2}\label{problem-2}

Given an Nx3 array of data (\texttt{randn(N,3)}) and a Nx1 array of
outcomes, produce the data matrix \texttt{X} which appends a column of
1's to the data matrix, and solve for the 4x1 array \texttt{β} via
\texttt{βX = b} using \texttt{qrfact} or \texttt{\textbackslash{}}.
(Note: This is linear regression)

\paragraph{Problem 3}\label{problem-3}

Compare your results to that of using \texttt{llsq} from
\texttt{MultivariateStats.jl} (note: you need to go find the
documentation to find out how to use this!)

\paragraph{Problem 4}\label{problem-4}

Compare your results to that of using ordinary least squares regression
from \texttt{GLM.jl}

\paragraph{Problem 5}\label{problem-5}

The logistic difference equation is defined by the recursion

\[ b_{n+1}=r*b_{n}(1-b_{n}) \]

where $b_{n}$ is the number of bunnies at time $n$. Starting with
$b_{0}=.25$, by around $400$ iterations this will reach a steady state.
This steady state (or steady periodic state) is dependent on $r$. Write
a function which solves for the steady state(s) for each given $r$, and
plot ``every state'' in the steady attractor for each $r$ (x-axis is
$r$, $y$=value seen in the attractor) using Plots.jl. Take
$r\in\left(2.9,4\right)$

    \subsection{Function Syntax}\label{function-syntax}

    \begin{Verbatim}[commandchars=\\\{\}]
{\color{incolor}In [{\color{incolor}1}]:} \PY{n}{f}\PY{p}{(}\PY{n}{x}\PY{p}{,}\PY{n}{y}\PY{p}{)} \PY{o}{=} \PY{l+m+mi}{2}\PY{n}{x}\PY{o}{+}\PY{n}{y} \PY{c}{\PYZsh{} Create an inline function}
\end{Verbatim}

            \begin{Verbatim}[commandchars=\\\{\}]
{\color{outcolor}Out[{\color{outcolor}1}]:} f (generic function with 1 method)
\end{Verbatim}
        
    \begin{Verbatim}[commandchars=\\\{\}]
{\color{incolor}In [{\color{incolor}22}]:} \PY{n}{f}\PY{p}{(}\PY{l+m+mi}{1}\PY{p}{,}\PY{l+m+mi}{2}\PY{p}{)} \PY{c}{\PYZsh{} Call the function}
\end{Verbatim}

            \begin{Verbatim}[commandchars=\\\{\}]
{\color{outcolor}Out[{\color{outcolor}22}]:} 4
\end{Verbatim}
        
    \begin{Verbatim}[commandchars=\\\{\}]
{\color{incolor}In [{\color{incolor}2}]:} \PY{k}{function}\PY{n+nf}{ }\PY{n+nf}{f}\PY{p}{(}\PY{n}{x}\PY{p}{)}
          \PY{n}{x}\PY{o}{+}\PY{l+m+mi}{2}  
        \PY{k}{end} \PY{c}{\PYZsh{} Long form definition}
\end{Verbatim}

            \begin{Verbatim}[commandchars=\\\{\}]
{\color{outcolor}Out[{\color{outcolor}2}]:} f (generic function with 2 methods)
\end{Verbatim}
        
    By default, Julia functions return the last value computed within them.

    \begin{Verbatim}[commandchars=\\\{\}]
{\color{incolor}In [{\color{incolor}26}]:} \PY{n}{f}\PY{p}{(}\PY{l+m+mi}{2}\PY{p}{)}
\end{Verbatim}

            \begin{Verbatim}[commandchars=\\\{\}]
{\color{outcolor}Out[{\color{outcolor}26}]:} 4
\end{Verbatim}
        
    A key feature of Julia is multiple dispatch. Notice here that there is
``one function'', \texttt{f}, with two methods. Methods are the
actionable parts of a function. Here, there is one method defined as
\texttt{(::Any,::Any)} and \texttt{(::Any)}, meaning that if you give
\texttt{f} two values then it will call the first method, and if you
give it one value then it will call the second method.

Multiple dispatch works on types. To define a dispatch on a type, use
the \texttt{::Type} signifier:

    \begin{Verbatim}[commandchars=\\\{\}]
{\color{incolor}In [{\color{incolor}3}]:} \PY{n}{f}\PY{p}{(}\PY{n}{x}\PY{p}{:}\PY{p}{:}\PY{k+kt}{Int}\PY{p}{,}\PY{n}{y}\PY{p}{:}\PY{p}{:}\PY{k+kt}{Int}\PY{p}{)} \PY{o}{=} \PY{l+m+mi}{3}\PY{n}{x}\PY{o}{+}\PY{l+m+mi}{2}\PY{n}{y}
\end{Verbatim}

            \begin{Verbatim}[commandchars=\\\{\}]
{\color{outcolor}Out[{\color{outcolor}3}]:} f (generic function with 3 methods)
\end{Verbatim}
        
    Julia will dispatch onto the strictest acceptible type signature.

    \begin{Verbatim}[commandchars=\\\{\}]
{\color{incolor}In [{\color{incolor}30}]:} \PY{n}{f}\PY{p}{(}\PY{l+m+mi}{2}\PY{p}{,}\PY{l+m+mi}{3}\PY{p}{)} \PY{c}{\PYZsh{} 3x+2y}
\end{Verbatim}

            \begin{Verbatim}[commandchars=\\\{\}]
{\color{outcolor}Out[{\color{outcolor}30}]:} 12
\end{Verbatim}
        
    \begin{Verbatim}[commandchars=\\\{\}]
{\color{incolor}In [{\color{incolor}32}]:} \PY{n}{f}\PY{p}{(}\PY{l+m+mf}{2.0}\PY{p}{,}\PY{l+m+mi}{3}\PY{p}{)} \PY{c}{\PYZsh{} 2x+y since 2.0 is not an Int}
\end{Verbatim}

            \begin{Verbatim}[commandchars=\\\{\}]
{\color{outcolor}Out[{\color{outcolor}32}]:} 7.0
\end{Verbatim}
        
    Types in signatures can be parametric. For example, we can define a
method for ``two values are passed in, both Numbers and having the same
type''. Note that \texttt{\textless{}:} means ``a subtype of''.

    \begin{Verbatim}[commandchars=\\\{\}]
{\color{incolor}In [{\color{incolor}4}]:} \PY{n}{f}\PY{p}{\PYZob{}}\PY{n}{T}\PY{o}{\PYZlt{}:}\PY{n}{Number}\PY{p}{\PYZcb{}}\PY{p}{(}\PY{n}{x}\PY{p}{:}\PY{p}{:}\PY{n}{T}\PY{p}{,}\PY{n}{y}\PY{p}{:}\PY{p}{:}\PY{n}{T}\PY{p}{)} \PY{o}{=} \PY{l+m+mi}{4}\PY{n}{x}\PY{o}{+}\PY{l+m+mi}{10}\PY{n}{y}
\end{Verbatim}

            \begin{Verbatim}[commandchars=\\\{\}]
{\color{outcolor}Out[{\color{outcolor}4}]:} f (generic function with 4 methods)
\end{Verbatim}
        
    \begin{Verbatim}[commandchars=\\\{\}]
{\color{incolor}In [{\color{incolor}37}]:} \PY{n}{f}\PY{p}{(}\PY{l+m+mi}{2}\PY{p}{,}\PY{l+m+mi}{3}\PY{p}{)} \PY{c}{\PYZsh{} 3x+2y since (::Int,::Int) is stricter}
\end{Verbatim}

            \begin{Verbatim}[commandchars=\\\{\}]
{\color{outcolor}Out[{\color{outcolor}37}]:} 12
\end{Verbatim}
        
    \begin{Verbatim}[commandchars=\\\{\}]
{\color{incolor}In [{\color{incolor}4}]:} \PY{n}{f}\PY{p}{(}\PY{l+m+mf}{2.0}\PY{p}{,}\PY{l+m+mf}{3.0}\PY{p}{)} \PY{c}{\PYZsh{} 4x+10y}
\end{Verbatim}

            \begin{Verbatim}[commandchars=\\\{\}]
{\color{outcolor}Out[{\color{outcolor}4}]:} 38.0
\end{Verbatim}
        
    Note that type parameterizations can have as many types as possible, and
do not need to declare a supertype. For example, we can say that there
is an \texttt{x} which must be a Number, while \texttt{y} and \texttt{z}
must match types:

    \begin{Verbatim}[commandchars=\\\{\}]
{\color{incolor}In [{\color{incolor}5}]:} \PY{n}{f}\PY{p}{\PYZob{}}\PY{n}{T}\PY{o}{\PYZlt{}:}\PY{n}{Number}\PY{p}{,}\PY{n}{T2}\PY{p}{\PYZcb{}}\PY{p}{(}\PY{n}{x}\PY{p}{:}\PY{p}{:}\PY{n}{T}\PY{p}{,}\PY{n}{y}\PY{p}{:}\PY{p}{:}\PY{n}{T2}\PY{p}{,}\PY{n}{z}\PY{p}{:}\PY{p}{:}\PY{n}{T2}\PY{p}{)} \PY{o}{=} \PY{l+m+mi}{5}\PY{n}{x} \PY{o}{+} \PY{l+m+mi}{5}\PY{n}{y} \PY{o}{+} \PY{l+m+mi}{5}\PY{n}{z}
\end{Verbatim}

            \begin{Verbatim}[commandchars=\\\{\}]
{\color{outcolor}Out[{\color{outcolor}5}]:} f (generic function with 5 methods)
\end{Verbatim}
        
    We will go into more depth on multiple dispatch later since this is the
core design feature of Julia. The key feature is that Julia functions
specialize on the types of their arguments. This means that \texttt{f}
is a separately compiled function for each method (and for parametric
types, each possible method). The first time it is called it will
compile.

    \begin{center}\rule{3in}{0.4pt}\end{center}

\paragraph{Question 2}\label{question-2}

Can you explain these timings?

    \begin{Verbatim}[commandchars=\\\{\}]
{\color{incolor}In [{\color{incolor}6}]:} \PY{n}{f}\PY{p}{(}\PY{n}{x}\PY{p}{,}\PY{n}{y}\PY{p}{,}\PY{n}{z}\PY{p}{,}\PY{n}{w}\PY{p}{)} \PY{o}{=} \PY{n}{x}\PY{o}{+}\PY{n}{y}\PY{o}{+}\PY{n}{z}\PY{o}{+}\PY{n}{w}
        \PY{p}{@}\PY{n}{time} \PY{n}{f}\PY{p}{(}\PY{l+m+mi}{1}\PY{p}{,}\PY{l+m+mi}{1}\PY{p}{,}\PY{l+m+mi}{1}\PY{p}{,}\PY{l+m+mi}{1}\PY{p}{)}
        \PY{p}{@}\PY{n}{time} \PY{n}{f}\PY{p}{(}\PY{l+m+mi}{1}\PY{p}{,}\PY{l+m+mi}{1}\PY{p}{,}\PY{l+m+mi}{1}\PY{p}{,}\PY{l+m+mi}{1}\PY{p}{)}
        \PY{p}{@}\PY{n}{time} \PY{n}{f}\PY{p}{(}\PY{l+m+mi}{1}\PY{p}{,}\PY{l+m+mi}{1}\PY{p}{,}\PY{l+m+mi}{1}\PY{p}{,}\PY{l+m+mi}{1}\PY{p}{)}
        \PY{p}{@}\PY{n}{time} \PY{n}{f}\PY{p}{(}\PY{l+m+mi}{1}\PY{p}{,}\PY{l+m+mi}{1}\PY{p}{,}\PY{l+m+mi}{1}\PY{p}{,}\PY{l+m+mf}{1.0}\PY{p}{)}
        \PY{p}{@}\PY{n}{time} \PY{n}{f}\PY{p}{(}\PY{l+m+mi}{1}\PY{p}{,}\PY{l+m+mi}{1}\PY{p}{,}\PY{l+m+mi}{1}\PY{p}{,}\PY{l+m+mf}{1.0}\PY{p}{)}
\end{Verbatim}

    \begin{Verbatim}[commandchars=\\\{\}]
  0.003880 seconds (387 allocations: 21.150 KB)
  0.000002 seconds (4 allocations: 160 bytes)
  0.000001 seconds (3 allocations: 144 bytes)
  0.002570 seconds (1.33 k allocations: 69.866 KB)
  0.000001 seconds (4 allocations: 160 bytes)

    \end{Verbatim}

            \begin{Verbatim}[commandchars=\\\{\}]
{\color{outcolor}Out[{\color{outcolor}6}]:} 4.0
\end{Verbatim}
        
    \begin{center}\rule{3in}{0.4pt}\end{center}

    Note that functions can also feature optional arguments:

    \begin{Verbatim}[commandchars=\\\{\}]
{\color{incolor}In [{\color{incolor}42}]:} \PY{k}{function}\PY{n+nf}{ }\PY{n+nf}{test\PYZus{}function}\PY{p}{(}\PY{n}{x}\PY{p}{,}\PY{n}{y}\PY{p}{;}\PY{n}{z}\PY{o}{=}\PY{l+m+mi}{0}\PY{p}{)} \PY{c}{\PYZsh{}z is an optional argument}
           \PY{k}{if} \PY{n}{z}\PY{o}{==}\PY{l+m+mi}{0}
             \PY{k}{return} \PY{n}{x}\PY{o}{+}\PY{n}{y}\PY{p}{,}\PY{n}{x}\PY{o}{*}\PY{n}{y} \PY{c}{\PYZsh{}Return a tuple}
           \PY{k}{else}
           \PY{k}{return} \PY{n}{x}\PY{o}{*}\PY{n}{y}\PY{o}{*}\PY{n}{z}\PY{p}{,}\PY{n}{x}\PY{o}{+}\PY{n}{y}\PY{o}{+}\PY{n}{z} \PY{c}{\PYZsh{}Return a different tuple}
           \PY{c}{\PYZsh{}whitespace is optional}
           \PY{k}{end} \PY{c}{\PYZsh{}End if statement}
         \PY{k}{end} \PY{c}{\PYZsh{}End function definition}
\end{Verbatim}

            \begin{Verbatim}[commandchars=\\\{\}]
{\color{outcolor}Out[{\color{outcolor}42}]:} test\_function (generic function with 1 method)
\end{Verbatim}
        
    Here, if z is not specified, then it's 0.

    \begin{Verbatim}[commandchars=\\\{\}]
{\color{incolor}In [{\color{incolor}45}]:} \PY{n}{x}\PY{p}{,}\PY{n}{y} \PY{o}{=} \PY{n}{test\PYZus{}function}\PY{p}{(}\PY{l+m+mi}{1}\PY{p}{,}\PY{l+m+mi}{2}\PY{p}{)}
\end{Verbatim}

            \begin{Verbatim}[commandchars=\\\{\}]
{\color{outcolor}Out[{\color{outcolor}45}]:} (3,2)
\end{Verbatim}
        
    \begin{Verbatim}[commandchars=\\\{\}]
{\color{incolor}In [{\color{incolor}46}]:} \PY{n}{x}\PY{p}{,}\PY{n}{y} \PY{o}{=} \PY{n}{test\PYZus{}function}\PY{p}{(}\PY{l+m+mi}{1}\PY{p}{,}\PY{l+m+mi}{2}\PY{p}{;}\PY{n}{z}\PY{o}{=}\PY{l+m+mi}{3}\PY{p}{)}
\end{Verbatim}

            \begin{Verbatim}[commandchars=\\\{\}]
{\color{outcolor}Out[{\color{outcolor}46}]:} (6,6)
\end{Verbatim}
        
    Notice that we also featured multiple return values.

    \begin{Verbatim}[commandchars=\\\{\}]
{\color{incolor}In [{\color{incolor}47}]:} \PY{n}{println}\PY{p}{(}\PY{n}{x}\PY{p}{)}\PY{p}{;} \PY{n}{println}\PY{p}{(}\PY{n}{y}\PY{p}{)}
\end{Verbatim}

    \begin{Verbatim}[commandchars=\\\{\}]
6
6

    \end{Verbatim}

    The return type for multiple return values is a Tuple. The syntax for a
tuple is \texttt{(x,y,z,...)} or inside of functions you can use the
shorthand \texttt{x,y,z,...} as shown.

Note that functions in Julia are ``first-class''. This means that
functions are just a type themselves. Therefore functions can make
functions, you can store functions as variables, pass them as variables,
etc. For example:

    \begin{Verbatim}[commandchars=\\\{\}]
{\color{incolor}In [{\color{incolor}6}]:} \PY{k}{function}\PY{n+nf}{ }\PY{n+nf}{function\PYZus{}playtime}\PY{p}{(}\PY{n}{x}\PY{p}{)} \PY{c}{\PYZsh{}z is an optional argument}
            \PY{n}{y} \PY{o}{=} \PY{l+m+mi}{2}\PY{o}{+}\PY{n}{x}
            \PY{k}{function}\PY{n+nf}{ }\PY{n+nf}{test}\PY{p}{(}\PY{p}{)}
                \PY{l+m+mi}{2}\PY{n}{y} \PY{c}{\PYZsh{} y is defined in the previous scope, so it\PYZsq{}s available here}
            \PY{k}{end}
            \PY{n}{z} \PY{o}{=} \PY{n}{test}\PY{p}{(}\PY{p}{)} \PY{o}{*} \PY{n}{test}\PY{p}{(}\PY{p}{)}
            \PY{k}{return} \PY{n}{z}\PY{p}{,}\PY{n}{test}
        \PY{k}{end} \PY{c}{\PYZsh{}End function definition}
        \PY{n}{z}\PY{p}{,}\PY{n}{test} \PY{o}{=} \PY{n}{function\PYZus{}playtime}\PY{p}{(}\PY{l+m+mi}{2}\PY{p}{)}
\end{Verbatim}

            \begin{Verbatim}[commandchars=\\\{\}]
{\color{outcolor}Out[{\color{outcolor}6}]:} (64,test)
\end{Verbatim}
        
    \begin{Verbatim}[commandchars=\\\{\}]
{\color{incolor}In [{\color{incolor}14}]:} \PY{n}{test}\PY{p}{(}\PY{p}{)}
\end{Verbatim}

            \begin{Verbatim}[commandchars=\\\{\}]
{\color{outcolor}Out[{\color{outcolor}14}]:} 8
\end{Verbatim}
        
    Notice that \texttt{test()} does not get passed in \texttt{y} but knows
what \texttt{y} is. This is due to the function scoping rules: an inner
function can know the variables defined in the same scope as the
function. This rule is recursive, leading us to the conclusion that the
top level scope is global. Yes, that means

    \begin{Verbatim}[commandchars=\\\{\}]
{\color{incolor}In [{\color{incolor}18}]:} \PY{n}{a} \PY{o}{=} \PY{l+m+mi}{2}
\end{Verbatim}

            \begin{Verbatim}[commandchars=\\\{\}]
{\color{outcolor}Out[{\color{outcolor}18}]:} 2
\end{Verbatim}
        
    defines a global variable. We will go into more detail on this.

    Lastly we show the anonymous function syntax. This allows you to define
a function inline.

    \begin{Verbatim}[commandchars=\\\{\}]
{\color{incolor}In [{\color{incolor}20}]:} \PY{n}{g} \PY{o}{=} \PY{p}{(}\PY{n}{x}\PY{p}{,}\PY{n}{y}\PY{p}{)} \PY{o}{\PYZhy{}\PYZgt{}} \PY{l+m+mi}{2}\PY{n}{x}\PY{o}{+}\PY{n}{y}
\end{Verbatim}

            \begin{Verbatim}[commandchars=\\\{\}]
{\color{outcolor}Out[{\color{outcolor}20}]:} (::\#5) (generic function with 1 method)
\end{Verbatim}
        
    Unlike named functions, \texttt{g} is simply a function in a variable
and can be overwritten at any time:

    \begin{Verbatim}[commandchars=\\\{\}]
{\color{incolor}In [{\color{incolor}21}]:} \PY{n}{g} \PY{o}{=} \PY{p}{(}\PY{n}{x}\PY{p}{)} \PY{o}{\PYZhy{}\PYZgt{}} \PY{l+m+mi}{2}\PY{n}{x}
\end{Verbatim}

            \begin{Verbatim}[commandchars=\\\{\}]
{\color{outcolor}Out[{\color{outcolor}21}]:} (::\#7) (generic function with 1 method)
\end{Verbatim}
        
    An anonymous function cannot have more than 1 dispatch. However, as of
v0.5, they are compiled and thus do not have any performance
disadvantages from named functions.

    \subsection{Type Declaration Syntax}\label{type-declaration-syntax}

A type is what in many other languages is an ``object''. If that is a
foreign concept, thing of a type as a thing which has named components.
A type is the idea for what the thing is, while an instantiation of the
type is a specific one. For example, you can think of a car as having an
make and a model. So that means a Toyota RAV4 is an instantiation of the
car type.

In Julia, we would define the car type as follows:

    \begin{Verbatim}[commandchars=\\\{\}]
{\color{incolor}In [{\color{incolor}48}]:} \PY{k}{type}\PY{n+nc}{ }\PY{n+nc}{Car}
             \PY{n}{make}
             \PY{n}{model}
         \PY{k}{end}
\end{Verbatim}

    We could then make the instance of a car as follows:

    \begin{Verbatim}[commandchars=\\\{\}]
{\color{incolor}In [{\color{incolor}49}]:} \PY{n}{mycar} \PY{o}{=} \PY{n}{Car}\PY{p}{(}\PY{l+s}{\PYZdq{}}\PY{l+s}{Toyota}\PY{l+s}{\PYZdq{}}\PY{p}{,}\PY{l+s}{\PYZdq{}}\PY{l+s}{Rav4}\PY{l+s}{\PYZdq{}}\PY{p}{)}
\end{Verbatim}

            \begin{Verbatim}[commandchars=\\\{\}]
{\color{outcolor}Out[{\color{outcolor}49}]:} Car("Toyota","Rav4")
\end{Verbatim}
        
    Here I introduced the string syntax for Julia which uses ``\ldots{}''
(like most other languages, I'm glaring at you MATLAB). I can grab the
``fields'' of my type using the \texttt{.} syntax:

    \begin{Verbatim}[commandchars=\\\{\}]
{\color{incolor}In [{\color{incolor}51}]:} \PY{n}{mycar}\PY{o}{.}\PY{n}{make}
\end{Verbatim}

            \begin{Verbatim}[commandchars=\\\{\}]
{\color{outcolor}Out[{\color{outcolor}51}]:} "Toyota"
\end{Verbatim}
        
    To ``enhance Julia's performance'', one usually likes to make the typing
stricter. For example, we can define a WorkshopParticipant (notice the
convention for types is capital letters, CamelCase) as having a name and
a field. The name will be a string and the field will be a Symbol type,
(defined by :Symbol, which we will go into plenty more detail later).

    \begin{Verbatim}[commandchars=\\\{\}]
{\color{incolor}In [{\color{incolor}52}]:} \PY{k}{type}\PY{n+nc}{ }\PY{n+nc}{WorkshopParticipant}
             \PY{n}{name}\PY{p}{:}\PY{p}{:}\PY{n}{String}
             \PY{n}{field}\PY{p}{:}\PY{p}{:}\PY{n}{Symbol}
         \PY{k}{end}
         \PY{n}{tony} \PY{o}{=} \PY{n}{WorkshopParticipant}\PY{p}{(}\PY{l+s}{\PYZdq{}}\PY{l+s}{Tony}\PY{l+s}{\PYZdq{}}\PY{p}{,}\PY{p}{:}\PY{n}{physics}\PY{p}{)}
\end{Verbatim}

            \begin{Verbatim}[commandchars=\\\{\}]
{\color{outcolor}Out[{\color{outcolor}52}]:} WorkshopParticipant("Tony",:physics)
\end{Verbatim}
        
    As with functions, types can be set ``parametrically''. For example, we
can have an StaffMember have a name and a field, but also an age. We can
allow this age to be any Number type as follows:

    \begin{Verbatim}[commandchars=\\\{\}]
{\color{incolor}In [{\color{incolor}1}]:} \PY{k}{type}\PY{n+nc}{ }\PY{n+nc}{StaffMember}\PY{p}{\PYZob{}}\PY{n}{T}\PY{o}{\PYZlt{}:}\PY{n}{Number}\PY{p}{\PYZcb{}}
            \PY{n}{name}\PY{p}{:}\PY{p}{:}\PY{n}{String}
            \PY{n}{field}\PY{p}{:}\PY{p}{:}\PY{n}{Symbol}
            \PY{n}{age}\PY{p}{:}\PY{p}{:}\PY{n}{T}
        \PY{k}{end}
        \PY{n}{ter} \PY{o}{=} \PY{n}{StaffMember}\PY{p}{(}\PY{l+s}{\PYZdq{}}\PY{l+s}{Terry}\PY{l+s}{\PYZdq{}}\PY{p}{,}\PY{p}{:}\PY{n}{football}\PY{p}{,}\PY{l+m+mi}{17}\PY{p}{)}
\end{Verbatim}

            \begin{Verbatim}[commandchars=\\\{\}]
{\color{outcolor}Out[{\color{outcolor}1}]:} StaffMember\{Int64\}("Terry",:football,17)
\end{Verbatim}
        
    The rules for parametric typing is the same as for functions. Note that
most of Julia's types, like Float64 and Int, are natively defined in
Julia in this manner. This means that there's no limit for user defined
types, only your imagination. Indeed, many of Julia's features first
start out as a prototyping package before it's ever moved into Base (the
Julia library that ships as the Base module in every installation).

Lastly, there exist abstract types. These types cannot be instantiated
but are used to build the type hierarchy. You've already seen one
abstract type, Number. We can define one for Person using the Abstract
keyword

    \begin{Verbatim}[commandchars=\\\{\}]
{\color{incolor}In [{\color{incolor}15}]:} \PY{k}{abstract}\PY{n+nc}{ }\PY{n+nc}{Person}
\end{Verbatim}

    Then we can set types as a subtype of person

    \begin{Verbatim}[commandchars=\\\{\}]
{\color{incolor}In [{\color{incolor}16}]:} \PY{k}{type}\PY{n+nc}{ }\PY{n+nc}{Student} \PY{o}{\PYZlt{}:} \PY{n}{Person}
             \PY{n}{name}
             \PY{n}{grade}
         \PY{k}{end}
\end{Verbatim}

    You can define type heirarchies on abstract types. See the beautiful
explanation at:
http://docs.julialang.org/en/release-0.5/manual/types/\#abstract-types

    \begin{Verbatim}[commandchars=\\\{\}]
{\color{incolor}In [{\color{incolor} }]:} \PY{k}{abstract}\PY{n+nc}{ }\PY{n+nc}{AbstractStudent} \PY{o}{\PYZlt{}:} \PY{n}{AbstractPerson}
\end{Verbatim}

    Another ``version'' of type is \texttt{immutable}. When one uses
\texttt{immutable}, the fields of the type cannot be changed. However,
Julia will automatically stack allocate immutable types, whereas
standard types are heap allocated. If this is unfamiliar terminology,
then think of this as meaning that immutable types are able to be stored
closer to the CPU and have less cost for memory access (this is a detail
not present in many scripting languages). Many things like Julia's
built-in Number types are defined as \texttt{immutable} in order to give
good performance.

    \begin{Verbatim}[commandchars=\\\{\}]
{\color{incolor}In [{\color{incolor}7}]:} \PY{k}{immutable} \PY{n}{Field}
            \PY{n}{name}
            \PY{n}{school}
        \PY{k}{end}
        \PY{n}{ds} \PY{o}{=} \PY{n}{Field}\PY{p}{(}\PY{p}{:}\PY{n}{DataScience}\PY{p}{,}\PY{p}{[}\PY{p}{:}\PY{n}{PhysicalScience}\PY{p}{;}\PY{p}{:}\PY{n}{ComputerScience}\PY{p}{]}\PY{p}{)}
\end{Verbatim}

            \begin{Verbatim}[commandchars=\\\{\}]
{\color{outcolor}Out[{\color{outcolor}7}]:} Field(:DataScience,Symbol[:PhysicalScience,:ComputerScience])
\end{Verbatim}
        
    \begin{center}\rule{3in}{0.4pt}\end{center}

\paragraph{Question 3}\label{question-3}

Can you explain this interesting quirk? Thus Field is immutable, meaning
that \texttt{ds.name} and \texttt{ds.school} cannot be changed:

    \begin{Verbatim}[commandchars=\\\{\}]
{\color{incolor}In [{\color{incolor}64}]:} \PY{n}{ds}\PY{o}{.}\PY{n}{name} \PY{o}{=} \PY{p}{:}\PY{n}{ComputationalStatistics}
\end{Verbatim}

    \begin{Verbatim}[commandchars=\\\{\}]

        LoadError: type Field is immutable
    while loading In[64], in expression starting on line 1

        

    \end{Verbatim}

    However, the following is allowed:

    \begin{Verbatim}[commandchars=\\\{\}]
{\color{incolor}In [{\color{incolor}65}]:} \PY{n}{push!}\PY{p}{(}\PY{n}{ds}\PY{o}{.}\PY{n}{school}\PY{p}{,}\PY{p}{:}\PY{n}{BiologicalScience}\PY{p}{)}
         \PY{n}{ds}\PY{o}{.}\PY{n}{school}
\end{Verbatim}

            \begin{Verbatim}[commandchars=\\\{\}]
{\color{outcolor}Out[{\color{outcolor}65}]:} 3-element Array\{Symbol,1\}:
          :PhysicalScience  
          :ComputerScience  
          :BiologicalScience
\end{Verbatim}
        
    (Hint: recall that an array is not the values itself, but a pointer to
the memory of the values)

\begin{center}\rule{3in}{0.4pt}\end{center}

    One important detail in Julia is that everything is a type (and every
piece of code is an Expression type, more on this later). Thus functions
are also types, which we can access the fields of. Not only is
everything compiled down to C, but all of the ``C parts'' are always
accessible. For example, we can, if we so choose, get a function
pointer:

    \begin{Verbatim}[commandchars=\\\{\}]
{\color{incolor}In [{\color{incolor}46}]:} \PY{n}{foo}\PY{p}{(}\PY{n}{x}\PY{p}{)} \PY{o}{=} \PY{l+m+mi}{2}\PY{n}{x}
         \PY{n}{first}\PY{p}{(}\PY{n}{methods}\PY{p}{(}\PY{n}{foo}\PY{p}{)}\PY{p}{)}\PY{o}{.}\PY{n}{lambda\PYZus{}template}\PY{o}{.}\PY{n}{fptr}
\end{Verbatim}

            \begin{Verbatim}[commandchars=\\\{\}]
{\color{outcolor}Out[{\color{outcolor}46}]:} Ptr\{Void\} @0x0000000000000000
\end{Verbatim}
        
    \begin{center}\rule{3in}{0.4pt}\end{center}

\paragraph{Problem 6}\label{problem-6}

Make a function \texttt{person\_info(x)} where, if \texttt{x} is a any
type of person, print their name. However, if \texttt{x} is a student,
print their name and their grade. Make a new type which is a graduate
student, and have it print their name and grade as well. If \texttt{x}
is anything else, throw an error. Do not using branching (\texttt{if}),
use multiple dispatch to solve the problem!

Note that in order to do this you will need to re-structure the type
hierarchy. Make an AbstractPerson and AbstractStudent type, define the
subclassing structure, and write dispatches on these abstract types.
Note that you cannot define subclasses of concrete types!

(Not only is the multiple-dispatch way more Julian, we will see later
that it also has a lot of performance enhancements due to how it
interacts with the compiler).

\paragraph{Distribution Quantile Problem (From Josh
Day)}\label{distribution-quantile-problem-from-josh-day}

To find the quantile of a number \texttt{q} in a distribution, one can
use a Newton method

\[ \theta_{n+1} = \theta_{n} - \frac{cdf(\theta)-q}{pdf(\theta)} \]

to have $\theta_{n} \rightarrow$ the value of for the \texttt{q}th
quantile. Use multiple dispatch to write a generic algorithm for which
calculates the \texttt{q}th quantile of any
\texttt{UnivariateDistribution} in Distributions.jl, and test your
result against the \texttt{quantile(d::UnivariateDistribution,q::Number}
function.

Hint: Use $\theta_{0} = $ mean of the distribution

\paragraph{Operator Problem}\label{operator-problem}

In mathematics, a matrix is known to be a linear operator. In many
cases, this can have huge performance advantages because, if you know a
function which ``acts like a matrix'' but does not form the matrix
itself, you can save the time that it takes to allocate the matrix
(sometimes the matrix may not fit in memory!)

Therefore, instead of solving regressions on matrices, let's be brave
and generalize our regression algorithm to work on any operator, and
make it solve the problem for the Laplacian operator. Here are the steps
that are required to do this:

\begin{itemize}
\itemsep1pt\parskip0pt\parsep0pt
\item
  Make an abstract type \texttt{AbstractOperator}
\item
  Re-define AbstractMatrix as a subtype
\item
  Define a concrete type \texttt{LaplacianOperator} which holds a
  function. This function \texttt{f(x)} should calculate the discrete
  Laplacian of \texttt{x} (i.e.~multiply it on the left by the Strang
  matrix, but without forming the matrix!)
\item
  Write dispatches for \texttt{*(::LaplacianOperator,::Vector)},
  \texttt{eltype(::LaplacianOperator)},
  \texttt{size(::LaplacianOperator,d::Integer)}.
\item
  Write \texttt{least\_square(::Operator,::Vector)} to solve the
  least-square approximation problem \texttt{Ax=b} for any operator.
  Note that since you do not have a matrix, you cannot use
  \texttt{\textbackslash{}} or factorizations like \texttt{qrfact}.
  Instead, take a look at \texttt{cg} in IterativeSolvers.jl
\item
  Test your least\_square function vs \texttt{llsq} and \texttt{lm}
  (where you use the data matrix as a Strang matrix). Do you get the
  same result?
\end{itemize}

\begin{center}\rule{3in}{0.4pt}\end{center}

    \subsection{Some Basic Types}\label{some-basic-types}

Julia provides many basic types. Indeed, you will come to know Julia as
a system of multiple dispatch on types, meaning that the interaction of
types with functions is core to the design.

\subsubsection{Lazy Iterator Types}\label{lazy-iterator-types}

While MATLAB or Python has easy functions for building arrays, Julia
tends to side-step the actual ``array'' part with specially made types.
One such example are ranges. To define a range, use the
\texttt{start:stepsize:end} syntax. For example:

    \begin{Verbatim}[commandchars=\\\{\}]
{\color{incolor}In [{\color{incolor}45}]:} \PY{n}{a} \PY{o}{=} \PY{l+m+mi}{1}\PY{p}{:}\PY{l+m+mi}{5}
         \PY{n}{println}\PY{p}{(}\PY{n}{a}\PY{p}{)}
         \PY{n}{b} \PY{o}{=} \PY{l+m+mi}{1}\PY{p}{:}\PY{l+m+mi}{2}\PY{p}{:}\PY{l+m+mi}{10}
         \PY{n}{println}\PY{p}{(}\PY{n}{b}\PY{p}{)}
\end{Verbatim}

    \begin{Verbatim}[commandchars=\\\{\}]
1:5
1:2:9

    \end{Verbatim}

    We can use them like any array. For example:

    \begin{Verbatim}[commandchars=\\\{\}]
{\color{incolor}In [{\color{incolor}47}]:} \PY{n}{println}\PY{p}{(}\PY{n}{a}\PY{p}{[}\PY{l+m+mi}{2}\PY{p}{]}\PY{p}{)}\PY{p}{;} \PY{n}{println}\PY{p}{(}\PY{n}{b}\PY{p}{[}\PY{l+m+mi}{3}\PY{p}{]}\PY{p}{)}
\end{Verbatim}

    \begin{Verbatim}[commandchars=\\\{\}]
2
5

    \end{Verbatim}

    But what is \texttt{b}?

    \begin{Verbatim}[commandchars=\\\{\}]
{\color{incolor}In [{\color{incolor}50}]:} \PY{n}{println}\PY{p}{(}\PY{n+nb}{typeof}\PY{p}{(}\PY{n}{b}\PY{p}{)}\PY{p}{)}
\end{Verbatim}

    \begin{Verbatim}[commandchars=\\\{\}]
StepRange\{Int64,Int64\}

    \end{Verbatim}

    \texttt{b} isn't an array, it's a StepRange. A StepRange has the ability
to act like an array using its fields:

    \begin{Verbatim}[commandchars=\\\{\}]
{\color{incolor}In [{\color{incolor}52}]:} \PY{n}{fieldnames}\PY{p}{(}\PY{n}{StepRange}\PY{p}{)}
\end{Verbatim}

            \begin{Verbatim}[commandchars=\\\{\}]
{\color{outcolor}Out[{\color{outcolor}52}]:} 3-element Array\{Symbol,1\}:
          :start
          :step 
          :stop 
\end{Verbatim}
        
    Note that at any time we can get the array from these kinds of type via
the \texttt{collect} function:

    \begin{Verbatim}[commandchars=\\\{\}]
{\color{incolor}In [{\color{incolor}55}]:} \PY{n}{c} \PY{o}{=} \PY{n}{collect}\PY{p}{(}\PY{n}{a}\PY{p}{)}
\end{Verbatim}

            \begin{Verbatim}[commandchars=\\\{\}]
{\color{outcolor}Out[{\color{outcolor}55}]:} 5-element Array\{Int64,1\}:
          1
          2
          3
          4
          5
\end{Verbatim}
        
    The reason why lazy iterator types are preferred is that they do not do
the computations until it's absolutely necessary, and they take up much
less space. We can check this with \texttt{@time}:

    \begin{Verbatim}[commandchars=\\\{\}]
{\color{incolor}In [{\color{incolor}7}]:} \PY{p}{@}\PY{n}{time} \PY{n}{a} \PY{o}{=} \PY{l+m+mi}{1}\PY{p}{:}\PY{l+m+mi}{100000}
        \PY{p}{@}\PY{n}{time} \PY{n}{a} \PY{o}{=} \PY{l+m+mi}{1}\PY{p}{:}\PY{l+m+mi}{100}
        \PY{p}{@}\PY{n}{time} \PY{n}{b} \PY{o}{=} \PY{n}{collect}\PY{p}{(}\PY{l+m+mi}{1}\PY{p}{:}\PY{l+m+mi}{100000}\PY{p}{)}\PY{p}{;}
\end{Verbatim}

    \begin{Verbatim}[commandchars=\\\{\}]
  0.000005 seconds (6 allocations: 240 bytes)
  0.000003 seconds (5 allocations: 192 bytes)
  0.007849 seconds (189 allocations: 792.844 KB)

    \end{Verbatim}

    Notice that the amount of time the range takes is much shorter. This is
mostly because there is a lot less memory allocation needed: only a
\texttt{StepRange} is built, and all that holds is the three numbers.
However, \texttt{b} has to hold \texttt{100000} numbers, leading to the
huge difference.

    \subsubsection{Dictionaries}\label{dictionaries}

Another common type is the Dictionary. It allows you to access
(key,value) pairs in a named manner. For example:

    \begin{Verbatim}[commandchars=\\\{\}]
{\color{incolor}In [{\color{incolor}1}]:} \PY{n}{d} \PY{o}{=} \PY{n}{Dict}\PY{p}{(}\PY{p}{:}\PY{n}{test}\PY{o}{=}\PY{o}{\PYZgt{}}\PY{l+m+mi}{2}\PY{p}{,}\PY{l+s}{\PYZdq{}}\PY{l+s}{silly}\PY{l+s}{\PYZdq{}}\PY{o}{=}\PY{o}{\PYZgt{}}\PY{p}{:}\PY{n}{suit}\PY{p}{)}
        \PY{n}{println}\PY{p}{(}\PY{n}{d}\PY{p}{[}\PY{p}{:}\PY{n}{test}\PY{p}{]}\PY{p}{)}
        \PY{n}{println}\PY{p}{(}\PY{n}{d}\PY{p}{[}\PY{l+s}{\PYZdq{}}\PY{l+s}{silly}\PY{l+s}{\PYZdq{}}\PY{p}{]}\PY{p}{)}
\end{Verbatim}

    \begin{Verbatim}[commandchars=\\\{\}]
2
suit

    \end{Verbatim}

    \subsubsection{Tuples}\label{tuples}

Tuples are immutable arrays. That means they can't be changed. However,
they are super fast. They are made with the \texttt{(x,y,z,...)} syntax
and are the standard return type of functions which return more than one
object.

    \begin{Verbatim}[commandchars=\\\{\}]
{\color{incolor}In [{\color{incolor}4}]:} \PY{n}{tup} \PY{o}{=} \PY{p}{(}\PY{l+m+mf}{2.}\PY{p}{,}\PY{l+m+mi}{3}\PY{p}{)} \PY{c}{\PYZsh{} Don\PYZsq{}t have to match types}
        \PY{n}{x}\PY{p}{,}\PY{n}{y} \PY{o}{=} \PY{p}{(}\PY{l+m+mf}{3.0}\PY{p}{,}\PY{l+s}{\PYZdq{}}\PY{l+s}{hi}\PY{l+s}{\PYZdq{}}\PY{p}{)} \PY{c}{\PYZsh{} Can separate a tuple to multiple variables}
\end{Verbatim}

            \begin{Verbatim}[commandchars=\\\{\}]
{\color{outcolor}Out[{\color{outcolor}4}]:} (3.0,"hi")
\end{Verbatim}
        
    \paragraph{Problem 8}\label{problem-8}

If you know \texttt{start}, \texttt{step}, and \texttt{stop}, how do you
calculate the \texttt{i}th value? Can you create a function MyRange
which where for \texttt{a} being a \texttt{MyRange}, and
\texttt{a{[}i{]}} is the correct value? Use the Julia array interface in
order to define the function for the \texttt{a{[}i{]}} syntax on your
type.

\paragraph{Problem 9}\label{problem-9}

Do ?linspace. Make your own LinSpace object using the array interface.

http://ucidatascienceinitiative.github.io/IntroToJulia/Html/ArrayIteratorInterfaces

Do your implementations obay dimensional analysis? Try using the package
\texttt{Unitful} to build arrays of numbers with units (i.e.~an array of
numbers who have values of Newtons), and see if you can make your
LinSpace not give errors.

\paragraph{Problem 10}\label{problem-10}

Check your implementation vs the source code of Ranges.jl. Tim Holy is
the master of Julia arrays, learn from him!

\paragraph{Problem 11}\label{problem-11}

Check out the call overloading notebook:

http://ucidatascienceinitiative.github.io/IntroToJulia/Html/CallOverloading

Overload the call on the UnitStepRange to give an interpolated value at
intermediate points, i.e.~if \texttt{a=1:2:10}, then \texttt{a(1.5)=2}.

    \subsection{Metaprogramming}\label{metaprogramming}

Metaprogramming is a huge feature of Julia. The key idea is that every
statement in Julia is of the type \texttt{Expression}. Julia operators
by building an Abstract Syntax Tree (AST) from the Expressions. You've
already been exposed to this a little bit: a \texttt{Symbol} (like
\texttt{:PhysicalSciences} is not a string because it is part of the
AST, and thus is part of the parsing/expression structure. One
interesting thing is that symbol comparisons are O(1) while string
comparisons, like always, are O(n)) is part of this, and macros (the
weird functions with an \texttt{@}) are functions on expressions.

Thus you can think of metaprogramming as ``code which takes in code and
outputs code''. One basic example is the \texttt{@time} macro:

    \begin{Verbatim}[commandchars=\\\{\}]
{\color{incolor}In [{\color{incolor}85}]:} \PY{k}{macro} \PY{n}{my\PYZus{}time}\PY{p}{(}\PY{n}{ex}\PY{p}{)}
           \PY{k}{return} \PY{k}{quote}
             \PY{k+kd}{local} \PY{n}{t0} \PY{o}{=} \PY{n}{time}\PY{p}{(}\PY{p}{)}
             \PY{k+kd}{local} \PY{n}{val} \PY{o}{=} \PY{o}{\PYZdl{}}\PY{n}{ex}
             \PY{k+kd}{local} \PY{n}{t1} \PY{o}{=} \PY{n}{time}\PY{p}{(}\PY{p}{)}
             \PY{n}{println}\PY{p}{(}\PY{l+s}{\PYZdq{}}\PY{l+s}{elapsed time: }\PY{l+s}{\PYZdq{}}\PY{p}{,} \PY{n}{t1}\PY{o}{\PYZhy{}}\PY{n}{t0}\PY{p}{,} \PY{l+s}{\PYZdq{}}\PY{l+s}{ seconds}\PY{l+s}{\PYZdq{}}\PY{p}{)}
             \PY{n}{val}
           \PY{k}{end}
         \PY{k}{end}
\end{Verbatim}

    \begin{Verbatim}[commandchars=\\\{\}]

        LoadError: error in method definition: function Base.@time must be explicitly imported to be extended
    while loading In[85], in expression starting on line 1

        

    \end{Verbatim}

    This takes in an expression \texttt{ex}, gets the time before and after
evaluation, and prints the elapsed time between (the real time macro
also calculates the allocations as seen earlier). Note that
\texttt{\$ex} ``interpolates'' the expression into the macro. Going into
detail on metaprogramming is a large step from standard scripting and
will be a later session.

Why macros? One reason is because it lets you define any syntax you
want. Since it operates on the expressions themselves, as long as you
know how to parse the expression into working code, you can ``choose any
syntax'' to be your syntax. A case study will be shown later. Another
reason is because these are done at ``parse time'' and those are only
called once (before the function compilation).

    \subsection{Steps for Julia Parsing and
Execution}\label{steps-for-julia-parsing-and-execution}

\begin{enumerate}
\def\labelenumi{\arabic{enumi}.}
\itemsep1pt\parskip0pt\parsep0pt
\item
  The AST after parsing \textless{}- Macros
\item
  The AST after lowering (@code\_typed)
\item
  The AST after type inference and optimization \textless{}- Generated
  Functions (@code\_lowered)
\item
  The LLVM IR \textless{}- Functions (@code\_llvm)
\item
  The assembly code (@code\_native)
\end{enumerate}


    % Add a bibliography block to the postdoc
    
    
    
    \end{document}
