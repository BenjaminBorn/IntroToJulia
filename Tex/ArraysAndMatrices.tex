
% Default to the notebook output style

    


% Inherit from the specified cell style.




    
\documentclass[11pt]{article}

    
    
    \usepackage[T1]{fontenc}
    % Nicer default font than Computer Modern for most use cases
    \usepackage{palatino}

    % Basic figure setup, for now with no caption control since it's done
    % automatically by Pandoc (which extracts ![](path) syntax from Markdown).
    \usepackage{graphicx}
    % We will generate all images so they have a width \maxwidth. This means
    % that they will get their normal width if they fit onto the page, but
    % are scaled down if they would overflow the margins.
    \makeatletter
    \def\maxwidth{\ifdim\Gin@nat@width>\linewidth\linewidth
    \else\Gin@nat@width\fi}
    \makeatother
    \let\Oldincludegraphics\includegraphics
    % Set max figure width to be 80% of text width, for now hardcoded.
    \renewcommand{\includegraphics}[1]{\Oldincludegraphics[width=.8\maxwidth]{#1}}
    % Ensure that by default, figures have no caption (until we provide a
    % proper Figure object with a Caption API and a way to capture that
    % in the conversion process - todo).
    \usepackage{caption}
    \DeclareCaptionLabelFormat{nolabel}{}
    \captionsetup{labelformat=nolabel}

    \usepackage{adjustbox} % Used to constrain images to a maximum size 
    \usepackage{xcolor} % Allow colors to be defined
    \usepackage{enumerate} % Needed for markdown enumerations to work
    \usepackage{geometry} % Used to adjust the document margins
    \usepackage{amsmath} % Equations
    \usepackage{amssymb} % Equations
    \usepackage{textcomp} % defines textquotesingle
    % Hack from http://tex.stackexchange.com/a/47451/13684:
    \AtBeginDocument{%
        \def\PYZsq{\textquotesingle}% Upright quotes in Pygmentized code
    }
    \usepackage{upquote} % Upright quotes for verbatim code
    \usepackage{eurosym} % defines \euro
    \usepackage[mathletters]{ucs} % Extended unicode (utf-8) support
    \usepackage[utf8x]{inputenc} % Allow utf-8 characters in the tex document
    \usepackage{fancyvrb} % verbatim replacement that allows latex
    \usepackage{grffile} % extends the file name processing of package graphics 
                         % to support a larger range 
    % The hyperref package gives us a pdf with properly built
    % internal navigation ('pdf bookmarks' for the table of contents,
    % internal cross-reference links, web links for URLs, etc.)
    \usepackage{hyperref}
    \usepackage{longtable} % longtable support required by pandoc >1.10
    \usepackage{booktabs}  % table support for pandoc > 1.12.2
    \usepackage[normalem]{ulem} % ulem is needed to support strikethroughs (\sout)
                                % normalem makes italics be italics, not underlines
    

    
    
    % Colors for the hyperref package
    \definecolor{urlcolor}{rgb}{0,.145,.698}
    \definecolor{linkcolor}{rgb}{.71,0.21,0.01}
    \definecolor{citecolor}{rgb}{.12,.54,.11}

    % ANSI colors
    \definecolor{ansi-black}{HTML}{3E424D}
    \definecolor{ansi-black-intense}{HTML}{282C36}
    \definecolor{ansi-red}{HTML}{E75C58}
    \definecolor{ansi-red-intense}{HTML}{B22B31}
    \definecolor{ansi-green}{HTML}{00A250}
    \definecolor{ansi-green-intense}{HTML}{007427}
    \definecolor{ansi-yellow}{HTML}{DDB62B}
    \definecolor{ansi-yellow-intense}{HTML}{B27D12}
    \definecolor{ansi-blue}{HTML}{208FFB}
    \definecolor{ansi-blue-intense}{HTML}{0065CA}
    \definecolor{ansi-magenta}{HTML}{D160C4}
    \definecolor{ansi-magenta-intense}{HTML}{A03196}
    \definecolor{ansi-cyan}{HTML}{60C6C8}
    \definecolor{ansi-cyan-intense}{HTML}{258F8F}
    \definecolor{ansi-white}{HTML}{C5C1B4}
    \definecolor{ansi-white-intense}{HTML}{A1A6B2}

    % commands and environments needed by pandoc snippets
    % extracted from the output of `pandoc -s`
    \providecommand{\tightlist}{%
      \setlength{\itemsep}{0pt}\setlength{\parskip}{0pt}}
    \DefineVerbatimEnvironment{Highlighting}{Verbatim}{commandchars=\\\{\}}
    % Add ',fontsize=\small' for more characters per line
    \newenvironment{Shaded}{}{}
    \newcommand{\KeywordTok}[1]{\textcolor[rgb]{0.00,0.44,0.13}{\textbf{{#1}}}}
    \newcommand{\DataTypeTok}[1]{\textcolor[rgb]{0.56,0.13,0.00}{{#1}}}
    \newcommand{\DecValTok}[1]{\textcolor[rgb]{0.25,0.63,0.44}{{#1}}}
    \newcommand{\BaseNTok}[1]{\textcolor[rgb]{0.25,0.63,0.44}{{#1}}}
    \newcommand{\FloatTok}[1]{\textcolor[rgb]{0.25,0.63,0.44}{{#1}}}
    \newcommand{\CharTok}[1]{\textcolor[rgb]{0.25,0.44,0.63}{{#1}}}
    \newcommand{\StringTok}[1]{\textcolor[rgb]{0.25,0.44,0.63}{{#1}}}
    \newcommand{\CommentTok}[1]{\textcolor[rgb]{0.38,0.63,0.69}{\textit{{#1}}}}
    \newcommand{\OtherTok}[1]{\textcolor[rgb]{0.00,0.44,0.13}{{#1}}}
    \newcommand{\AlertTok}[1]{\textcolor[rgb]{1.00,0.00,0.00}{\textbf{{#1}}}}
    \newcommand{\FunctionTok}[1]{\textcolor[rgb]{0.02,0.16,0.49}{{#1}}}
    \newcommand{\RegionMarkerTok}[1]{{#1}}
    \newcommand{\ErrorTok}[1]{\textcolor[rgb]{1.00,0.00,0.00}{\textbf{{#1}}}}
    \newcommand{\NormalTok}[1]{{#1}}
    
    % Additional commands for more recent versions of Pandoc
    \newcommand{\ConstantTok}[1]{\textcolor[rgb]{0.53,0.00,0.00}{{#1}}}
    \newcommand{\SpecialCharTok}[1]{\textcolor[rgb]{0.25,0.44,0.63}{{#1}}}
    \newcommand{\VerbatimStringTok}[1]{\textcolor[rgb]{0.25,0.44,0.63}{{#1}}}
    \newcommand{\SpecialStringTok}[1]{\textcolor[rgb]{0.73,0.40,0.53}{{#1}}}
    \newcommand{\ImportTok}[1]{{#1}}
    \newcommand{\DocumentationTok}[1]{\textcolor[rgb]{0.73,0.13,0.13}{\textit{{#1}}}}
    \newcommand{\AnnotationTok}[1]{\textcolor[rgb]{0.38,0.63,0.69}{\textbf{\textit{{#1}}}}}
    \newcommand{\CommentVarTok}[1]{\textcolor[rgb]{0.38,0.63,0.69}{\textbf{\textit{{#1}}}}}
    \newcommand{\VariableTok}[1]{\textcolor[rgb]{0.10,0.09,0.49}{{#1}}}
    \newcommand{\ControlFlowTok}[1]{\textcolor[rgb]{0.00,0.44,0.13}{\textbf{{#1}}}}
    \newcommand{\OperatorTok}[1]{\textcolor[rgb]{0.40,0.40,0.40}{{#1}}}
    \newcommand{\BuiltInTok}[1]{{#1}}
    \newcommand{\ExtensionTok}[1]{{#1}}
    \newcommand{\PreprocessorTok}[1]{\textcolor[rgb]{0.74,0.48,0.00}{{#1}}}
    \newcommand{\AttributeTok}[1]{\textcolor[rgb]{0.49,0.56,0.16}{{#1}}}
    \newcommand{\InformationTok}[1]{\textcolor[rgb]{0.38,0.63,0.69}{\textbf{\textit{{#1}}}}}
    \newcommand{\WarningTok}[1]{\textcolor[rgb]{0.38,0.63,0.69}{\textbf{\textit{{#1}}}}}
    
    
    % Define a nice break command that doesn't care if a line doesn't already
    % exist.
    \def\br{\hspace*{\fill} \\* }
    % Math Jax compatability definitions
    \def\gt{>}
    \def\lt{<}
    % Document parameters
    \title{ArraysAndMatrices}
    
    
    

    % Pygments definitions
    
\makeatletter
\def\PY@reset{\let\PY@it=\relax \let\PY@bf=\relax%
    \let\PY@ul=\relax \let\PY@tc=\relax%
    \let\PY@bc=\relax \let\PY@ff=\relax}
\def\PY@tok#1{\csname PY@tok@#1\endcsname}
\def\PY@toks#1+{\ifx\relax#1\empty\else%
    \PY@tok{#1}\expandafter\PY@toks\fi}
\def\PY@do#1{\PY@bc{\PY@tc{\PY@ul{%
    \PY@it{\PY@bf{\PY@ff{#1}}}}}}}
\def\PY#1#2{\PY@reset\PY@toks#1+\relax+\PY@do{#2}}

\def\PY@tok@gd{\def\PY@tc##1{\textcolor[rgb]{0.63,0.00,0.00}{##1}}}
\def\PY@tok@gu{\let\PY@bf=\textbf\def\PY@tc##1{\textcolor[rgb]{0.50,0.00,0.50}{##1}}}
\def\PY@tok@gt{\def\PY@tc##1{\textcolor[rgb]{0.00,0.25,0.82}{##1}}}
\def\PY@tok@gs{\let\PY@bf=\textbf}
\def\PY@tok@gr{\def\PY@tc##1{\textcolor[rgb]{1.00,0.00,0.00}{##1}}}
\def\PY@tok@cm{\let\PY@it=\textit\def\PY@tc##1{\textcolor[rgb]{0.25,0.50,0.50}{##1}}}
\def\PY@tok@vg{\def\PY@tc##1{\textcolor[rgb]{0.10,0.09,0.49}{##1}}}
\def\PY@tok@m{\def\PY@tc##1{\textcolor[rgb]{0.40,0.40,0.40}{##1}}}
\def\PY@tok@mh{\def\PY@tc##1{\textcolor[rgb]{0.40,0.40,0.40}{##1}}}
\def\PY@tok@go{\def\PY@tc##1{\textcolor[rgb]{0.50,0.50,0.50}{##1}}}
\def\PY@tok@ge{\let\PY@it=\textit}
\def\PY@tok@vc{\def\PY@tc##1{\textcolor[rgb]{0.10,0.09,0.49}{##1}}}
\def\PY@tok@il{\def\PY@tc##1{\textcolor[rgb]{0.40,0.40,0.40}{##1}}}
\def\PY@tok@cs{\let\PY@it=\textit\def\PY@tc##1{\textcolor[rgb]{0.25,0.50,0.50}{##1}}}
\def\PY@tok@cp{\def\PY@tc##1{\textcolor[rgb]{0.74,0.48,0.00}{##1}}}
\def\PY@tok@gi{\def\PY@tc##1{\textcolor[rgb]{0.00,0.63,0.00}{##1}}}
\def\PY@tok@gh{\let\PY@bf=\textbf\def\PY@tc##1{\textcolor[rgb]{0.00,0.00,0.50}{##1}}}
\def\PY@tok@ni{\let\PY@bf=\textbf\def\PY@tc##1{\textcolor[rgb]{0.60,0.60,0.60}{##1}}}
\def\PY@tok@nl{\def\PY@tc##1{\textcolor[rgb]{0.63,0.63,0.00}{##1}}}
\def\PY@tok@nn{\let\PY@bf=\textbf\def\PY@tc##1{\textcolor[rgb]{0.00,0.00,1.00}{##1}}}
\def\PY@tok@no{\def\PY@tc##1{\textcolor[rgb]{0.53,0.00,0.00}{##1}}}
\def\PY@tok@na{\def\PY@tc##1{\textcolor[rgb]{0.49,0.56,0.16}{##1}}}
\def\PY@tok@nb{\def\PY@tc##1{\textcolor[rgb]{0.00,0.50,0.00}{##1}}}
\def\PY@tok@nc{\let\PY@bf=\textbf\def\PY@tc##1{\textcolor[rgb]{0.00,0.00,1.00}{##1}}}
\def\PY@tok@nd{\def\PY@tc##1{\textcolor[rgb]{0.67,0.13,1.00}{##1}}}
\def\PY@tok@ne{\let\PY@bf=\textbf\def\PY@tc##1{\textcolor[rgb]{0.82,0.25,0.23}{##1}}}
\def\PY@tok@nf{\def\PY@tc##1{\textcolor[rgb]{0.00,0.00,1.00}{##1}}}
\def\PY@tok@si{\let\PY@bf=\textbf\def\PY@tc##1{\textcolor[rgb]{0.73,0.40,0.53}{##1}}}
\def\PY@tok@s2{\def\PY@tc##1{\textcolor[rgb]{0.73,0.13,0.13}{##1}}}
\def\PY@tok@vi{\def\PY@tc##1{\textcolor[rgb]{0.10,0.09,0.49}{##1}}}
\def\PY@tok@nt{\let\PY@bf=\textbf\def\PY@tc##1{\textcolor[rgb]{0.00,0.50,0.00}{##1}}}
\def\PY@tok@nv{\def\PY@tc##1{\textcolor[rgb]{0.10,0.09,0.49}{##1}}}
\def\PY@tok@s1{\def\PY@tc##1{\textcolor[rgb]{0.73,0.13,0.13}{##1}}}
\def\PY@tok@sh{\def\PY@tc##1{\textcolor[rgb]{0.73,0.13,0.13}{##1}}}
\def\PY@tok@sc{\def\PY@tc##1{\textcolor[rgb]{0.73,0.13,0.13}{##1}}}
\def\PY@tok@sx{\def\PY@tc##1{\textcolor[rgb]{0.00,0.50,0.00}{##1}}}
\def\PY@tok@bp{\def\PY@tc##1{\textcolor[rgb]{0.00,0.50,0.00}{##1}}}
\def\PY@tok@c1{\let\PY@it=\textit\def\PY@tc##1{\textcolor[rgb]{0.25,0.50,0.50}{##1}}}
\def\PY@tok@kc{\let\PY@bf=\textbf\def\PY@tc##1{\textcolor[rgb]{0.00,0.50,0.00}{##1}}}
\def\PY@tok@c{\let\PY@it=\textit\def\PY@tc##1{\textcolor[rgb]{0.25,0.50,0.50}{##1}}}
\def\PY@tok@mf{\def\PY@tc##1{\textcolor[rgb]{0.40,0.40,0.40}{##1}}}
\def\PY@tok@err{\def\PY@bc##1{\fcolorbox[rgb]{1.00,0.00,0.00}{1,1,1}{##1}}}
\def\PY@tok@kd{\let\PY@bf=\textbf\def\PY@tc##1{\textcolor[rgb]{0.00,0.50,0.00}{##1}}}
\def\PY@tok@ss{\def\PY@tc##1{\textcolor[rgb]{0.10,0.09,0.49}{##1}}}
\def\PY@tok@sr{\def\PY@tc##1{\textcolor[rgb]{0.73,0.40,0.53}{##1}}}
\def\PY@tok@mo{\def\PY@tc##1{\textcolor[rgb]{0.40,0.40,0.40}{##1}}}
\def\PY@tok@kn{\let\PY@bf=\textbf\def\PY@tc##1{\textcolor[rgb]{0.00,0.50,0.00}{##1}}}
\def\PY@tok@mi{\def\PY@tc##1{\textcolor[rgb]{0.40,0.40,0.40}{##1}}}
\def\PY@tok@gp{\let\PY@bf=\textbf\def\PY@tc##1{\textcolor[rgb]{0.00,0.00,0.50}{##1}}}
\def\PY@tok@o{\def\PY@tc##1{\textcolor[rgb]{0.40,0.40,0.40}{##1}}}
\def\PY@tok@kr{\let\PY@bf=\textbf\def\PY@tc##1{\textcolor[rgb]{0.00,0.50,0.00}{##1}}}
\def\PY@tok@s{\def\PY@tc##1{\textcolor[rgb]{0.73,0.13,0.13}{##1}}}
\def\PY@tok@kp{\def\PY@tc##1{\textcolor[rgb]{0.00,0.50,0.00}{##1}}}
\def\PY@tok@w{\def\PY@tc##1{\textcolor[rgb]{0.73,0.73,0.73}{##1}}}
\def\PY@tok@kt{\def\PY@tc##1{\textcolor[rgb]{0.69,0.00,0.25}{##1}}}
\def\PY@tok@ow{\let\PY@bf=\textbf\def\PY@tc##1{\textcolor[rgb]{0.67,0.13,1.00}{##1}}}
\def\PY@tok@sb{\def\PY@tc##1{\textcolor[rgb]{0.73,0.13,0.13}{##1}}}
\def\PY@tok@k{\let\PY@bf=\textbf\def\PY@tc##1{\textcolor[rgb]{0.00,0.50,0.00}{##1}}}
\def\PY@tok@se{\let\PY@bf=\textbf\def\PY@tc##1{\textcolor[rgb]{0.73,0.40,0.13}{##1}}}
\def\PY@tok@sd{\let\PY@it=\textit\def\PY@tc##1{\textcolor[rgb]{0.73,0.13,0.13}{##1}}}

\def\PYZbs{\char`\\}
\def\PYZus{\char`\_}
\def\PYZob{\char`\{}
\def\PYZcb{\char`\}}
\def\PYZca{\char`\^}
\def\PYZsh{\char`\#}
\def\PYZpc{\char`\%}
\def\PYZdl{\char`\$}
\def\PYZti{\char`\~}
% for compatibility with earlier versions
\def\PYZat{@}
\def\PYZlb{[}
\def\PYZrb{]}
\makeatother


    % Exact colors from NB
    \definecolor{incolor}{rgb}{0.0, 0.0, 0.5}
    \definecolor{outcolor}{rgb}{0.545, 0.0, 0.0}



    
    % Prevent overflowing lines due to hard-to-break entities
    \sloppy 
    % Setup hyperref package
    \hypersetup{
      breaklinks=true,  % so long urls are correctly broken across lines
      colorlinks=true,
      urlcolor=urlcolor,
      linkcolor=linkcolor,
      citecolor=citecolor,
      }
    % Slightly bigger margins than the latex defaults
    
    \geometry{verbose,tmargin=1in,bmargin=1in,lmargin=1in,rmargin=1in}
    
    

    \begin{document}
    
    
    \maketitle
    
    

    
    \section{More Details on Arrays and
Matrices}\label{more-details-on-arrays-and-matrices}

    \subsection{Array Operations}\label{array-operations}

    One key feature for doing math in Julia are the \texttt{broadcast} and
\texttt{map} operations. The \texttt{map} operation is like an R or
MATLAB \texttt{apply} which applies a function to each element of an
array. For example, we can apply the anonymous function
\texttt{(x)-\textgreater{}x\^{}2} to each element via:

    \begin{Verbatim}[commandchars=\\\{\}]
{\color{incolor}In [{\color{incolor}5}]:} map((x)->x\PYZca{}2,1:5)
\end{Verbatim}

            \begin{Verbatim}[commandchars=\\\{\}]
{\color{outcolor}Out[{\color{outcolor}5}]:} 5-element Array\{Int64,1\}:
          1
          4
          9
         16
         25
\end{Verbatim}
        
    The \texttt{broadcast} operation is similar except it is for ``elements
which have a shape'' and it will ``broadcast the shaped objects to a
common size by expanding singleton dimensions''. For example, the
following broadcast's \texttt{+} onto \texttt{A} and \texttt{B}:

    \begin{Verbatim}[commandchars=\\\{\}]
{\color{incolor}In [{\color{incolor}8}]:} A = 1:5 \PYZsh{} Acts like a column vector, Julia is "column-major" so columns come first
        B = [1 2
             3 4
             5 6
             7 8
             9 10]
        broadcast(+,A,B)
\end{Verbatim}

            \begin{Verbatim}[commandchars=\\\{\}]
{\color{outcolor}Out[{\color{outcolor}8}]:} 5×2 Array\{Int64,2\}:
          2   3
          5   6
          8   9
         11  12
         14  15
\end{Verbatim}
        
    If \texttt{A} and \texttt{B} are the same size, then broadcasting is the
same as mapping.

    \subsubsection{In-Depth Julia 1: Special Broadcasting
Syntax}\label{in-depth-julia-1-special-broadcasting-syntax}

One major area (which is under a lot of active development) is the
specialized broadcasting syntax. The short summary is, putting a
\texttt{.} with a function or operator causes it to broadcast. For
example, we can broadcast any function with the syntax \texttt{f.(x)},
and broadcast operators by \texttt{.+} and related. For example:

    \begin{Verbatim}[commandchars=\\\{\}]
{\color{incolor}In [{\color{incolor}2}]:} A = 1:5
        B = [2;3;4;5;6]
        A.*B
\end{Verbatim}

            \begin{Verbatim}[commandchars=\\\{\}]
{\color{outcolor}Out[{\color{outcolor}2}]:} 5-element Array\{Int64,1\}:
          2
          6
         12
         20
         30
\end{Verbatim}
        
    People coming from MATLAB might recognize this as ``element-wise
multiplication''. If this was a basic introduction to Julia, I'd say
this was element-wise multiplication and be done with it. However, this
is the non-trivial introduction. {[}Note: Some of this is not true right
now (v0.5) but is becoming true\ldots{}{]}.

While it looks the same to the user, the implementation is very
different In MATLAB and earlier versions of Julia, \texttt{.*} was an
operator. In Julia's more refined world, we can explain this as
\texttt{.*\{T\textless{}:Number,N\}(x::Array\{T,N\},y::Array\{T,N\})}
being a function, and \texttt{A.*B} calling this function. However, if
\texttt{.*} is just a function, then

    \begin{Verbatim}[commandchars=\\\{\}]
{\color{incolor}In [{\color{incolor}3}]:} C = [3;4;5;2;1]
        A.*B.*C
\end{Verbatim}

            \begin{Verbatim}[commandchars=\\\{\}]
{\color{outcolor}Out[{\color{outcolor}3}]:} 5-element Array\{Int64,1\}:
          6
         24
         60
         40
         30
\end{Verbatim}
        
    the operation \texttt{A.*B.*C} actually expands into
\texttt{.*(A,.*(B,C))}. Let's think of how we'd implement \texttt{.*}.

\begin{center}\rule{3in}{0.4pt}\end{center}

\paragraph{Question 1}\label{question-1}

How would you implement \texttt{.*} as a function (not using broadcast)?
Don't peak below!

\begin{center}\rule{3in}{0.4pt}\end{center}

    \begin{Verbatim}[commandchars=\\\{\}]
{\color{incolor}In [{\color{incolor}13}]:} function .*\PYZob{}T<:Number,N\PYZcb{}(x::Array\PYZob{}T,N\PYZcb{},y::Array\PYZob{}T,N\PYZcb{})
             output = similar(x) \PYZsh{} Makes an array of similar size and shape as x
             for i in eachindex(x) \PYZsh{} Let the iterator choose the fast linear indexing for x
                 output[i] = x[i]*y[i]
             end
             output
         end
\end{Verbatim}

    \begin{Verbatim}[commandchars=\\\{\}]

        LoadError: error in method definition: function Base..* must be explicitly imported to be extended
    while loading In[13], in expression starting on line 1

        

    \end{Verbatim}

    Notice that \texttt{.*} creates an array every time it is called.
Therefore a naive approach where \texttt{.*} is a function creates two
arrays in the call \texttt{A.*B.*C}. We saw earlier that reducing memory
allocations leads to vastly improved performance, so a better
implementation would be to do this all together as one loop:

    \begin{Verbatim}[commandchars=\\\{\}]
{\color{incolor}In [{\color{incolor}16}]:} function .*\PYZob{}T<:Number,N\PYZcb{}(x::Array\PYZob{}T,N\PYZcb{},y::Array\PYZob{}T,N\PYZcb{},z::Array\PYZob{}T,N\PYZcb{})
             output = similar(x) \PYZsh{} Makes an array of similar size and shape as x
             for i in eachindex(x) \PYZsh{} Let the iterator choose the fast linear indexing for x
                 output[i] = x[i]*y[i]*z[i]
             end
             output
         end
\end{Verbatim}

    \begin{Verbatim}[commandchars=\\\{\}]

        LoadError: error in method definition: function Base..* must be explicitly imported to be extended
    while loading In[16], in expression starting on line 1

        

    \end{Verbatim}

    (but notice this doesn't really work because now \texttt{.*} isn't a
binary operator and therefore the inline syntax won't work). This
optimization is known as ``loop fusing''. Julia does this by searching
for all of the broadcasts in a line and putting them together into one
broadcast statement during parsing time. Therefore, in Julia
\texttt{A.*B.*C} creates an anonymous function and broadcasts on it,
like

    \begin{Verbatim}[commandchars=\\\{\}]
{\color{incolor}In [{\color{incolor}17}]:} broadcast((x,y,z)->x*y*z,A,B,C)
\end{Verbatim}

            \begin{Verbatim}[commandchars=\\\{\}]
{\color{outcolor}Out[{\color{outcolor}17}]:} 5-element Array\{Int64,1\}:
           6
          24
          60
          40
          30
\end{Verbatim}
        
    Notice that this is equivalent to our 1-loop solution. However, because
all array-based math uses this broadcasting syntax with a \texttt{.},
Julia can fuse the broadcasts on all sorts of mathematical expressions
on arrays:

    \begin{Verbatim}[commandchars=\\\{\}]
{\color{incolor}In [{\color{incolor}7}]:} A.*B.*sin.(C)
\end{Verbatim}

            \begin{Verbatim}[commandchars=\\\{\}]
{\color{outcolor}Out[{\color{outcolor}7}]:} 5-element Array\{Float64,1\}:
           0.28224
          -4.54081
         -11.5071 
          18.1859 
          25.2441 
\end{Verbatim}
        
    One last thing to note is that we can also broadcast \texttt{=}. This
would be the same thing is as the loop \texttt{A{[}i{]} = ...} and thus
requires the array \texttt{A} to already be define. Thus for example, if
we let

    \begin{Verbatim}[commandchars=\\\{\}]
{\color{incolor}In [{\color{incolor}12}]:} D = similar(C)
\end{Verbatim}

            \begin{Verbatim}[commandchars=\\\{\}]
{\color{outcolor}Out[{\color{outcolor}12}]:} 5-element Array\{Int64,1\}:
          140539342502768
          140539342502832
          140539342502896
          140539342502672
          140539342537392
\end{Verbatim}
        
    then the operation

    \begin{Verbatim}[commandchars=\\\{\}]
{\color{incolor}In [{\color{incolor}16}]:} @time D.=A.*B.*C
\end{Verbatim}

    \begin{Verbatim}[commandchars=\\\{\}]
  0.000008 seconds (10 allocations: 512 bytes)

    \end{Verbatim}

            \begin{Verbatim}[commandchars=\\\{\}]
{\color{outcolor}Out[{\color{outcolor}16}]:} 5-element Array\{Int64,1\}:
           6
          24
          60
          40
          30
\end{Verbatim}
        
    does not allocate any arrays. Reducing temporary array allocations is
one way Julia outperforms other scientific computing languages.

Summary: \texttt{.} makes operations element-wise, but in a very smart
way.

    \subsection{Vectors, Matrices, and Linear
Algebra}\label{vectors-matrices-and-linear-algebra}

Julia's linear algebra syntax follows MATLAB's to a large extent (it's
just about the only thing MATLAB got right!). We already saw this a
little bit by seeing Julia's array indexing syntax. For example, we can
get the first three elements by \texttt{1:3}:

    \begin{Verbatim}[commandchars=\\\{\}]
{\color{incolor}In [{\color{incolor}20}]:} A = rand(4,4) \PYZsh{} Generate a 4x4 random matrix
         A[1:3,1:3] \PYZsh{} Take the top left 3-3 matrix
\end{Verbatim}

            \begin{Verbatim}[commandchars=\\\{\}]
{\color{outcolor}Out[{\color{outcolor}20}]:} 3×3 Array\{Float64,2\}:
          0.168453  0.121436  0.395455
          0.531294  0.545059  0.08028 
          0.813272  0.732868  0.336742
\end{Verbatim}
        
    Note that Julia is column-major, meaning that columns come first in both
indexing order and in the computer's internal representation.

    \subsubsection{In-Depth Julia 2: Views}\label{in-depth-julia-2-views}

Notice that \texttt{A{[}1:3,1:3{]}} returned an array. Where did this
array come from? Well, since there was no 3x3 array before,
\texttt{A{[}1:3,1:3{]}} created an array (i.e.~it had to allocate
memory)

    \begin{Verbatim}[commandchars=\\\{\}]
{\color{incolor}In [{\color{incolor}22}]:} @time A[1:3,1:3]
\end{Verbatim}

    \begin{Verbatim}[commandchars=\\\{\}]
  0.000005 seconds (8 allocations: 416 bytes)

    \end{Verbatim}

            \begin{Verbatim}[commandchars=\\\{\}]
{\color{outcolor}Out[{\color{outcolor}22}]:} 3×3 Array\{Float64,2\}:
          0.168453  0.121436  0.395455
          0.531294  0.545059  0.08028 
          0.813272  0.732868  0.336742
\end{Verbatim}
        
    Do you always have to allocate memory when making new arrays? We saw
before this wasn't the case when dealing with references. Recall the
example where modifying one array modified another:

    \begin{Verbatim}[commandchars=\\\{\}]
{\color{incolor}In [{\color{incolor}26}]:} a = [1;3;5]
         @time b = a
         a[2] = 10
         a
         @time c = copy(a)
\end{Verbatim}

    \begin{Verbatim}[commandchars=\\\{\}]
  0.000001 seconds (3 allocations: 144 bytes)
  0.000004 seconds (6 allocations: 320 bytes)

    \end{Verbatim}

            \begin{Verbatim}[commandchars=\\\{\}]
{\color{outcolor}Out[{\color{outcolor}26}]:} 3-element Array\{Int64,1\}:
           1
          10
           5
\end{Verbatim}
        
    Notice that in the first case making \texttt{b} didn't allocate an
array: it just made an object with a pointer (an Int64), and had that
pointer point to the same array as \texttt{a}. To better understand this
behavior and exploit to for major performance gains, we need to make a
distinction. The array itself is the memory layout. For Julia arrays,
this is actually a C-pointer to a contiguous 1-dimensional slot of
memory. The \texttt{Array} type in Julia (and thus \texttt{Vector} and
\texttt{Matrix} which are type-alises for \texttt{Array\{T,1\}} and
\texttt{Array\{T,2\}} respectively) is a ``view'' to to that actual
array. A view is a type which points to an array, and has a
compatibility layer that changes how things like the indexing works. For
example: if we define the matrix

    \begin{Verbatim}[commandchars=\\\{\}]
{\color{incolor}In [{\color{incolor}28}]:} A = rand(4,4)
\end{Verbatim}

            \begin{Verbatim}[commandchars=\\\{\}]
{\color{outcolor}Out[{\color{outcolor}28}]:} 4×4 Array\{Float64,2\}:
          0.348922   0.716248  0.960668  0.952156
          0.795233   0.888537  0.728711  0.655722
          0.0666479  0.396674  0.462853  0.738748
          0.191717   0.941639  0.112687  0.946103
\end{Verbatim}
        
    then the array that we made is actually a 16-number long sequence (of
64-bit Floats) in memory, and \texttt{A} is a view to that array which
makes it index ``like'' it was 2-dimensional (reading down the columns).
This tells us one thing already: looping through the columns is faster
than looping through the rows. Indeed we can easily test this:

    \begin{Verbatim}[commandchars=\\\{\}]
{\color{incolor}In [{\color{incolor}47}]:} function testloops()
             b = rand(1000,1000)
             c = 0 \PYZsh{} Need this so that way the compiler doesn't optimize away the loop!
             @time for i in 1:1000, j in 1:1000
                 c+=b[i,j]
             end
             @time for j in 1:1000, i in 1:1000
                 c+=b[i,j]
             end
             bidx = eachindex(b)
             @time for i in bidx
                 c+=b[i]
             end
         end
         testloops()
\end{Verbatim}

    \begin{Verbatim}[commandchars=\\\{\}]
  0.037648 seconds (3.00 M allocations: 45.776 MB, 9.13\% gc time)
  0.028310 seconds (3.00 M allocations: 45.776 MB, 15.80\% gc time)
  0.026682 seconds (3.00 M allocations: 45.776 MB, 11.06\% gc time)

    \end{Verbatim}

    \begin{Verbatim}[commandchars=\\\{\}]
WARNING: Method definition testloops() in module Main at In[46]:2 overwritten at In[47]:2.

    \end{Verbatim}

    One should normally use the \texttt{eachindex} function since this will
return the indices in the ``fast'' order for general iterator types.

In this terminology \texttt{A{[}1:3,1:3{]}} isn't a view to the same
memory. We can check this by noticing that it doesn't mutate the
original array:

    \begin{Verbatim}[commandchars=\\\{\}]
{\color{incolor}In [{\color{incolor}49}]:} println(A)
         B = A[1:3,1:3]
         B[1,1]=100
         println(A)
\end{Verbatim}

    \begin{Verbatim}[commandchars=\\\{\}]
[0.348922 0.716248 0.960668 0.952156; 0.795233 0.888537 0.728711 0.655722; 0.0666479 0.396674 0.462853 0.738748; 0.191717 0.941639 0.112687 0.946103]
[0.348922 0.716248 0.960668 0.952156; 0.795233 0.888537 0.728711 0.655722; 0.0666479 0.396674 0.462853 0.738748; 0.191717 0.941639 0.112687 0.946103]

    \end{Verbatim}

    If we instead want a view, then we can use the \texttt{view} function:

    \begin{Verbatim}[commandchars=\\\{\}]
{\color{incolor}In [{\color{incolor}51}]:} B = view(A,1:3,1:3) \PYZsh{} No copy involved
         B[1,1] = 100 \PYZsh{} Will mutate A
         println(A)
\end{Verbatim}

    \begin{Verbatim}[commandchars=\\\{\}]
[100.0 0.716248 0.960668 0.952156; 0.795233 0.888537 0.728711 0.655722; 0.0666479 0.396674 0.462853 0.738748; 0.191717 0.941639 0.112687 0.946103]

    \end{Verbatim}

    There are many cases where you might want to use a view. For example, if
a function needs the \texttt{i}th column, you may naively think of doing
\texttt{f(A{[}i,:{]})}. But, if \texttt{A} won't be changed in the loop,
we can avoid the memory allocation (and thus make things faster) by
sending a view to the original array which is simply the column:
\texttt{f(view(A,i,:))}. Two functions can be used to give common views.
\texttt{vec} gives a view of the array as a Vector and \texttt{reshape}
builds a view in a different shape. For example:

    \begin{Verbatim}[commandchars=\\\{\}]
{\color{incolor}In [{\color{incolor}55}]:} C = vec(A)
         println(C)
         C = reshape(A,8,2) \PYZsh{} C is an 8x2 matrix
         C
\end{Verbatim}

    \begin{Verbatim}[commandchars=\\\{\}]
[100.0,0.795233,0.0666479,0.191717,0.716248,0.888537,0.396674,0.941639,0.960668,0.728711,0.462853,0.112687,0.952156,0.655722,0.738748,0.946103]

    \end{Verbatim}

            \begin{Verbatim}[commandchars=\\\{\}]
{\color{outcolor}Out[{\color{outcolor}55}]:} 8×2 Array\{Float64,2\}:
          100.0        0.960668
            0.795233   0.728711
            0.0666479  0.462853
            0.191717   0.112687
            0.716248   0.952156
            0.888537   0.655722
            0.396674   0.738748
            0.941639   0.946103
\end{Verbatim}
        
    Since these operations do not copy the array, they are very cheap and
can be used without worrying about performance issues.

    \subsubsection{Back to Linear Algebra}\label{back-to-linear-algebra}

Julia performs functions on matrices by default for dispatches on
matrices. For example, \texttt{+} is the matrix addition, while
\texttt{*} is matrix multiplication. Julia's \texttt{*} calls into a
program known as OpenBLAS so that way \texttt{*} is an optimized
multithreaded operation. For example:

    \begin{Verbatim}[commandchars=\\\{\}]
{\color{incolor}In [{\color{incolor}57}]:} A = rand(4,4); B = rand(4,4)
         C = A*B \PYZsh{} Matrix Multiplication
         D = A.*B \PYZsh{} Element-wise Multiplication
         C-D \PYZsh{} Not zero 
\end{Verbatim}

            \begin{Verbatim}[commandchars=\\\{\}]
{\color{outcolor}Out[{\color{outcolor}57}]:} 4×4 Array\{Float64,2\}:
          0.906899  1.09457   1.13941   0.673966
          0.663752  0.397312  0.515964  0.442822
          1.12845   1.27351   0.972741  0.140849
          0.956337  1.03546   0.662021  0.720551
\end{Verbatim}
        
    A common operation is to solve the linear system \texttt{Ax=b}. In Julia
this is done by \texttt{A\textbackslash{}b}:

    \begin{Verbatim}[commandchars=\\\{\}]
{\color{incolor}In [{\color{incolor}59}]:} b = 1:4
         A\PYZbs{}b
\end{Verbatim}

            \begin{Verbatim}[commandchars=\\\{\}]
{\color{outcolor}Out[{\color{outcolor}59}]:} 4-element Array\{Float64,1\}:
           10.2694
          -45.8265
           70.7536
          -63.3756
\end{Verbatim}
        
    Note that this uses a direct solver. Iterative solvers for linear
equations can be found in IterativeSolvers.jl hosted by the JuliaMath
organization.

    \subsection{A note about
``Vectorization''}\label{a-note-about-vectorization}

In MATLAB/Python/R you're told to ``vectorize'' your options, i.e.~use
\texttt{apply} or these \texttt{.*} operations, in order to speed up
computations. This is because these operations call C programs which
will be faster than any interpreted MATLAB/Python/R loop. In Julia,
that's not the case: as long as your functions are type-stable, you will
get maximal performance. Thus vectorization does not improve
performance.

In fact, vectorization can reduce performance by creating ``temporary
arrays''. Those are the intermediate array allocations that come up with
doing operations like \texttt{C{[}i,:{]} = A{[}i,:{]} .* B{[}i,:{]}}. In
general, for the best performance one should avoid vectorized calls or
be careful to use the broadcast/view syntax to define a statement
without temporaries:

    \begin{Verbatim}[commandchars=\\\{\}]
{\color{incolor}In [{\color{incolor} }]:} C[i,:] .= view(A,i,:) .* view(B,i,:)
\end{Verbatim}

    Note the odd quirk: array indexing is a view when on the left-hand side

\paragraph{Discussion: why is this the
case?}\label{discussion-why-is-this-the-case}

    \subsection{Sprase Matrices}\label{sprase-matrices}

Sprase Matrix capabilities are provided by SuiteSparse. Note that these
are saved in a table format, where there are triplets (i,j,value) which
denote the existance of a non-zero element at \texttt{(i,j)} of value
\texttt{value}. A sparse matrix can be created through the
\texttt{sparse} command:

    \begin{Verbatim}[commandchars=\\\{\}]
{\color{incolor}In [{\color{incolor}14}]:} A = sparse([1;2;3],[2;2;1],[3;4;5])
\end{Verbatim}

            \begin{Verbatim}[commandchars=\\\{\}]
{\color{outcolor}Out[{\color{outcolor}14}]:} 3×2 sparse matrix with 3 Int64 nonzero entries:
         	[3, 1]  =  5
         	[1, 2]  =  3
         	[2, 2]  =  4
\end{Verbatim}
        
    They can be converted into a dense matrix with the \texttt{full} command

    \begin{Verbatim}[commandchars=\\\{\}]
{\color{incolor}In [{\color{incolor}15}]:} full(A)
\end{Verbatim}

            \begin{Verbatim}[commandchars=\\\{\}]
{\color{outcolor}Out[{\color{outcolor}15}]:} 3×2 Array\{Int64,2\}:
          0  3
          0  4
          5  0
\end{Verbatim}
        
    The documentation shows a lot more that you can do.

    \subsection{Special Matrix Types}\label{special-matrix-types}

    Like the rest of Julia, types and multiple dispatch is used to
``secretly enhance performance''. There are many matrix types, so I will
just show a few and leave the rest to the documentation.

\subsubsection{Matrix Forms}\label{matrix-forms}

Many matrices follow specific forms: diagonal, tridiagonal, etc. Julia
has special types for these common matrix forms. For example, we can
define a \texttt{Tridiagonal} by giving it three vectors:

    \begin{Verbatim}[commandchars=\\\{\}]
{\color{incolor}In [{\color{incolor}17}]:} A = Tridiagonal(2:5,1:5,1:4)
\end{Verbatim}

            \begin{Verbatim}[commandchars=\\\{\}]
{\color{outcolor}Out[{\color{outcolor}17}]:} 5×5 Tridiagonal\{Int64\}:
          1  1  ⋅  ⋅  ⋅
          2  2  2  ⋅  ⋅
          ⋅  3  3  3  ⋅
          ⋅  ⋅  4  4  4
          ⋅  ⋅  ⋅  5  5
\end{Verbatim}
        
    We can inspect it to see its internal form:

    \begin{Verbatim}[commandchars=\\\{\}]
{\color{incolor}In [{\color{incolor}19}]:} fieldnames(A)
\end{Verbatim}

            \begin{Verbatim}[commandchars=\\\{\}]
{\color{outcolor}Out[{\color{outcolor}19}]:} 4-element Array\{Symbol,1\}:
          :dl 
          :d  
          :du 
          :du2
\end{Verbatim}
        
    \begin{Verbatim}[commandchars=\\\{\}]
{\color{incolor}In [{\color{incolor}21}]:} A.d
\end{Verbatim}

            \begin{Verbatim}[commandchars=\\\{\}]
{\color{outcolor}Out[{\color{outcolor}21}]:} 5-element Array\{Int64,1\}:
          1
          2
          3
          4
          5
\end{Verbatim}
        
    Notice that what the array stores is the vectors for the diagonals
themselves. It's clear to see that this gives a memory enhancement over
a dense matrix, and it gives a performance advantage because a dense
matrix would have an extra operation for each \texttt{0}. However, it's
also faster than a sprase matrix since a sparse matrix is stored as a
table \texttt{(i,j,value)} and retriving from the table has a bit of
overhead, while this is stored as 3 (actually 4\ldots{}) contiguous
arrays. Therefore you get a performance boost by using a special matrix
form like \texttt{Tridiagonal} whenever one is available. Note that
these special matrix forms are outfitted with dispatches so that
operations on them work seamlessly like with normal matrices. For
example, we can multiply a Tridiagonal by a dense matrix:

    \begin{Verbatim}[commandchars=\\\{\}]
{\color{incolor}In [{\color{incolor}23}]:} A*rand(5,5)
\end{Verbatim}

            \begin{Verbatim}[commandchars=\\\{\}]
{\color{outcolor}Out[{\color{outcolor}23}]:} 5×5 Array\{Float64,2\}:
          0.780741  0.74277  0.193545  0.675908  1.248  
          3.30878   3.08334  1.79876   2.54081   3.93285
          5.81165   5.22405  4.35947   3.73634   6.53809
          7.1226    4.86664  5.38682   6.70923   7.78302
          4.535     2.08879  3.20436   5.41406   6.13665
\end{Verbatim}
        
    \subsubsection{The UniformScaling
Operator}\label{the-uniformscaling-operator}

One interesting type is the \texttt{UniformScaling} operator \texttt{I}.
Essentially, I uses dispatches to cleverly act like the identity matrix
without ever forming a matrix. For example, to mathematically subtract a
scalar \texttt{λ} from a matrix \texttt{A} we use the notation
\[ A - \lambda I\]

We can do this naturally with the \texttt{I} operator:

    \begin{Verbatim}[commandchars=\\\{\}]
{\color{incolor}In [{\color{incolor}31}]:} λ = 2
         A - λ*I
\end{Verbatim}

            \begin{Verbatim}[commandchars=\\\{\}]
{\color{outcolor}Out[{\color{outcolor}31}]:} 5×5 Tridiagonal\{Int64\}:
          -1  1  ⋅  ⋅  ⋅
           2  0  2  ⋅  ⋅
           ⋅  3  1  3  ⋅
           ⋅  ⋅  4  2  4
           ⋅  ⋅  ⋅  5  3
\end{Verbatim}
        
    The MATLAB or NumPy way would be to create the identity matrix via a
command like \texttt{eye(5)}, but notice this prevents the allocation of
a 5x5 array. For large matrices, this operation is huge and thus this
could lead to some good performance improvements.

    \subsubsection{Factorization Forms}\label{factorization-forms}

One last type of interest are the factorization forms. In many cases,
you factorize a matrix using some factorization command in order to
speed up subsequence \texttt{A\textbackslash{}b} calls. Normally you
have to remember how this is done. For example, we can use a QR
factorization to solve a linear system like:

    \begin{Verbatim}[commandchars=\\\{\}]
{\color{incolor}In [{\color{incolor}42}]:} b = 1:5
         A = rand(5,5)
         Q,R = qr(A)
         println(A\PYZbs{}b)
         println(inv(R)*Q'*b)
\end{Verbatim}

    \begin{Verbatim}[commandchars=\\\{\}]
[76.7207,-41.2925,10.9053,24.3147,-27.5507]
[76.7207,-41.2925,10.9053,24.3147,-27.5507]

    \end{Verbatim}

    Thus we can save the variables \texttt{Q} and \texttt{R} and use
\texttt{inv(R)*Q'*b} instead of \texttt{A\textbackslash{}b} and get
better performance. This is the NumPy/MATLAB way. However, that requires
remembering the details of the factorization. Instead, we can have Julia
return a factorization type:

    \begin{Verbatim}[commandchars=\\\{\}]
{\color{incolor}In [{\color{incolor}44}]:} q = qrfact(A)
\end{Verbatim}

            \begin{Verbatim}[commandchars=\\\{\}]
{\color{outcolor}Out[{\color{outcolor}44}]:} Base.LinAlg.QRCompactWY\{Float64,Array\{Float64,2\}\}([-0.59303 -0.936694 -1.45618 -1.1015 -1.56616; 0.175783 -0.424785 0.173332 -0.556283 0.156938; 0.144953 0.12557 0.387682 0.385917 0.496761; 0.245034 -0.697802 -0.470008 -0.372761 -0.333139; 0.354059 -0.029529 -0.000566456 -0.459193 -0.127657],[1.61641 -0.0269726 -0.0582957 -0.295851 0.0; 6.91248e-310 1.33017 -0.988298 0.736517 0.0; 6.91248e-310 0.0 1.63813 1.27101 0.0; 6.91248e-310 0.0 6.91249e-310 1.65172 0.0; 6.91248e-310 0.0 6.91249e-310 6.91249e-310 0.0])
\end{Verbatim}
        
    What this does is it internally stores \texttt{Qt} (\texttt{Qt = Q'})
and \texttt{Rinv} (\texttt{Rinv = inv(R)}). There is a dispatch defined
for this type on ~which makes the \texttt{QRCompactWY} type perform the
fast algorithm \texttt{Rinv*Qt*b}, giving you the performance without
having to remember anything:

    \begin{Verbatim}[commandchars=\\\{\}]
{\color{incolor}In [{\color{incolor}46}]:} q\PYZbs{}b
\end{Verbatim}

            \begin{Verbatim}[commandchars=\\\{\}]
{\color{outcolor}Out[{\color{outcolor}46}]:} 5-element Array\{Float64,1\}:
           76.7207
          -41.2925
           10.9053
           24.3147
          -27.5507
\end{Verbatim}
        
    The result is fast algorithms with clean code.

    \subsection{Random Numbers}\label{random-numbers}

One last little detail is for random numbers. Uniform random numbers are
generated by the \texttt{rand} function, while normal random numbers are
generated by the \texttt{randn} function.

    \begin{Verbatim}[commandchars=\\\{\}]
{\color{incolor}In [{\color{incolor}6}]:} rand(5)
\end{Verbatim}

            \begin{Verbatim}[commandchars=\\\{\}]
{\color{outcolor}Out[{\color{outcolor}6}]:} 5-element Array\{Float64,1\}:
         0.308529
         0.427921
         0.196477
         0.297359
         0.128141
\end{Verbatim}
        
    \begin{Verbatim}[commandchars=\\\{\}]
{\color{incolor}In [{\color{incolor}8}]:} randn(5,5)
\end{Verbatim}

            \begin{Verbatim}[commandchars=\\\{\}]
{\color{outcolor}Out[{\color{outcolor}8}]:} 5×5 Array\{Float64,2\}:
         -0.239053  -1.08253   -0.648596   0.167503   2.22728 
          0.557901  -0.312473   1.29566    0.264857  -0.232809
          0.172738  -1.58695   -0.180161  -2.93838    0.787948
         -0.258095  -0.775395   1.67288   -0.641336   1.0621  
          0.805768   1.29153   -0.50146    0.276417  -0.601687
\end{Verbatim}
        
    The argument is the size of the array. You can make random numbers which
match another array with the \texttt{size} function:

    \begin{Verbatim}[commandchars=\\\{\}]
{\color{incolor}In [{\color{incolor}11}]:} a = [1 2
              3 4]
         randn(size(a))
\end{Verbatim}

            \begin{Verbatim}[commandchars=\\\{\}]
{\color{outcolor}Out[{\color{outcolor}11}]:} 2×2 Array\{Float64,2\}:
          -0.291548   0.193372
           1.53665   -0.830749
\end{Verbatim}
        

    % Add a bibliography block to the postdoc
    
    
    
    \end{document}
