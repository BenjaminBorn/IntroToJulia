
% Default to the notebook output style

    


% Inherit from the specified cell style.




    
\documentclass[11pt]{article}

    
    
    \usepackage[T1]{fontenc}
    % Nicer default font than Computer Modern for most use cases
    \usepackage{palatino}

    % Basic figure setup, for now with no caption control since it's done
    % automatically by Pandoc (which extracts ![](path) syntax from Markdown).
    \usepackage{graphicx}
    % We will generate all images so they have a width \maxwidth. This means
    % that they will get their normal width if they fit onto the page, but
    % are scaled down if they would overflow the margins.
    \makeatletter
    \def\maxwidth{\ifdim\Gin@nat@width>\linewidth\linewidth
    \else\Gin@nat@width\fi}
    \makeatother
    \let\Oldincludegraphics\includegraphics
    % Set max figure width to be 80% of text width, for now hardcoded.
    \renewcommand{\includegraphics}[1]{\Oldincludegraphics[width=.8\maxwidth]{#1}}
    % Ensure that by default, figures have no caption (until we provide a
    % proper Figure object with a Caption API and a way to capture that
    % in the conversion process - todo).
    \usepackage{caption}
    \DeclareCaptionLabelFormat{nolabel}{}
    \captionsetup{labelformat=nolabel}

    \usepackage{adjustbox} % Used to constrain images to a maximum size 
    \usepackage{xcolor} % Allow colors to be defined
    \usepackage{enumerate} % Needed for markdown enumerations to work
    \usepackage{geometry} % Used to adjust the document margins
    \usepackage{amsmath} % Equations
    \usepackage{amssymb} % Equations
    \usepackage{textcomp} % defines textquotesingle
    % Hack from http://tex.stackexchange.com/a/47451/13684:
    \AtBeginDocument{%
        \def\PYZsq{\textquotesingle}% Upright quotes in Pygmentized code
    }
    \usepackage{upquote} % Upright quotes for verbatim code
    \usepackage{eurosym} % defines \euro
    \usepackage[mathletters]{ucs} % Extended unicode (utf-8) support
    \usepackage[utf8x]{inputenc} % Allow utf-8 characters in the tex document
    \usepackage{fancyvrb} % verbatim replacement that allows latex
    \usepackage{grffile} % extends the file name processing of package graphics 
                         % to support a larger range 
    % The hyperref package gives us a pdf with properly built
    % internal navigation ('pdf bookmarks' for the table of contents,
    % internal cross-reference links, web links for URLs, etc.)
    \usepackage{hyperref}
    \usepackage{longtable} % longtable support required by pandoc >1.10
    \usepackage{booktabs}  % table support for pandoc > 1.12.2
    \usepackage[normalem]{ulem} % ulem is needed to support strikethroughs (\sout)
                                % normalem makes italics be italics, not underlines
    

    
    
    % Colors for the hyperref package
    \definecolor{urlcolor}{rgb}{0,.145,.698}
    \definecolor{linkcolor}{rgb}{.71,0.21,0.01}
    \definecolor{citecolor}{rgb}{.12,.54,.11}

    % ANSI colors
    \definecolor{ansi-black}{HTML}{3E424D}
    \definecolor{ansi-black-intense}{HTML}{282C36}
    \definecolor{ansi-red}{HTML}{E75C58}
    \definecolor{ansi-red-intense}{HTML}{B22B31}
    \definecolor{ansi-green}{HTML}{00A250}
    \definecolor{ansi-green-intense}{HTML}{007427}
    \definecolor{ansi-yellow}{HTML}{DDB62B}
    \definecolor{ansi-yellow-intense}{HTML}{B27D12}
    \definecolor{ansi-blue}{HTML}{208FFB}
    \definecolor{ansi-blue-intense}{HTML}{0065CA}
    \definecolor{ansi-magenta}{HTML}{D160C4}
    \definecolor{ansi-magenta-intense}{HTML}{A03196}
    \definecolor{ansi-cyan}{HTML}{60C6C8}
    \definecolor{ansi-cyan-intense}{HTML}{258F8F}
    \definecolor{ansi-white}{HTML}{C5C1B4}
    \definecolor{ansi-white-intense}{HTML}{A1A6B2}

    % commands and environments needed by pandoc snippets
    % extracted from the output of `pandoc -s`
    \providecommand{\tightlist}{%
      \setlength{\itemsep}{0pt}\setlength{\parskip}{0pt}}
    \DefineVerbatimEnvironment{Highlighting}{Verbatim}{commandchars=\\\{\}}
    % Add ',fontsize=\small' for more characters per line
    \newenvironment{Shaded}{}{}
    \newcommand{\KeywordTok}[1]{\textcolor[rgb]{0.00,0.44,0.13}{\textbf{{#1}}}}
    \newcommand{\DataTypeTok}[1]{\textcolor[rgb]{0.56,0.13,0.00}{{#1}}}
    \newcommand{\DecValTok}[1]{\textcolor[rgb]{0.25,0.63,0.44}{{#1}}}
    \newcommand{\BaseNTok}[1]{\textcolor[rgb]{0.25,0.63,0.44}{{#1}}}
    \newcommand{\FloatTok}[1]{\textcolor[rgb]{0.25,0.63,0.44}{{#1}}}
    \newcommand{\CharTok}[1]{\textcolor[rgb]{0.25,0.44,0.63}{{#1}}}
    \newcommand{\StringTok}[1]{\textcolor[rgb]{0.25,0.44,0.63}{{#1}}}
    \newcommand{\CommentTok}[1]{\textcolor[rgb]{0.38,0.63,0.69}{\textit{{#1}}}}
    \newcommand{\OtherTok}[1]{\textcolor[rgb]{0.00,0.44,0.13}{{#1}}}
    \newcommand{\AlertTok}[1]{\textcolor[rgb]{1.00,0.00,0.00}{\textbf{{#1}}}}
    \newcommand{\FunctionTok}[1]{\textcolor[rgb]{0.02,0.16,0.49}{{#1}}}
    \newcommand{\RegionMarkerTok}[1]{{#1}}
    \newcommand{\ErrorTok}[1]{\textcolor[rgb]{1.00,0.00,0.00}{\textbf{{#1}}}}
    \newcommand{\NormalTok}[1]{{#1}}
    
    % Additional commands for more recent versions of Pandoc
    \newcommand{\ConstantTok}[1]{\textcolor[rgb]{0.53,0.00,0.00}{{#1}}}
    \newcommand{\SpecialCharTok}[1]{\textcolor[rgb]{0.25,0.44,0.63}{{#1}}}
    \newcommand{\VerbatimStringTok}[1]{\textcolor[rgb]{0.25,0.44,0.63}{{#1}}}
    \newcommand{\SpecialStringTok}[1]{\textcolor[rgb]{0.73,0.40,0.53}{{#1}}}
    \newcommand{\ImportTok}[1]{{#1}}
    \newcommand{\DocumentationTok}[1]{\textcolor[rgb]{0.73,0.13,0.13}{\textit{{#1}}}}
    \newcommand{\AnnotationTok}[1]{\textcolor[rgb]{0.38,0.63,0.69}{\textbf{\textit{{#1}}}}}
    \newcommand{\CommentVarTok}[1]{\textcolor[rgb]{0.38,0.63,0.69}{\textbf{\textit{{#1}}}}}
    \newcommand{\VariableTok}[1]{\textcolor[rgb]{0.10,0.09,0.49}{{#1}}}
    \newcommand{\ControlFlowTok}[1]{\textcolor[rgb]{0.00,0.44,0.13}{\textbf{{#1}}}}
    \newcommand{\OperatorTok}[1]{\textcolor[rgb]{0.40,0.40,0.40}{{#1}}}
    \newcommand{\BuiltInTok}[1]{{#1}}
    \newcommand{\ExtensionTok}[1]{{#1}}
    \newcommand{\PreprocessorTok}[1]{\textcolor[rgb]{0.74,0.48,0.00}{{#1}}}
    \newcommand{\AttributeTok}[1]{\textcolor[rgb]{0.49,0.56,0.16}{{#1}}}
    \newcommand{\InformationTok}[1]{\textcolor[rgb]{0.38,0.63,0.69}{\textbf{\textit{{#1}}}}}
    \newcommand{\WarningTok}[1]{\textcolor[rgb]{0.38,0.63,0.69}{\textbf{\textit{{#1}}}}}
    
    
    % Define a nice break command that doesn't care if a line doesn't already
    % exist.
    \def\br{\hspace*{\fill} \\* }
    % Math Jax compatability definitions
    \def\gt{>}
    \def\lt{<}
    % Document parameters
    \title{WhyJulia}
    
    
    

    % Pygments definitions
    
\makeatletter
\def\PY@reset{\let\PY@it=\relax \let\PY@bf=\relax%
    \let\PY@ul=\relax \let\PY@tc=\relax%
    \let\PY@bc=\relax \let\PY@ff=\relax}
\def\PY@tok#1{\csname PY@tok@#1\endcsname}
\def\PY@toks#1+{\ifx\relax#1\empty\else%
    \PY@tok{#1}\expandafter\PY@toks\fi}
\def\PY@do#1{\PY@bc{\PY@tc{\PY@ul{%
    \PY@it{\PY@bf{\PY@ff{#1}}}}}}}
\def\PY#1#2{\PY@reset\PY@toks#1+\relax+\PY@do{#2}}

\def\PY@tok@gd{\def\PY@tc##1{\textcolor[rgb]{0.63,0.00,0.00}{##1}}}
\def\PY@tok@gu{\let\PY@bf=\textbf\def\PY@tc##1{\textcolor[rgb]{0.50,0.00,0.50}{##1}}}
\def\PY@tok@gt{\def\PY@tc##1{\textcolor[rgb]{0.00,0.25,0.82}{##1}}}
\def\PY@tok@gs{\let\PY@bf=\textbf}
\def\PY@tok@gr{\def\PY@tc##1{\textcolor[rgb]{1.00,0.00,0.00}{##1}}}
\def\PY@tok@cm{\let\PY@it=\textit\def\PY@tc##1{\textcolor[rgb]{0.25,0.50,0.50}{##1}}}
\def\PY@tok@vg{\def\PY@tc##1{\textcolor[rgb]{0.10,0.09,0.49}{##1}}}
\def\PY@tok@m{\def\PY@tc##1{\textcolor[rgb]{0.40,0.40,0.40}{##1}}}
\def\PY@tok@mh{\def\PY@tc##1{\textcolor[rgb]{0.40,0.40,0.40}{##1}}}
\def\PY@tok@go{\def\PY@tc##1{\textcolor[rgb]{0.50,0.50,0.50}{##1}}}
\def\PY@tok@ge{\let\PY@it=\textit}
\def\PY@tok@vc{\def\PY@tc##1{\textcolor[rgb]{0.10,0.09,0.49}{##1}}}
\def\PY@tok@il{\def\PY@tc##1{\textcolor[rgb]{0.40,0.40,0.40}{##1}}}
\def\PY@tok@cs{\let\PY@it=\textit\def\PY@tc##1{\textcolor[rgb]{0.25,0.50,0.50}{##1}}}
\def\PY@tok@cp{\def\PY@tc##1{\textcolor[rgb]{0.74,0.48,0.00}{##1}}}
\def\PY@tok@gi{\def\PY@tc##1{\textcolor[rgb]{0.00,0.63,0.00}{##1}}}
\def\PY@tok@gh{\let\PY@bf=\textbf\def\PY@tc##1{\textcolor[rgb]{0.00,0.00,0.50}{##1}}}
\def\PY@tok@ni{\let\PY@bf=\textbf\def\PY@tc##1{\textcolor[rgb]{0.60,0.60,0.60}{##1}}}
\def\PY@tok@nl{\def\PY@tc##1{\textcolor[rgb]{0.63,0.63,0.00}{##1}}}
\def\PY@tok@nn{\let\PY@bf=\textbf\def\PY@tc##1{\textcolor[rgb]{0.00,0.00,1.00}{##1}}}
\def\PY@tok@no{\def\PY@tc##1{\textcolor[rgb]{0.53,0.00,0.00}{##1}}}
\def\PY@tok@na{\def\PY@tc##1{\textcolor[rgb]{0.49,0.56,0.16}{##1}}}
\def\PY@tok@nb{\def\PY@tc##1{\textcolor[rgb]{0.00,0.50,0.00}{##1}}}
\def\PY@tok@nc{\let\PY@bf=\textbf\def\PY@tc##1{\textcolor[rgb]{0.00,0.00,1.00}{##1}}}
\def\PY@tok@nd{\def\PY@tc##1{\textcolor[rgb]{0.67,0.13,1.00}{##1}}}
\def\PY@tok@ne{\let\PY@bf=\textbf\def\PY@tc##1{\textcolor[rgb]{0.82,0.25,0.23}{##1}}}
\def\PY@tok@nf{\def\PY@tc##1{\textcolor[rgb]{0.00,0.00,1.00}{##1}}}
\def\PY@tok@si{\let\PY@bf=\textbf\def\PY@tc##1{\textcolor[rgb]{0.73,0.40,0.53}{##1}}}
\def\PY@tok@s2{\def\PY@tc##1{\textcolor[rgb]{0.73,0.13,0.13}{##1}}}
\def\PY@tok@vi{\def\PY@tc##1{\textcolor[rgb]{0.10,0.09,0.49}{##1}}}
\def\PY@tok@nt{\let\PY@bf=\textbf\def\PY@tc##1{\textcolor[rgb]{0.00,0.50,0.00}{##1}}}
\def\PY@tok@nv{\def\PY@tc##1{\textcolor[rgb]{0.10,0.09,0.49}{##1}}}
\def\PY@tok@s1{\def\PY@tc##1{\textcolor[rgb]{0.73,0.13,0.13}{##1}}}
\def\PY@tok@sh{\def\PY@tc##1{\textcolor[rgb]{0.73,0.13,0.13}{##1}}}
\def\PY@tok@sc{\def\PY@tc##1{\textcolor[rgb]{0.73,0.13,0.13}{##1}}}
\def\PY@tok@sx{\def\PY@tc##1{\textcolor[rgb]{0.00,0.50,0.00}{##1}}}
\def\PY@tok@bp{\def\PY@tc##1{\textcolor[rgb]{0.00,0.50,0.00}{##1}}}
\def\PY@tok@c1{\let\PY@it=\textit\def\PY@tc##1{\textcolor[rgb]{0.25,0.50,0.50}{##1}}}
\def\PY@tok@kc{\let\PY@bf=\textbf\def\PY@tc##1{\textcolor[rgb]{0.00,0.50,0.00}{##1}}}
\def\PY@tok@c{\let\PY@it=\textit\def\PY@tc##1{\textcolor[rgb]{0.25,0.50,0.50}{##1}}}
\def\PY@tok@mf{\def\PY@tc##1{\textcolor[rgb]{0.40,0.40,0.40}{##1}}}
\def\PY@tok@err{\def\PY@bc##1{\fcolorbox[rgb]{1.00,0.00,0.00}{1,1,1}{##1}}}
\def\PY@tok@kd{\let\PY@bf=\textbf\def\PY@tc##1{\textcolor[rgb]{0.00,0.50,0.00}{##1}}}
\def\PY@tok@ss{\def\PY@tc##1{\textcolor[rgb]{0.10,0.09,0.49}{##1}}}
\def\PY@tok@sr{\def\PY@tc##1{\textcolor[rgb]{0.73,0.40,0.53}{##1}}}
\def\PY@tok@mo{\def\PY@tc##1{\textcolor[rgb]{0.40,0.40,0.40}{##1}}}
\def\PY@tok@kn{\let\PY@bf=\textbf\def\PY@tc##1{\textcolor[rgb]{0.00,0.50,0.00}{##1}}}
\def\PY@tok@mi{\def\PY@tc##1{\textcolor[rgb]{0.40,0.40,0.40}{##1}}}
\def\PY@tok@gp{\let\PY@bf=\textbf\def\PY@tc##1{\textcolor[rgb]{0.00,0.00,0.50}{##1}}}
\def\PY@tok@o{\def\PY@tc##1{\textcolor[rgb]{0.40,0.40,0.40}{##1}}}
\def\PY@tok@kr{\let\PY@bf=\textbf\def\PY@tc##1{\textcolor[rgb]{0.00,0.50,0.00}{##1}}}
\def\PY@tok@s{\def\PY@tc##1{\textcolor[rgb]{0.73,0.13,0.13}{##1}}}
\def\PY@tok@kp{\def\PY@tc##1{\textcolor[rgb]{0.00,0.50,0.00}{##1}}}
\def\PY@tok@w{\def\PY@tc##1{\textcolor[rgb]{0.73,0.73,0.73}{##1}}}
\def\PY@tok@kt{\def\PY@tc##1{\textcolor[rgb]{0.69,0.00,0.25}{##1}}}
\def\PY@tok@ow{\let\PY@bf=\textbf\def\PY@tc##1{\textcolor[rgb]{0.67,0.13,1.00}{##1}}}
\def\PY@tok@sb{\def\PY@tc##1{\textcolor[rgb]{0.73,0.13,0.13}{##1}}}
\def\PY@tok@k{\let\PY@bf=\textbf\def\PY@tc##1{\textcolor[rgb]{0.00,0.50,0.00}{##1}}}
\def\PY@tok@se{\let\PY@bf=\textbf\def\PY@tc##1{\textcolor[rgb]{0.73,0.40,0.13}{##1}}}
\def\PY@tok@sd{\let\PY@it=\textit\def\PY@tc##1{\textcolor[rgb]{0.73,0.13,0.13}{##1}}}

\def\PYZbs{\char`\\}
\def\PYZus{\char`\_}
\def\PYZob{\char`\{}
\def\PYZcb{\char`\}}
\def\PYZca{\char`\^}
\def\PYZsh{\char`\#}
\def\PYZpc{\char`\%}
\def\PYZdl{\char`\$}
\def\PYZti{\char`\~}
% for compatibility with earlier versions
\def\PYZat{@}
\def\PYZlb{[}
\def\PYZrb{]}
\makeatother


    % Exact colors from NB
    \definecolor{incolor}{rgb}{0.0, 0.0, 0.5}
    \definecolor{outcolor}{rgb}{0.545, 0.0, 0.0}



    
    % Prevent overflowing lines due to hard-to-break entities
    \sloppy 
    % Setup hyperref package
    \hypersetup{
      breaklinks=true,  % so long urls are correctly broken across lines
      colorlinks=true,
      urlcolor=urlcolor,
      linkcolor=linkcolor,
      citecolor=citecolor,
      }
    % Slightly bigger margins than the latex defaults
    
    \geometry{verbose,tmargin=1in,bmargin=1in,lmargin=1in,rmargin=1in}
    
    

    \begin{document}
    
    
    \maketitle
    
    

    
    \section{Why Does Julia Work So
Well?}\label{why-does-julia-work-so-well}

There is an obvious reason to choose Julia:

\begin{quote}
it's faster than other scripting languages, allowing you to have the
rapid development of Python/MATLAB/R while producing code that is as
fast as C/Fortran
\end{quote}

Newcommers to Julia might be a little weary of that statement.

\begin{enumerate}
\def\labelenumi{\arabic{enumi}.}
\itemsep1pt\parskip0pt\parsep0pt
\item
  Why not just make other scripting languages faster? If Julia can do
  it, why can't others?
\item
  How do you interpert Julia benchmarks to confirm this? (This is
  surprisingly difficult for many!)
\item
  That sounds like it violates the No-Free-Lunch heuristic. Is there
  really nothing lost?
\end{enumerate}

Many people believe Julia is fast \textbf{because it is Just-In-Time
(JIT) compiled} (i.e.~every statement is run using compiled functions
which are either compiled right before they are used, or cached
compilations from before). This leads to questions about what Julia
gives over JIT'd implementations of Python/R (and MATLAB by default uses
a JIT). These JIT compilers have been optimized for far longer than
Julia, so why should we be crazy and believe that somehow Julia quickly
out-optimized all of them? However, that is a complete misunderstanding
of Julia. What I want show, in a very visual way, is that Julia is fast
because of its design decisions. The core design decision,
\textbf{type-stability through specialization via multiple-dispatch} is
what allows Julia to be very easy for a compiler to make into efficient
code, but also allow the code to be very concise and ``look like a
scripting language''. This will lead to some very clear performance
gains.

But what we will see in this example is that Julia does not always act
like other scripting languages. There are some ``lunches lost'' that we
will have to understand. Understanding how this design decision effects
the way you must code is crucial to producing efficient Julia code.

To see the difference, we only need to go as far as basic math.

    \subsection{Arithmetic in Julia}\label{arithmetic-in-julia}

In general, math in Julia looks the same as in other scripting
languages. One detail to note is that the numbers are ``true numbers'',
as in a \texttt{Float64} is truly the same thing as a 64-bit floating
point number or a ``double'' in C. A \texttt{Vector\{Float64\}} is the
same memory layout as an array of doubles in C, both making interop with
C easy (indeed, in some sense ``Julia is a layer on top of C'') and it
leads to high performance (the same is true for NumPy arrays).

Some math in Julia:

    \begin{Verbatim}[commandchars=\\\{\}]
{\color{incolor}In [{\color{incolor}3}]:} a = 2+2
        b = a/3
        c = a÷3 \PYZsh{}\PYZbs{}div tab completion, means integer division
        d = 4*5
        println([a;b;c;d])
\end{Verbatim}

    \begin{Verbatim}[commandchars=\\\{\}]
[4.0,1.33333,1.0,20.0]

    \end{Verbatim}

    Note here that I showed off Julia's unicode tab completion. Julia allows
for unicode characters, and these can be used by tab completing
Latex-like statements. Also, multiplication by a number is allowed
without the * if followed by a variable. For example, the following is
allowed Julia code:

    \begin{Verbatim}[commandchars=\\\{\}]
{\color{incolor}In [{\color{incolor}4}]:} α = 0.5
        ∇f(u) = α*u; ∇f(2)
        sin(2π)
\end{Verbatim}

    \begin{Verbatim}[commandchars=\\\{\}]
WARNING: Method definition ∇f(Any) in module Main at In[2]:2 overwritten at In[4]:2.

    \end{Verbatim}

            \begin{Verbatim}[commandchars=\\\{\}]
{\color{outcolor}Out[{\color{outcolor}4}]:} -2.4492935982947064e-16
\end{Verbatim}
        
    \subsection{Type-stability and Code
Introspection}\label{type-stability-and-code-introspection}

Type stability is the idea that there is only 1 possible type which can
be outputtted from a method. For example, the reasonable type to output
from \texttt{*(::Float64,::Float64)} is a \texttt{Float64}. No matter
what you give it, it will spit out a \texttt{Float64}. This right here
is multiple-dispatch: the \texttt{*} operator calls a different method
depending on the types that it sees. When it sees floats, it will spit
out floats. Julia provides code introspection macros so that way you can
see what your code actually compiles to. Thus Julia is not just a
scripting language, it's a scripting language which lets you deal with
assembly! If we check the LLVM-compiled code (LLVM is the Julia
compiler)

    \begin{Verbatim}[commandchars=\\\{\}]
{\color{incolor}In [{\color{incolor}5}]:} @code\PYZus{}llvm 2*5
\end{Verbatim}

    \begin{Verbatim}[commandchars=\\\{\}]

define i64 @"jlsys\_*\_49090"(i64, i64) \#0 \{
top:
  \%2 = mul i64 \%1, \%0
  ret i64 \%2
\}

    \end{Verbatim}

    This output is saying that a floating point multiplication operation is
performed and the answer is returned. We can even look at the assembly

    \begin{Verbatim}[commandchars=\\\{\}]
{\color{incolor}In [{\color{incolor}6}]:} @code\PYZus{}native 2*5
\end{Verbatim}

    \begin{Verbatim}[commandchars=\\\{\}]
	.text
Filename: int.jl
	pushq	\%rbp
	movq	\%rsp, \%rbp
Source line: 33
	imulq	\%rsi, \%rdi
	movq	\%rdi, \%rax
	popq	\%rbp
	retq
	nopl	(\%rax)

    \end{Verbatim}

    This shows us that the \texttt{*} function has compiled down to exactly
the same operation as what happens in C/Fortran, meaning it achieves the
same performance (even though it's defined in Julia). Thus it is
possible to not just get ``close'' to C, but actually get the same C
code out. In what cases does this happen?

The interesting thing about Julia is that, asking which cases this
happens is not the right question. the right question is, in what cases
does the code not compile to something as efficient as C/Fortran? The
key here is type-stability. If a function is type-stable, then the
compiler can know what the type will be at all points in the function
and smartly optimize it to the same assembly as C/Fortran. If it is not
type-stable, Julia has to add expensive ``boxing'' to ensure types are
found/known before operations.

\paragraph{This is the key difference between Julia and other scripting
languages}\label{this-is-the-key-difference-between-julia-and-other-scripting-languages}

The upside is that Julia's functions, when type stable, are essentially
C/Fortran functions. Thus \texttt{\^{}} (exponentiation) is fast.
However, \texttt{\^{}(::Int64,::Int64)} is type-stable, so what type
should it output?

    \begin{Verbatim}[commandchars=\\\{\}]
{\color{incolor}In [{\color{incolor}7}]:} 2\PYZca{}5
\end{Verbatim}

            \begin{Verbatim}[commandchars=\\\{\}]
{\color{outcolor}Out[{\color{outcolor}7}]:} 32
\end{Verbatim}
        
    \begin{Verbatim}[commandchars=\\\{\}]
{\color{incolor}In [{\color{incolor}8}]:} 2\PYZca{}-5
\end{Verbatim}

    \begin{Verbatim}[commandchars=\\\{\}]

        LoadError: DomainError:
    Cannot raise an integer x to a negative power -n. 
    Make x a float by adding a zero decimal (e.g. 2.0\^{}-n instead of 2\^{}-n), or write 1/x\^{}n, float(x)\^{}-n, or (x//1)\^{}-n.
    while loading In[8], in expression starting on line 1

        

         in power\_by\_squaring(::Int64, ::Int64) at ./intfuncs.jl:118

         in \^{}(::Int64, ::Int64) at ./intfuncs.jl:142

    \end{Verbatim}

    Here we get an error. In order to guarantee to the compiler that
\texttt{\^{}} will give an Int64 back, it has to throw an error. If you
do this in MATLAB, Python, or R, it will not throw an error. That is
because those languages do not have their entire language built around
type stability.

What happens when we don't have type stability? Let's inspect this code:

    \begin{Verbatim}[commandchars=\\\{\}]
{\color{incolor}In [{\color{incolor}9}]:} @code\PYZus{}native 2\PYZca{}5
\end{Verbatim}

    \begin{Verbatim}[commandchars=\\\{\}]
	.text
Filename: intfuncs.jl
	pushq	\%rbp
	movq	\%rsp, \%rbp
Source line: 142
	callq	power\_by\_squaring
	popq	\%rbp
	retq
	nopl	(\%rax,\%rax)

    \end{Verbatim}

    Now let's define our own exponentiation on integers. Let's make it
``safe'' like the form seen in other scripting languages:

    \begin{Verbatim}[commandchars=\\\{\}]
{\color{incolor}In [{\color{incolor}7}]:} function expo(x,y)
            if y>0
                return x\PYZca{}y
            else
                x = convert(Float64,x)
                return x\PYZca{}y
            end
        end
\end{Verbatim}

            \begin{Verbatim}[commandchars=\\\{\}]
{\color{outcolor}Out[{\color{outcolor}7}]:} expo (generic function with 1 method)
\end{Verbatim}
        
    Let's make sure it works:

    \begin{Verbatim}[commandchars=\\\{\}]
{\color{incolor}In [{\color{incolor}12}]:} println(expo(2,5))
         expo(2,-5)
\end{Verbatim}

    \begin{Verbatim}[commandchars=\\\{\}]
32

    \end{Verbatim}

            \begin{Verbatim}[commandchars=\\\{\}]
{\color{outcolor}Out[{\color{outcolor}12}]:} 0.03125
\end{Verbatim}
        
    What happens if we inspect this code?

    \begin{Verbatim}[commandchars=\\\{\}]
{\color{incolor}In [{\color{incolor}14}]:} @code\PYZus{}native expo(2,5)
\end{Verbatim}

    \begin{Verbatim}[commandchars=\\\{\}]
	.text
Filename: In[10]
	pushq	\%rbp
	movq	\%rsp, \%rbp
	pushq	\%r15
	pushq	\%r14
	pushq	\%r12
	pushq	\%rbx
	subq	\$64, \%rsp
	movq	\%rsi, \%rbx
	movq	\%fs:0, \%r14
	addq	\$-2672, \%r14            \# imm = 0xFFFFFFFFFFFFF590
	xorps	\%xmm0, \%xmm0
	movups	\%xmm0, -48(\%rbp)
	movups	\%xmm0, -64(\%rbp)
	movq	\$8, -80(\%rbp)
	movq	(\%r14), \%rax
	movq	\%rax, -72(\%rbp)
	leaq	-80(\%rbp), \%rax
	movq	\%rax, (\%r14)
Source line: 2
	movabsq	\$jl\_box\_int64, \%r15
	callq	*\%r15
	movq	\%rax, -64(\%rbp)
	testq	\%rbx, \%rbx
	jle	L137
Source line: 3
	movq	\%rax, -56(\%rbp)
	movq	(\%rax), \%rdi
	movabsq	\$power\_by\_squaring, \%rax
	movq	\%rbx, \%rsi
	callq	*\%rax
	movq	\%rax, \%rdi
	callq	*\%r15
	movq	-72(\%rbp), \%rcx
	movq	\%rcx, (\%r14)
	addq	\$64, \%rsp
	popq	\%rbx
	popq	\%r12
	popq	\%r14
	popq	\%r15
	popq	\%rbp
	retq
L137:
	movabsq	\$140082976556080, \%r15  \# imm = 0x7F679C0E8430
Source line: 5
	movq	\%rax, -48(\%rbp)
	movq	(\%rax), \%rax
Source line: 6
	cvtsi2sdq	\%rax, \%xmm0
Source line: 5
	movsd	\%xmm0, -88(\%rbp)
	movabsq	\$jl\_gc\_pool\_alloc, \%r12
	movl	\$1432, \%esi             \# imm = 0x598
	movl	\$16, \%edx
	movq	\%r14, \%rdi
	callq	*\%r12
	movsd	-88(\%rbp), \%xmm0        \# xmm0 = mem[0],zero
	movq	\%r15, -8(\%rax)
	movsd	\%xmm0, (\%rax)
	movq	\%rax, -64(\%rbp)
Source line: 6
	movq	\%rax, -40(\%rbp)
	movslq	\%ebx, \%rax
	cmpq	\%rbx, \%rax
	jne	L287
	movabsq	\$\_\_powidf2, \%rax
	movl	\%ebx, \%edi
	callq	*\%rax
	movsd	\%xmm0, -88(\%rbp)
	movl	\$1432, \%esi             \# imm = 0x598
	movl	\$16, \%edx
	movq	\%r14, \%rdi
	callq	*\%r12
	movq	\%r15, -8(\%rax)
	movsd	-88(\%rbp), \%xmm0        \# xmm0 = mem[0],zero
	movsd	\%xmm0, (\%rax)
	movq	-72(\%rbp), \%rcx
	movq	\%rcx, (\%r14)
	addq	\$64, \%rsp
	popq	\%rbx
	popq	\%r12
	popq	\%r14
	popq	\%r15
	popq	\%rbp
	retq
L287:
	addq	\$-479504, \%r15          \# imm = 0xFFFFFFFFFFF8AEF0
	movabsq	\$jl\_throw, \%rax
	movq	\%r15, \%rdi
	callq	*\%rax
	nopw	\%cs:(\%rax,\%rax)

    \end{Verbatim}

    That's a very visual demonstration on why Julia achieves such higher
performance than other scripting languages.

    \section{Core Idea: Multiple Dispatch + Type Stability =\textgreater{}
Speed +
Readibility}\label{core-idea-multiple-dispatch-type-stability-speed-readibility}

Type stability is one crucial feature which separates Julia apart from
other scripting languages. In fact, the core idea of Julia is the
following statement:

\paragraph{Multiple dispatch allows for a language to dispatch function
calls onto type-stable
functions.}\label{multiple-dispatch-allows-for-a-language-to-dispatch-function-calls-onto-type-stable-functions.}

This is what Julia is all about, so let's take some time to dig into
it.If you have type stability inside of a function (meaning, any
function call within the function is also type-stable), then the
compiler can know the types of the variables at every step. Therefore it
can compile the function with the full amount of optimizations since at
this point the code is essentially the same as C/Fortran code.
Multiple-dispatch works into this story because it means that \texttt{*}
can be a type-stable function: it just means different things for
different inputs. But if the compiler can know the types of \texttt{a}
and \texttt{b} before calling \texttt{*}, then it knows which \texttt{*}
method to use, and therefore it knows the output type of \texttt{c=a*b}.
Thus it can propogate the type information all the way down, knowing all
of the types along the way, allowing for full optimiziations. Multiple
dispatch allows \texttt{*} to mean the ``right thing'' every time you
use it, almost magically allowing this optimization.

There are a few things we learn from this. For one, in order to achieve
this level of optimization, you must have type-stability. This is not
featured in the standard libraries of most languages, and was choice
that was made to make the experience a little easier for users.
Secondly, multiple dispatch was required to be able to specialize the
functions for types which allows for the scripting language syntax to be
``more explicit than meets the eye''. Lastly, a robust type system is
required. In order to build the type-unstable exponentiation (which may
be needed) we needed functionalities like convert. Thus the language
must be designed to be type-stable with multiple dispatch and centered
around a robust type system in order to achieve this raw performance
while maintaining the syntax/ease-of-use of a scripting language. You
can put a JIT on Python, but to really make it Julia, you would have to
design it to be Julia.

    \subsection{The Julia Benchmarks}\label{the-julia-benchmarks}

The Julia benchmarks, featured on \href{http://julialang.org/}{the Julia
website}, test components of the programming language for speed.
\textbf{This doesn't mean it's testing the fastest implemention}. That
is where a major misconception occurs. You'll have an R programmer look
at the R code for the Fibonacci calculator and say ``wow, that's
terrible R code. You're not supposed to use recursion in R. Of course
it's slow''. However, the Fibonacci problem is designed to test
recursion, not the fastest implementation to the the ith Fibonacci
number. The other problems are the same way: testing basic components of
the langauge to see how fast they are.

Julia is built up using multiple-dispatch on type-stable functions. As a
result, even the earliest versions of Julia were easy for compilers to
optimize to C/Fortran efficiency. It's clear that in almost every case
Julia is close to C. Where it is not close to C actually has a few
details. The first is the Fibonacci problem where Julia is 2.11x from C.
This is because it is a test of recursion, and Julia does not fully
optimize recursion (but still does very well on this problem!). The
optimization which is used to receive the fastest times for this type of
problem is known as Tail-Call Optimization. Julia can at any time add
this optimization, though
\href{https://github.com/JuliaLang/julia/issues/4964}{there are reasons}
why
\href{https://groups.google.com/forum/?fromgroups=\#!searchin/julia-users/tail\$20call/julia-users/qHRDj80rIvA/T3AylpjsASEJ}{they
choose not to}. The main reason is: any case where tail-call
optimization is possible, a loop can also be used. But a loop is also
more robust for optimizations (there are many recursive calls which will
fail to tail-call optimize) and thus they want to just recommend using
loops instead of using fragile TCO.

The other cases where Julia doesn't do as well are the
\texttt{rand\_mat\_stat} and the \texttt{parse\_int} tests. However,
this is largely due to a feature known as bounds checking. In most
scripting languages, you will receive an error if you try to index an
array outside of its bounds. Julia will do this by default:

    \begin{Verbatim}[commandchars=\\\{\}]
{\color{incolor}In [{\color{incolor}28}]:} function test1()
             a = zeros(3)
             for i=1:4
                 a[i] = i
             end
         end
         test1()
\end{Verbatim}

    \begin{Verbatim}[commandchars=\\\{\}]

        LoadError: BoundsError: attempt to access 3-element Array\{Float64,1\} at index [4]
    while loading In[28], in expression starting on line 7

        

         in test1() at ./In[28]:4

    \end{Verbatim}

    However, Julia allows you to turn this off using the \texttt{@inbounds}
macro:

    \begin{Verbatim}[commandchars=\\\{\}]
{\color{incolor}In [{\color{incolor}30}]:} function test2()
             a = zeros(3)
             @inbounds for i=1:4
                 a[i] = i
             end
         end
         test2()
\end{Verbatim}

    \begin{Verbatim}[commandchars=\\\{\}]
WARNING: Method definition test2() in module Main at In[29]:2 overwritten at In[30]:2.

    \end{Verbatim}

    This gives you the same unsafe behavior as C/Fortran, but also the same
speed (indeed, if you add these to the benchmarks they will speed up
close to C). This is another interesting feature of Julia: it lets you
\textbf{by default have the safety of a scripting language, but turn off
these features when necessary (/after testing and debugging) to get full
performance.}

    \section{Small Expansion of the Idea: Strict
Typing}\label{small-expansion-of-the-idea-strict-typing}

    Type-stability is not the only necessity. You also need strict typing.
In Python you can put anything into an array. In Julia, you can only put
types of \texttt{T} into a \texttt{Vector\{T\}}. To give generality,
Julia offers various non-strict forms of types. The biggest example is
\texttt{Any}. Anything satisfies \texttt{T:\textless{}Any} (hence the
name). Therefore, if you need it, you can create a
\texttt{Vector\{Any\}}. For example:

    \begin{Verbatim}[commandchars=\\\{\}]
{\color{incolor}In [{\color{incolor}3}]:} a = Vector\PYZob{}Any\PYZcb{}(3)
        a[1] = 1.0
        a[2] = "hi!"
        a[3] = :Symbolic
        a
\end{Verbatim}

            \begin{Verbatim}[commandchars=\\\{\}]
{\color{outcolor}Out[{\color{outcolor}3}]:} 3-element Array\{Any,1\}:
         1.0       
          "hi!"    
          :Symbolic
\end{Verbatim}
        
    A less extreme form of an abstract type is a Union type, which is just
what it sounds like. For example:

    \begin{Verbatim}[commandchars=\\\{\}]
{\color{incolor}In [{\color{incolor}2}]:} a = Vector\PYZob{}Union\PYZob{}Float64,Int\PYZcb{}\PYZcb{}(3)
        a[1] = 1.0
        a[2] = 3
        a[3] = 1/4
        a
\end{Verbatim}

            \begin{Verbatim}[commandchars=\\\{\}]
{\color{outcolor}Out[{\color{outcolor}2}]:} 3-element Array\{Union\{Float64,Int64\},1\}:
         1.0 
         3   
         0.25
\end{Verbatim}
        
    This will only accept floating point numbers and integers. However, it
is still an abstract type. A function which is called on an abstract
type cannot know the type of any element (since, in this example, any
element can be either a float or an integer). Thus the optimization that
was achieved by multiple-dispatch, knowing the type each step of the
way, is no longer present. Therefore the optimizations are gone and
Julia will slow down to the speed of other scripting languages.

This leads to the performance principle: use strict typing whenever
possible. There are other advantages: a strictly typed
\texttt{Vector\{Float64\}} is actually byte-compatible with C/Fortran,
and so it can be used directly by C/Fortran programs without conversion.

    \section{Lunch Money}\label{lunch-money}

It's clear that Julia made clever design decisions in order to achieve
its performance goals while still being a scripting language. However,
what exactly is lost? Next I will show you a few pecularities of Julia
that come from this design decision, and the tools Julia gives you to
handle them.

    \subsection{Performance as Optional}\label{performance-as-optional}

One thing I already showed is that Julia gives many ways to achieve high
performance (like \texttt{@inbounds}), but they don't have to be used.
You can write type-unstable functions. It will be as slow as
MATLAB/R/Python, but you can do it. In places where you don't need the
best performance, it's nice to have this as an option.

    \subsection{Checking for
Type-Stability}\label{checking-for-type-stability}

Since type-stability is so essential, Julia gives you tools to check
that your functions are type stable. The most important is the
\texttt{@code\_warntype} macro. Let's use it to check a type-stable
function:

    \begin{Verbatim}[commandchars=\\\{\}]
{\color{incolor}In [{\color{incolor}4}]:} @code\PYZus{}warntype 2\PYZca{}5
\end{Verbatim}

    \begin{Verbatim}[commandchars=\\\{\}]
Variables:
  \#self\#::Base.\#\^{}
  x::Int64
  p::Int64

Body:
  begin 
      return \$(Expr(:invoke, LambdaInfo for power\_by\_squaring(::Int64, ::Int64), :(Base.power\_by\_squaring), :(x), :(p)))
  end::Int64

    \end{Verbatim}

    Notice that it shows all of the variables in the function as strictly
typed. What about in our \texttt{expo}?

    \begin{Verbatim}[commandchars=\\\{\}]
{\color{incolor}In [{\color{incolor}9}]:} @code\PYZus{}warntype expo(2,5)
\end{Verbatim}

    \begin{Verbatim}[commandchars=\\\{\}]
Variables:
  \#self\#::\#expo
  x@\_2::Int64
  y::Int64
  x@\_4::ANY

Body:
  begin 
      x@\_4::ANY = x@\_2::Int64
      unless (Base.slt\_int)(0,y::Int64)::Bool goto 5 \# line 3:
      return \$(Expr(:invoke, LambdaInfo for power\_by\_squaring(::Int64, ::Int64), :(Base.power\_by\_squaring), :(x@\_4::Int64), :(y)))
      5:  \# line 5:
      x@\_4::ANY = (Base.box)(Float64,(Base.sitofp)(Float64,x@\_4::Int64)) \# line 6:
      return (Base.Math.box)(Base.Math.Float64,(Base.Math.powi\_llvm)(x@\_4::Float64,(Base.box)(Int32,(Base.checked\_trunc\_sint)(Int32,y::Int64))))::Float64
  end::UNION\{FLOAT64,INT64\}

    \end{Verbatim}

    Notice that it has to make a temporary variable \texttt{x@\_4} which is
translates our int at the beginning of the function, and then do
type-checking in order to find the write function, and then its output
type is the non-strict \texttt{Union\{Float64,Int64\}}. The quick way to
read this is to see that \texttt{x@\_4::ANY} has a non-strict type,
indicating a type-instability. This gives you a tool to know how to
optimize.

    \subsection{Dealing With Necessary
Type-Instabilities}\label{dealing-with-necessary-type-instabilities}

For one, I already showed that some functions will error while in other
scripting languages they will ``read your mind''. In many cases you will
realize that you could've just used a different type from the start and
achieved type-stability (why not just do \texttt{2.0\^{}-5}?). However,
there are some cases where you won't find an appropriate type. This can
be easily fixed by conversions, though you then lose type stability.
Instead you have to think about your design and cleverly use multiple
dispatch.

So let's say that we have \texttt{a} as a
\texttt{Vector\{Union\{Float64,Int\}\}}. We may run into a case where we
have to use \texttt{a}. Assume that on each element of \texttt{a} we
have to perform a lot of operations. In this case, knowing the type of a
given element will lead to massive performance gains, but since it's in
a \texttt{Vector\{Union\{Float64,Int\}\}}, they will not be known in a
function like:

    \begin{Verbatim}[commandchars=\\\{\}]
{\color{incolor}In [{\color{incolor}10}]:} function foo\PYZob{}T,N\PYZcb{}(array::Array\PYZob{}T,N\PYZcb{})
           for i in eachindex(array)
             val = array[i]
             \PYZsh{} do algorithm X on val
           end
         end
\end{Verbatim}

            \begin{Verbatim}[commandchars=\\\{\}]
{\color{outcolor}Out[{\color{outcolor}10}]:} foo (generic function with 1 method)
\end{Verbatim}
        
    However, we can fix this with multiple dispatch. We can write a dispatch
on elements:

    \begin{Verbatim}[commandchars=\\\{\}]
{\color{incolor}In [{\color{incolor}11}]:} function inner\PYZus{}foo\PYZob{}T<:Number\PYZcb{}(val::T)
           \PYZsh{} Do algorithm X on val
         end
\end{Verbatim}

            \begin{Verbatim}[commandchars=\\\{\}]
{\color{outcolor}Out[{\color{outcolor}11}]:} inner\_foo (generic function with 1 method)
\end{Verbatim}
        
    and instead define foo as:

    \begin{Verbatim}[commandchars=\\\{\}]
{\color{incolor}In [{\color{incolor}12}]:} function foo2\PYZob{}T,N\PYZcb{}(array::Array\PYZob{}T,N\PYZcb{})
           for i in eachindex(array)
             inner\PYZus{}foo(array[i])
           end
         end
\end{Verbatim}

            \begin{Verbatim}[commandchars=\\\{\}]
{\color{outcolor}Out[{\color{outcolor}12}]:} foo2 (generic function with 1 method)
\end{Verbatim}
        
    Since types are checked for dispatch, the function \texttt{inner\_foo}
is strictly typed. Thus if \texttt{inner\_foo} is type-stable, then we
can achieve high performance by allowing it to specialize within
\texttt{inner\_foo}. This leads to a general design principle that, if
you're dealing with odd/non-strict types, you can use an outer function
to handle the type logic while using an inner function for all of the
hard calculations and achieve close to optimal performance while still
having the generic abilities of a scripting language.

    \subsection{REPL ``Globals'' Have Bad
Performance}\label{repl-globals-have-bad-performance}

    Globals in Julia have awful performance. Not using globals is
\href{http://docs.julialang.org/en/release-0.5/manual/performance-tips/}{the
first fact in the Performance Tips}. However, what newcommers don't
realize is that the REPL is the global scope. To see why, recall that
Julia has nested scopes. For example, if you have a function inside of a
function, then the inner function has all of the variables of the outer
function.

    \begin{Verbatim}[commandchars=\\\{\}]
{\color{incolor}In [{\color{incolor}13}]:} function test(x)
             y = x+2
             function test2()
                 y+3
             end
             test2()
         end
\end{Verbatim}

            \begin{Verbatim}[commandchars=\\\{\}]
{\color{outcolor}Out[{\color{outcolor}13}]:} test (generic function with 1 method)
\end{Verbatim}
        
    In \texttt{test2}, \texttt{y} is known because it is defined in
\texttt{test}. This will all work to give something performant if
\texttt{y} is type-stable since \texttt{test2} could then assume that
\texttt{y} is always an integer. But now look at what happens at the
highest scope (and thus effectively the global scope):

    \begin{Verbatim}[commandchars=\\\{\}]
{\color{incolor}In [{\color{incolor}15}]:} a = 3
         function badidea()
             a + 2
         end
         a = 3.0
\end{Verbatim}

    \begin{Verbatim}[commandchars=\\\{\}]
WARNING: Method definition badidea() in module Main at In[14]:3 overwritten at In[15]:3.

    \end{Verbatim}

            \begin{Verbatim}[commandchars=\\\{\}]
{\color{outcolor}Out[{\color{outcolor}15}]:} 3.0
\end{Verbatim}
        
    Because no dispatch is used to specialize \texttt{badidea}, and we can
change the type of \texttt{a} at any time, and therefore
\texttt{badidea} cannot add optimizations when compiling since the type
of \texttt{a} is unknown during compile time. However, Julia allows us
to specify variables as constant:

    \begin{Verbatim}[commandchars=\\\{\}]
{\color{incolor}In [{\color{incolor}16}]:} const a = 3
         function badidea()
             a + 2
         end
         a = 4 \PYZsh{} Works
         a = 3.0 \PYZsh{} Fails
\end{Verbatim}

    \begin{Verbatim}[commandchars=\\\{\}]

        LoadError: cannot declare a constant; it already has a value
    while loading In[16], in expression starting on line 1

        

    \end{Verbatim}

    Because \texttt{const} is about types, it acts slightly differently than
one would expect. \texttt{const} is a declaration that the type of
\texttt{a} will be constant, not the value. Therefore we can change
\texttt{a} from 3 to 4 since it goes from an \texttt{Int} to an
\texttt{Int}. However, trying to change \texttt{a} to \texttt{3.0} fails
because it cannot change to a Float64.

This will show up when trying to do benchmarks. The most common human
error to see is for newcommers to benchmark Julia like:

    \begin{Verbatim}[commandchars=\\\{\}]
{\color{incolor}In [{\color{incolor}24}]:} a = 3.0
         @time for i = 1:4
             a += i
         end
\end{Verbatim}

    \begin{Verbatim}[commandchars=\\\{\}]
  0.000005 seconds (4 allocations: 64 bytes)

    \end{Verbatim}

    However, if we put this in a function, it will optimize (in fact, it
will optimize away the loop and stick in the answer)

    \begin{Verbatim}[commandchars=\\\{\}]
{\color{incolor}In [{\color{incolor}28}]:} function timetest()
             a = 3.0
             @time for i = 1:4
                 a += i
             end
         end
         timetest() \PYZsh{} First time compiles
         timetest()
\end{Verbatim}

    \begin{Verbatim}[commandchars=\\\{\}]
  0.000000 seconds
  0.000000 seconds

    \end{Verbatim}

    \begin{Verbatim}[commandchars=\\\{\}]
WARNING: Method definition timetest() in module Main at In[27]:2 overwritten at In[28]:2.

    \end{Verbatim}

    This is a very easy problem to fall for: don't benchmark or time things
in the REPL's global scope. Always wrap things in a function or declare
them as const. There is a developer thread
\href{https://github.com/JuliaLang/julia/issues/8870}{to make the global
performance less awful} but, given the information from this notebook,
you can already see that it will never be ``not awful'', it will just be
``less awful''.

    \section{Conclusion}\label{conclusion}

Julia is fast by design. Type stability and multiple dispatch is
necessary to do the specialization that is involved in Julia's
compilation to make it work so well. The robust type system is required
to make working with types at such a fine level in order to effectively
achieve type-stability whenever possible, and manage optimizations when
it's not totally possible.

\subsection{Discussion / Project}\label{discussion-project}

Here's a good learning project: how would you design a new type
\texttt{EasyFloats} to build MATLAB/Python/R arithmetic into Julia? Time
the results and see what the difference from optimal is.


    % Add a bibliography block to the postdoc
    
    
    
    \end{document}
